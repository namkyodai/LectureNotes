%%%%%%%%%%%%%%%%%%%%% chapter.tex %%%%%%%%%%%%%%%%%%%%%%%%%%%%%%%%%
%
% sample chapter
%
% Use this file as a template for your own input.
%
%%%%%%%%%%%%%%%%%%%%%%%% Springer-Verlag %%%%%%%%%%%%%%%%%%%%%%%%%%
%\motto{Use the template \emph{chapter.tex} to style the various elements of your chapter content.}
\chapter{Level of services indicators - reliability }
\label{reliabilityc} % Always give a unique label
% use \chaptermark{}
% to alter or adjust the chapter heading in the running head
\chapterauthor{Bryan T. Adey and Nam Lethanh}
\section{Introduction}
In order to define an adequate LOS, to determine if an adequate LOS is provided,
and to determine optimal intervention strategies and work programs, it would be
best to focus on the maximizing the net benefit from the road infrastructure, as
defined in an all-inclusive impact hierarchy over a specific time period. Due to
the significant effort required to determine the values of some impact types, and
collecting these values over time, proxies are often used. Four commonly used
proxies are the reliability, availability, maintainability and safety of the
infrastructure. Of these, reliability is discussed in the remaining part of this
document. Availability and maintainability are discussed in the subsequent
document. Safety refers to the likelihood of persons being injured or killed due
to the infrastructure or use of the infrastructure. It is not discussed here
further. The definitions of the three performance indicators discussed are:

\begin{itemize}
	\item Reliability is the probability that an item to provide an adequate LOS.
	\item Availability is the proportion of time an item provides an adequate LOS.
	\item Maintainability is the ease with which an item can be maintained
\end{itemize}

The use of the word item in all definitions is to allow it to be replaced by any
term that refers to physical infrastructure that is composed of smaller parts.
For example, a network is comprised of objects, objects are comprised of
elements, and elements are comprised of segments. In such a case, the word item
in each of the three definitions could be replaced by either the word
``network'', the word ``object'', the word ``element'' or the word ``segment''.
For example, reliability is the ability of a network to provide an adequate level
of service, or reliability is the ability of an object to provide an adequate
level of service.
%
\section{Reliability of an item}
\subsection{General}
When managing infrastructure the reliability of an item is often a useful bit of
information, i.e. how likely is it that an item is going to provide an adequate
LOS for a specified period of time. If one speaks of the collapse of a bridge as
being the first level of inadequate service, one often refers to structural
reliability.

If the adequate LOS and the specified time period are defined, then one is
interested in knowing the reliability of the item. If the adequate LOS is defined
but not the specified time period then one is interested in knowing the amount of
time required until an item provides an inadequate LOS. This is, often referred
to as the service life of the item

It is important to realize that there is a tight connection between reliability,
the definition of adequate LOS, and the service life of an item. For example, two
identical items could have the same reliability, if one an adequate LOS is
defined to be high and the service life short, and the other one had an adequate
LOS defined to be low and the service life long.
\subsection{Reliability of an item over a specified time period}
The reliability of an item in the time interval from 0 to \textit{t} is
equivalent to the probability that the item provides an adequate LOS for a longer
time than \textit{t}, which can be expressed as:
\begin{eqnarray}
&& {R_i}\left( t \right) = P\left( {{\tau _i} > t} \right)
\label{eqreliability:1}
%1
\end{eqnarray}
Where:
\begin{adjustwidth}{1cm}{}
\begin{description}
\item[$\tau$:] is the amount of time that an item provides an adequate LOS,
\item[$i$:] is the item identifier
\end{description}
%\end{flushright}
\end{adjustwidth}
When item \textit{i} provides an adequate LOS for $\tau $ \textit{tus,} which is
a continuous positive random variable with c.d.f, ${\tau _i}$ represents the
amounts of time that the items provide adequate LOSs.

Mathematically,
\begin{eqnarray}
&& \begin{array}{l}
{X_i}\left( t \right) = 1\;\hspace{2mm}if \hspace{2mm}{\tau _i} > t\\
{X_i}\left( t \right) = 0\;\hspace{2mm}if\hspace{2mm}{\tau _i} \le t
\end{array}
\label{eqreliability:2}
%(2)
\end{eqnarray}
When multiple items are being investigated \textit{\textbf{X}(t) =
(X$_{1}$(t),\ldots{}X$_{n}$(t)) }is the item state vector at time \textit{t.}

In the estimation of reliability of items over multiple time periods \textit{t},
it is important to realize that it is the same items that survive each successive
interval. In other words, it is assumed that the items provide an adequate LOS at
t = 0, that there are no interventions executed on the items over the time period
investigated, and that the probability of an item providing an inadequate LOS at
\textit{t} = 2 depends on it providing an adequate level of service until
\textit{t} = 2, i.e. it is conditional on the item surviving until \textit{t} =
2.
\subsection{Conditional probability of failure and failure rate}
The probability of failure in the time interval \textit{(t, t+$\Delta{}$],}
given that the item provides an adequate LOS at \textit{t}, is the conditional
probability of failure, as shown in Equation \eqref{eqreliability:3}.
\begin{eqnarray}
&& P\left( t \le \tau \le t+\Delta \right|\tau > t) = \frac{P\left( \text{failure appears in } (t,t+\Delta) \right)}{P(\tau >t)}
\label{eqreliability:3}
%(2)
\end{eqnarray}
If the \textit{$\tau{}$} is a positive random variable with d.f. \textit{f(t)}
and c.d.f \textit{F(t)}, then for small delta the conditional failure probability
in \textit{(t, t+$\Delta{}$]} given the item is operational at \textit{t} is, for
small delta, approximately equal to the failure rate \textit{f(t) / (1-F(t)),
i.e.}
\begin{eqnarray}
&& P\left( t \le \tau \le t+\Delta \right|\tau > t) = \frac{f(t)\cdot \Delta}{1-F(t)}
\label{eqreliability:4}
%(2)
\end{eqnarray}
\begin{eqnarray}
&& {\rm{Prob}}(t \le \tau  \le t + \Delta t|\tau  \ge t) = \lambda (t)\Delta t =
\frac{{f(t)\Delta t}}{{\tilde F(t)}}
\label{eqreliability:5}
%(4)
\end{eqnarray}
Where:
\begin{adjustwidth}{1cm}{}
\begin{description}
\item[$f(t)$:] is the probability that an item will not provide an adequate LOS at any instance from 0 to $t$,
\item[$1-F(t)$:] is the probability that an item provides an adequate LOS from 0 to $t$. This is often referred to as the survival function. 
\end{description}
%\end{flushright}
\end{adjustwidth}
The failure rate can then be expressed as:
\begin{eqnarray}
&& \lambda (t) = \frac{{f(t)}}{{\tilde F(t)}}
%(4)
\end{eqnarray}
Note that \textit{$\lambda{}$(t)} is defined only for \textit{t} such that
\textit{F(t) $<$ 1}.
%
\subsection{Failure rates}
As can be seen from equation \eqref{eqreliability:5}, the failure rate is a time dependent, i.e. it
is not necessarily constant over time. For example, some engineering systems have
failure rates that exhibit bathtub curve characteristics (Figure \ref{reliability:1}).
It is common to divide these failure rates over time for these systems in three
phases. In the first phase is there is a monotonic decrease in the failure rate,
in the second phase the failure rate is more or less constant, and in the third
phase the failure rate monotonically increases.
\begin{figure}[h]
% \begin{center}
\includegraphics[width=326pt]{reliability-1.eps}
\caption{Bathtub curve}\label{reliability:1}
% \end{center}
\end{figure}
When the failure can be considered to be constant, it can be approximated by
assuming that the probability of failure can be modeled using an exponential
distribution function. In this case,
\begin{eqnarray}
&& \lambda (t) = \theta 
%(5)
\end{eqnarray}
where the probability density function and the cumulative distribution function
are:
\begin{eqnarray}
&& f(t) = \theta \exp ( - \theta  \cdot t) \\ 
&& F(t) = 1- \exp ( - \theta  \cdot t)
\end{eqnarray}
\subsection{Failure rate and rate of occurrence of failures}
Failure rate should not be confused with the rate of occurrence of failures.
Failure rate is defined only for a random variable describing the lifetime of a
non-renewable item. The rate of occurrence of failures is the derivative of the
mean number of failures \textit{[0,t]} with respect to \textit{t} of a point
process N\textit{(t), t $>$= 0,} describes failures of a renewable item, i.e.
\begin{eqnarray}
&& v(t)=\frac{E\left( N(t) \right)}{dt} \label{eqreliability:6}
\end{eqnarray}
\subsection{Reliability of an item with a time invariant failure rate}
When an item has a constant failure rate, the reliability of the item, which is
sometimes referred to as the survival probability of the item, is given as:
\begin{eqnarray}
&& R(t) = \tilde F(t) = 1 - F(t) = \exp ( - \theta  \cdot t)
\label{eqreliability:7}
%(9)
\end{eqnarray}
Example of how the reliability of the item changes over time for three different
constant failure rates is given in Figure \ref{reliability-2} and the corresponding
probability density functions are given in Figure \ref{reliability-3}.
%
\begin{figure}[h]
% \begin{center}
\includegraphics[width=386pt]{reliability-2.eps}
\caption{Reliability}\label{reliability-2}
% \end{center}
\end{figure}
\begin{figure}[h]
% \begin{center}
\includegraphics[width=394pt]{reliability-3.eps}
\caption{Probability density function}\label{reliability-3}
% \end{center}
\end{figure}
Interestingly the reliability of items with constant failure rates, regardless
of their values, is approximately 37\% at the mean life of the item. This can be
seen easily mathematically, since the length of time that an item provides an
adequate LOS if it has a constant failure rate can be calculated as:
\begin{eqnarray}
&& \Theta  = \int\limits_0^\infty  {\tilde F(t)dt}  = \int\limits_0^\infty  {\exp (
- \theta  \cdot t)dt}  = \frac{1}{\theta } \label{eqreliability:8}
%(10)
\end{eqnarray}
where $\theta $ is the failure rate and its reliability is then given by:
\begin{eqnarray}
&& R(t) = R(\frac{1}{\theta }) = \exp ( - \theta  \cdot \frac{1}{\theta }) = \exp (
- 1) = 0.3678794 \approx 0.37
\label{eqreliability:9}
%(11)
\end{eqnarray}

No matter what the value of the failure rate, the reliability at the mean
service life is 0.37, i.e. if the failure rate is 0.5, and, therefore, the mean
service life is 2 \textit{tus}, then the probability that the item provides an
adequate LOS for longer than 2 \textit{tus} is 0.37. This is illustrated
graphically in Figure \ref{reliability-4}, for items with failure rates of 0.5, 1, 1.5,
where the mean service lives are 2, 1 and 0.67.
%
\begin{figure}[h]
% \begin{center}
\includegraphics[width=346pt]{reliability-4.eps}
\caption{Probable service life and reliability}\label{reliability-4}
% \end{center}
\end{figure}
The codes for generating these graphs are given in Appendix \ref{appen31}.
\subsection{Estimating the remaining amount of time an item will provide an
adequate LOS}
In many cases when it is know that an item has provided an adequate LOS until
\textit{t} it is desired to know how much longer it might continue to provide an
adequate LOS. For example, if it known that an item provides an adequate LOS for
37 years and it has now been in service 37 years, what is the expected amount of
time that it will continue to provide an adequate LOS. How to calculate this was
explained in class.
\subsection{Reliability of an item with time-variant failure rate} \label{reliaweibl}
When an item has a time variant failure rate, the probability of failure cannot
be modeled using an exponential distribution. If it is modeled with the Weibull
function, a common selection, the reliability of the item can be estimated as:
\begin{eqnarray}
&& R(t) = {e^{ - \alpha{{( t)}^m}}}
\label{eqreliability:10}
%(12)
\end{eqnarray}
where $\alpha $ and m are scale and shape parameters, respectively.

In this case the probability density function is:
\begin{eqnarray}
&& f(t) = \alpha  \cdot m \cdot {t^{m - 1}} \cdot {e^{ - \alpha{{(t)}^m}}}
\label{eqreliability:11}
%(13)
\end{eqnarray}
And the failure rate
\begin{eqnarray}
&& \lambda (t) = \alpha  \cdot m \cdot {t^{m - 1}}
\label{eqreliability:12}
%(14)
\end{eqnarray}
Here, it can be see that if the value of $m=1$, this equation is identical to the
one used to estimate the constant failure rate.
\section{Estimating reliability when LOS data available}
The reliability of an item can be estimated in different ways. The way to be
used depends on the availability of information and the composition or structure
of the item (e.g. the item can be composed of many sub-items). If data is
available it can be used as described above to estimate the failure rate of an
item and then estimate the reliability of an item. An example is shown for a case
where the failure is time invariant and a case where the failure is time variant.
\subsection{Example 1: Reliability of an item with time invariant failure rate}
A railway tests 10 switches, over 4'180 days of use under the same loads. It was
observed that:
\begin{itemize}
	\item Switch 1 failed in 75 days
	\item Switch 2 failed in 125 days
	\item Switch 3 failed in 130 days
	\item Switch 4 failed in 325 days
	\item Switch 5 failed in 525 days
\end{itemize}
The failure rate is estimated as
\begin{eqnarray}
\theta  = \frac{5}{{4'180}} = 0.001196
\nonumber
%(15)
\end{eqnarray}
Assuming that this rate is constant, then the reliability of a switch of this
type is estimated as:
\begin{eqnarray}
R(t) = \exp ( - \theta  \cdot t) = \exp ( - 0.001196 \cdot t)
\nonumber
%(16)
\end{eqnarray}
Which shows, for example, that the reliability of a switch, i.e. the probability
that a switch will provide an adequate LOS for 1000 days, is 0.302.
\subsection{Example 2: Reliability of an item with time-variant failure rate}
The lights in street lamps frequently burn out. Regular inspections are done to
see if the lights are function or not. The data shown in Table \ref{tbl-reliability:1} has
been collected. In the table, the column ``time'' is the units of time that the
items have been in service. The column ``status'' refers to the state of the
lighting, 1 is ``burnt out'' and 0 is ``working''.
\begin{table}[h]
\caption{Failure data of the lighting system}
\vspace{3pt} \noindent
\begin{tabular}{|l|l|l|l|l|l|l|l|l|l|l|l|l|l|l|}
\hline
\multicolumn{1}{|c|}{Item} & \multicolumn{1}{c|}{time} & \multicolumn{1}{c|}{status} & \multicolumn{1}{c|}{Item} & \multicolumn{1}{c|}{time} & \multicolumn{1}{c|}{status} & \multicolumn{1}{c|}{Item} & \multicolumn{1}{c|}{time} & \multicolumn{1}{c|}{status} & \multicolumn{1}{c|}{Item} & \multicolumn{1}{c|}{time} & \multicolumn{1}{c|}{status} & \multicolumn{1}{c|}{Item} & \multicolumn{1}{c|}{time} & \multicolumn{1}{c|}{status} \\ 
\hline
\multicolumn{1}{|c|}{1} & \multicolumn{1}{c|}{9} & \multicolumn{1}{c|}{1} & \multicolumn{1}{c|}{6} & \multicolumn{1}{c|}{28} & \multicolumn{1}{c|}{0} & \multicolumn{1}{c|}{11} & \multicolumn{1}{c|}{161} & \multicolumn{1}{c|}{0} & \multicolumn{1}{c|}{16} & \multicolumn{1}{c|}{12} & \multicolumn{1}{c|}{1} & \multicolumn{1}{c|}{21} & \multicolumn{1}{c|}{33} & \multicolumn{1}{c|}{1} \\ 
\hline
\multicolumn{1}{|c|}{2} & \multicolumn{1}{c|}{13} & \multicolumn{1}{c|}{1} & \multicolumn{1}{c|}{7} & \multicolumn{1}{c|}{31} & \multicolumn{1}{c|}{1} & \multicolumn{1}{c|}{12} & \multicolumn{1}{c|}{5} & \multicolumn{1}{c|}{1} & \multicolumn{1}{c|}{17} & \multicolumn{1}{c|}{16} & \multicolumn{1}{c|}{0} & \multicolumn{1}{c|}{22} & \multicolumn{1}{c|}{43} & \multicolumn{1}{c|}{1} \\ 
\hline
\multicolumn{1}{|c|}{3} & \multicolumn{1}{c|}{13} & \multicolumn{1}{c|}{0} & \multicolumn{1}{c|}{8} & \multicolumn{1}{c|}{34} & \multicolumn{1}{c|}{1} & \multicolumn{1}{c|}{13} & \multicolumn{1}{c|}{5} & \multicolumn{1}{c|}{1} & \multicolumn{1}{c|}{18} & \multicolumn{1}{c|}{23} & \multicolumn{1}{c|}{1} & \multicolumn{1}{c|}{23} & \multicolumn{1}{c|}{45} & \multicolumn{1}{c|}{1} \\ 
\hline
\multicolumn{1}{|c|}{4} & \multicolumn{1}{c|}{18} & \multicolumn{1}{c|}{1} & \multicolumn{1}{c|}{9} & \multicolumn{1}{c|}{45} & \multicolumn{1}{c|}{0} & \multicolumn{1}{c|}{14} & \multicolumn{1}{c|}{8} & \multicolumn{1}{c|}{1} & \multicolumn{1}{c|}{19} & \multicolumn{1}{c|}{27} & \multicolumn{1}{c|}{1} & \multicolumn{1}{c|}{} & \multicolumn{1}{c|}{} & \multicolumn{1}{c|}{} \\ 
\hline
\multicolumn{1}{|c|}{5} & \multicolumn{1}{c|}{23} & \multicolumn{1}{c|}{1} & \multicolumn{1}{c|}{10} & \multicolumn{1}{c|}{48} & \multicolumn{1}{c|}{1} & \multicolumn{1}{c|}{15} & \multicolumn{1}{c|}{8} & \multicolumn{1}{c|}{1} & \multicolumn{1}{c|}{20} & \multicolumn{1}{c|}{30} & \multicolumn{1}{c|}{1} & \multicolumn{1}{c|}{} & \multicolumn{1}{c|}{} & \multicolumn{1}{c|}{} \\ 
\hline
\end{tabular}\vspace{2pt}
\label{tbl-reliability:1}\end{table}
Using the Weibull distribution as described in equation \eqref{eqreliability:11}, the value of
$\theta $ and m are 0.027 and 1.097, respectively. These values are estimated
using the Maximum likelihood estimation approach. The R code is given in Appendix
\ref{appen32}.

With these values, the failure rate is given by:
\begin{eqnarray}
&& \lambda (t) = 0.027 \times 1.097 \cdot {t^{1.097 - 1}}
\nonumber
%(17)
\end{eqnarray}
And the probability density function and the reliability function are therefore:
\begin{eqnarray}
&& f(t) = 0.027 \times 1.097 \cdot {t^{1.097 - 1}} \cdot {e^{ - {{(0.027 \cdot
t)}^{1.097}}}}
\nonumber
%(18)
\end{eqnarray}
\begin{eqnarray}
&& R(t) = {e^{ - {{(0.027 \cdot t)}^{1.097}}}}
\nonumber
%(19)
\end{eqnarray}
The evolution of reliability and failure rate over time are shown in Figure \ref{reliability-5}\pagebreak{}
\begin{figure}[h]
% \begin{center}
\includegraphics[width=436pt]{reliability-5.eps}
\caption{Reliability and failure rate - Weibull distribution
function}\label{reliability-5}
% \end{center}
\end{figure}
\section{Estimating reliability when LOS data not available}
There are many cases where the LOS data is not available, e.g. the failure of a
bridge. One way to estimate the reliability of the item in this case is to model
the uncertainty in the variables of the models used and compare the expected
resistance to the expected loads.
%
\subsection{Based on models and uncertainty of variables}
%
In this case a model of the performance of the infrastructure object is selected
and laboratory tests or simulations are done to determine probabilistic
distributions to represent the uncertainties in key parameters in the models. For
example, if one was to estimate the reliability of a bridge with respect to
landslides, one almost certainly does not have enough data to say with any
confidence the reliability of a bridge in the future, so one may embark upon
determining the probability that the bridge would continue to provide an adequate
LOS if it came in contact with specific volumes of rock. This is done by modeling
the performance of the bridge if it came in contact with rock falls of different
intensities, taking into consideration, for example, the strength of the
concrete, the strength of the reinforcement, the location of the reinforcement,
and how the elements in the bridge work together as a system. Such information is
often included in fragility curves. An example is given in  Figure \ref{reliability-6} and more can be found in  \cite{Schultz2010} and \cite{Lethanh2015}
%
\begin{figure}[h]
% \begin{center}
\includegraphics[width=418pt]{reliability-6.eps}
\caption{Example bridge deck fragility curves}\label{reliability-6}
% \end{center}
\end{figure}
The fragility curves shown in  Figure \ref{reliability-6} represent the probability (vertical axis) of
bridge deck failure, taking into consideration the initial condition of the
bridge deck, i.e. the manifest deterioration state of the bridge deck (\textit{i
}= 1, 2, and 3) prior to hazard occurrence given the volume of rock fall that
comes in contact with the deck. As can be seen from the figure, if 15 m$^{3}$ of
rock come in contact with a bridge deck in state 1, there is a 0.089 probability
(point A) that the bridge deck would fail. It can also be seen that the higher
the initial state the higher the probability of bridge deck failure. e.g. when
the bridge deck is initially in state 2 and state 3 the probability that the
bridge deck would fail is 0.372 and 0.716, respectively).

The probability of each failure state, given the intensity of the rock fall,
which are indicated with different curves in Figure \ref{reliability-6}, are given by
\begin{eqnarray}
&& \Omega_{s,i}^l=prob\left[CS>l|s,i\right] \label{eqreliability:13}
%(18)
\end{eqnarray}
where the probability of exceeding condition state is conditional on the intensity of the load effect event and the
condition state of the object, immediately prior to occurrence of the load effect event.

Of course, to estimate the reliability of the bridge, in general, the probability of occurrence of the load effect event itself has to be estimated, i.e.
\begin{eqnarray}
&& H_s(t)=Prob\left[\text{intensity} > s(0,t) \right] \label{eqreliability:14}
%(18)
\end{eqnarray}
\subsection{Based on reliability of sub-items}
When data does not exist on the LOS of the item itself, but it does on the
sub-items of which it is composed, the sub-items can be combined in ways to allow
the estimation of the reliability of the item. Four basic ways to combine things
are as sub-items in series, sub-items in parallel, parallel sub-items connected
in series, and series sub-items connected in parallel. These are explained in the
following four sub-sections.
\subsubsection{Sub-items in series}
If the sub-items are in series (e.g. Figure \ref{reliability-17}) and the LOS provided
by the item is equal to the LOS provided by the worst sub-item, then the LOS
provided by the item is given by:
\begin{eqnarray}
&& \Phi(\overrightarrow{x})=\prod_{i=1}^n x_i= \min\limits_{1 \leq i \leq n} x_i \label{eqreliability:15}
%(18)
\end{eqnarray}
where \textit{x$_{i}$} is the state of sub-item \textit{i}, \textit{i} =
\textit{1,\ldots{}n}, and it is represented as a binary variable, where 1
indicates the sub-item provides an adequate LOS and the 0 indicates that it does
not.

The reliability that the item will provide an adequate LOS is, therefore, given by:
\begin{eqnarray}
&& \vec R(t) = \prod\limits_{n = 1}^N {{R_n}(t)}
\label{eqreliability:17}
%(22)
\end{eqnarray}

\begin{figure}[h]
% \begin{center}
\includegraphics[width=410pt]{reliability-17.eps}
\caption{Sub-items in series}\label{reliability-17}
% \end{center}
\end{figure}
for example, a bridge can be considered to provide an adequate LOS if and only
if the abutment 1 and 2, and the deck provide adequate LOSs, and the probabilities
of each providing an inadequate LOS in one year are 0.01, 0.02 and 0.03,
respectively, and their failure rates are constant, then the reliabilities of the
bridge over 5 years are:

\begin{eqnarray}
&& \begin{array}{l}
\vec R(t = 5) = \prod\limits_{n = 1}^N {{R_n}(5)}  = {R_{abutment1}}(5) \cdot
{R_{deck}}(5) \cdot {R_{abutment2}}(5)\\
\hspace{20mm}= {e^{ - 0.02 \times 5}} \cdot {e^{ - 0.01 \times 5}} \cdot {e^{ - 0.03 \times
5}} = {e^{ - (0.02 + 0.01 + 0.03) \cdot 5}} \approx 0.74
\end{array}
%\label{eqreliability:18}
%(23)
\end{eqnarray}
\subsubsection{Sub-items in parallels}
If the sub-items are in parallel (e.g. Figure \ref{reliability-19}), and the LOS provided
by the item is equal to the LOS provided by the best sub-item, then the LOS
provided by the item is given by:
\begin{eqnarray}
&& \Phi(\overrightarrow{x})=1-\prod_{i=1}^n (1-x_i)= \max\limits_{1 \leq i \leq n} x_i \label{eqreliability:18}
%(18)
\end{eqnarray}
\begin{figure}[h]
% \begin{center}
\includegraphics[width=399pt]{reliability-19.eps}
\caption{Sub-items in parallel}\label{reliability-19}
% \end{center}
\end{figure}
The reliability that the item will provide an adequate LOS is, therefore, given by:
\begin{eqnarray}
&& \vec R(t) = 1 - \prod\limits_{i = 1}^N {\left( {1 - {R_n}(t)} \right)}
\label{eqreliability:19}
%(25)
\end{eqnarray}
\subsubsection{Parallel sub-items connected in series}
In some cases items are comprised of sub-items that can be thought of as being
groups of sub-items connected in parallel where the groups are then connected in
series (e.g. Figure \ref{reliability-21}). In this case the LOS provided by the item is
connected to the LOS provided by the sub-items as follows:
\begin{eqnarray}
&& \Phi (\vec x) = \left[ {1 - (1 - {x_1}) \cdot (1 - {x_2})} \right] \cdot \left[ {1 - (1
- {x_3}) \cdot (1 - {x_4})} \right]
\label{eqreliability:201}
%(27)
\end{eqnarray}
\begin{figure}[h]
% \begin{center}
\includegraphics[width=450pt]{reliability-21.eps}
\caption{Parallel sub-items connected in series}\label{reliability-21}
% \end{center}
\end{figure}
The reliability that the item will provide an adequate LOS is, therefore, given by:
\begin{eqnarray}
&& \vec R = \left[ {1 - (1 - {R_1}) \cdot (1 - {R_2})} \right] \cdot \left[ {1 - (1
- {R_3}) \cdot (1 - {R_4})} \right]
\label{eqreliability:20}
%(27)
\end{eqnarray}
\subsubsection{Series sub-items connected in parallel}
In some cases items are comprised of sub-items that can be thought of as being
groups of sub-items connected in series where the groups are then connected in
parallel (e.g. Figure \ref{reliability-23}). In this case the LOS provided by the item is
connected to the LOS provided by the sub-items as follows: 
%
%(29)
\begin{eqnarray}
&& \Phi (\vec x) = 1 - (1 - {x_1} \cdot {x_2}) \cdot (1 - {x_3} \cdot {x_4} \cdot {x_5})
\label{eqreliability:211}
\end{eqnarray}
\begin{figure}[h]
% \begin{center}
\includegraphics[width=450pt]{reliability-23.eps}
\caption{Series sub-items connected in parallel}\label{reliability-23}
% \end{center}
\end{figure}
The reliability that the item will provide an adequate LOS is, therefore, given by:
\begin{eqnarray}
&& \vec R = 1 - (1 - {R_1} \cdot {R_2}) \cdot (1 - {R_3} \cdot {R_4} \cdot {R_5})
\label{eqreliability:21}
\end{eqnarray}
\subsubsection{Complex configurations}
As not all items can be thought of as neat combinations of series and parallel
sub-items, it is useful to have a systematic way to determine the structure
function for the item. Three ways are pivotal decomposition, minimal paths and
minimal cuts.
\paragraph{Pivotal decomposition}
Pivotal decomposition is a method to determine the reliability of an item by
replacing unreliable sub-items with either 100\% reliable sub-items and/or 0\%
reliable (i.e. failed) sub-items until a configuration is created for which the
reliability is easily computable. The reliability of the item is then calculated
as:
\begin{eqnarray}
&& R\left( {\overrightarrow x } \right) = {R_i} \cdot R\left(
{{1_i},\overrightarrow x } \right) + \left( {1 - {R_i}} \right) \cdot R\left(
{{0_i},\overrightarrow x } \right)
\end{eqnarray}
Where \textit{ }$\left( {{\alpha _i},\overrightarrow x }
\right)$ represents the vector of sub-items  with $\overrightarrow x $ the sub-item
\textit{i} replaced by \textit{$\alpha{}$$_{i}$}. , which is either totally
reliable and has the value of 1 or failed and has the value of 0.
\paragraph{Minimal path and minimal cuts}
In the case of minimal cuts, it is necessary to consider
\begin{itemize}
	\item the vector of sub-item states a \textit{cut vector} if \textit{$\varphi{}$}(x) =
0,
	\item the specific sub-items in the cut vector that are required to not provide an
adequate LOS to mean that the item does not provide an adequate LOS, form the
\textit{cut set}, and
	\item a cut set in which there is no possibility to improve the LOS provided by a
sub-item without improving the LOS provided by the item is a \textit{minimal cut
set}.
\end{itemize}
In the case of minimal paths, it is necessary to consider
\begin{itemize}
	\item the vector of sub-item states a \textit{path vector} if \textit{$\varphi{}$}(x)
= 1
	\item the specific sub-items in the path vector that are required to provide an
adequate LIS to mean that the item provides an adequate LOS, form the
\textit{path set}, and
	\item a path set in which there is no possibility to reduce the LOS provided by a
sub-item without reducing the LOS provided by the item is a \textit{minimal path
set}.
\end{itemize}
\paragraph{Example}
Question

Define the structure function to be used to estimate the reliability of the item
composed of 5 sub-items as shown in Figure \ref{reliability-24}.
%
\begin{figure}[h]
% \begin{center}
\includegraphics[width=450pt]{reliability-24.eps}
\caption{Item with 5 sub-items in a complex structure}\label{reliability-24}
% \end{center}
\end{figure}

Answer

Using the pivotal decomposition, the item structure function can be determined
by substituting sub-item 3 with a sub-item with a reliability of 1 or 0, yielding
either equation \eqref{eqreliability:22} or \eqref{eqreliability:23},respectively. These are illustrated in Figure \ref{reliability-25} and Figure \ref{reliability-26}, respectively
\begin{eqnarray}
R({1_3},x) = \left[ {1 - (1 - {x_1})(1 - {x_2})} \right] \cdot \left[ {1 - (1 -
{x_4})(1 - {x_5})} \right]
\label{eqreliability:22}
%(30)
\end{eqnarray}
%
\begin{figure}[h]
% \begin{center}
\includegraphics[width=450pt]{reliability-25.eps}
\caption{Sub-item 3 provides an adequate LOS}\label{reliability-25}
% \end{center}
\end{figure}
%
\begin{eqnarray}
&& R({0_3},\vec x) = \left[ {1 - (1 - {x_1} \cdot {x_4})} \cdot (1 - {x_2} \cdot {x_5}) \right]
\label{eqreliability:23}
%(31)
\end{eqnarray}
\begin{figure}[h]
% \begin{center}
\includegraphics[width=450pt]{reliability-26.eps}
\caption{Sub-item 3 does not provide an adequate LOS}\label{reliability-26}
% \end{center}
\end{figure}
This means that the structure function of the item to be used to estimate the
reliability of the item is:
\begin{eqnarray}
&& R(\vec x) = {R_3} \cdot R({1_3},\vec x) + (1 - {R_3}) \cdot R({0_3},\vec x)
\label{eqreliability:24}
%(32)
\end{eqnarray}
\begin{eqnarray}
&& \begin{array}{l}
R(\vec x) = {R_1}{R_3}{R_4} + {R_1}{R_3}{R_5} + {R_2}{R_3}{R_4} +
{R_2}{R_3}{R_5} + {R_1}{R_2}{R_4}{R_5}\\
{\rm{          }} - {R_1}{R_2}{R_3}{R_4} - {R_1}{R_3}{R_4}{R_5} -
{R_1}{R_2}{R_3}{R_5} - {R_2}{R_3}{R_4}{R_5}
\end{array}
\label{eqreliability:25}
%(33)
\end{eqnarray}

Question

Show that the minimal path method will give the same answer as pivotal
decomposition.

Answer

The ``minimal path sets'' include
\begin{eqnarray}
&& \begin{array}{l}
{P_1} = (1,4)\\
{P_2} = (1,3,5)\\
{P_3} = (2,5)\\
{P_1} = (2,3,4)
\end{array} \nonumber
\end{eqnarray}
%
\begin{eqnarray}
&& \begin{array}{l}
\phi (\vec x) = 1 - \prod\limits_{s = 1}^S {\left( {1 - \prod\limits_{n \in
{P_s}} {{x_n}} } \right)} \\
{\rm{         = }}1 - (1 - {x_1}{x_4})(1 - {x_1}{x_3}{x_5})(1 - {x_2}{x_5})(1 -
{x_2}{x_3}{x_4})\\
{\rm{         = }}{x_1}{x_3}{x_4} + {x_1}{x_3}{x_5} + {x_2}{x_3}{x_4} +
{x_2}{x_3}{x_5} + {x_1}{x_2}{x_4}{x_5}\\
{\rm{          }} - {x_1}{x_2}{x_3}{x_4} - {x_1}{x_3}{x_4}{x_5} -
{x_1}{x_2}{x_3}{x_5} - {x_2}{x_3}{x_4}{x_5}
\end{array}
\label{eqreliability:26}
%(34)
\end{eqnarray}
\section{Conclusion}
Once adequate LOSs are defined, one is often interested in estimating the
likelihood that they will be provided. Without thinking of the specific
consequences that might occur, reliability can be used as a performance
indicator. Keep in mind though that the consequences related to the occurrence of
inadequate services are almost never constant, and, therefore, focusing
exclusively on reliability as a performance indicator will not allow the
prioriotization of activities to maximize net benefits related to infrastructure.
An example is included at the end of the next chapter.

When using reliability as a performance indicator for an item, there are
multiple ways to estimate the reliability. When using reliability as a
performance indicator for a sub-item one should not forget that the LOS it
provides depends on how it is integrated into the item as a whole and the
reliability of the other sub-items in the item. There are different ways to
measure this that vary in terms of accuracy and effort.
%
\section{Assignments}
\subsection{Problem A}
\subsubsection{Question A1}
Determine the reliability of the road network shown in Figure \ref{reliability-a-1}, when
the reliability of each object is as given in Table \ref{tbl-reliability-a:1}.
\begin{figure}[h]
% \begin{center}
\includegraphics[width=406pt]{reliability-a-1.eps}
\caption{A network of 7 objects}\label{reliability-a-1}
% \end{center}
\end{figure}
\begin{table}[h]
\caption{Object reliabilities}
% \vspace{3pt} \noindent
\begin{tabular}{|l|l|}
\hline
\multicolumn{1}{|c|}{Object} & \multicolumn{1}{c|}{Reliability} \\ 
\hline
\multicolumn{1}{|c|}{A} & \multicolumn{1}{c|}{0.95} \\ 
\hline
\multicolumn{1}{|c|}{B} & \multicolumn{1}{c|}{0.97} \\ 
\hline
\multicolumn{1}{|c|}{C} & \multicolumn{1}{c|}{0.92} \\ 
\hline
\multicolumn{1}{|c|}{D} & \multicolumn{1}{c|}{0.94} \\ 
\hline
\multicolumn{1}{|c|}{E} & \multicolumn{1}{c|}{0.90} \\ 
\hline
\multicolumn{1}{|c|}{F} & \multicolumn{1}{c|}{0.88} \\ 
\hline
\multicolumn{1}{|c|}{G} & \multicolumn{1}{c|}{0.98} \\ 
\hline
\end{tabular}
% \vspace{2pt}
\label{tbl-reliability-a:1}\end{table}
\subsubsection{Answer A1}
Seeing that the network can be broken down into series and parallel networks and
remembering that the structure function for series, and parallel objects are as
given in Equations \eqref{eqreliability:17} and \eqref{eqreliability:19}, respectively.
\begin{eqnarray}
&& {\vec R_C} = 1 - \left[ {1 - ({R_A} \cdot {R_B} \cdot {R_{CD}} \cdot {R_{EF}})}
\right] \cdot \left[ {1 - {R_G}} \right]
\label{eq-reliability-a3}
%(3)
\end{eqnarray}
where ${R_{CD}},{R_{EF}}$ represents the reliability of the parallel components
composed of item C, D and E, F, respectively. ${\vec R_{Network}}$ represents
reliability of the network configuration.
%
\begin{eqnarray}
&& \begin{array}{l}
{{\vec R}_{Network}} = 1 - \left[ {1 - ({R_A} \cdot {R_B} \cdot {R_{CD}} \cdot
{R_{EF}})} \right] \cdot \left[ {1 - {R_G}} \right]\\
{\rm{      =  }}1 - \left[ {1 - ({R_A} \cdot {R_B} \cdot \left\{ {(1 - (1 -
{R_C}) \cdot (1 - {R_D})} \right\} \cdot \left\{ {(1 - (1 - {R_E}) \cdot (1 -
{R_F})} \right\})} \right] \cdot \left[ {1 - {R_G}} \right]
\end{array}
\label{eq-reliability-a4}
%(4)
\end{eqnarray}
Using the value of reliability for each item shown in Figure \ref{fig:1},
following result is obtained
\begin{eqnarray}
&& {\vec R_{Network}} = 0.998121 \nonumber
%(3)
\end{eqnarray}
%
\subsection{Problem B}
Imagine that the infrastructure organization offers bid to external contractors
for construction and installation of its infrastructure network. The networks
consist of 17 objects (Table \ref{tbl-reliability-a:2}) and, minimum reliability of the network
reliability must be 70\%. The A, B and C bid 57'000 CHF, 39'000 CHF, and 42'000
CHF, respectively and proposed the configuration shown in Figure \ref{reliability-a-2},
Figure \ref{reliability-a-3}, and Figure \ref{reliability-a-3}.
%
\begin{figure}[h]
% \begin{center}
\includegraphics[width=408pt]{reliability-a-2.eps}
\caption{Configuration A}\label{reliability-a-2}
% \end{center}
\end{figure}
%
\begin{figure}[h]
% \begin{center}
\includegraphics[width=371pt]{reliability-a-3.eps}
\caption{Configuration B}\label{reliability-a-3}
% \end{center}
\end{figure}
%
\begin{figure}[h]
% \begin{center}
\includegraphics[width=415pt]{reliability-a-4.eps}
\caption{Configuration B}\label{reliability-a-4}
% \end{center}
\end{figure}
%
\begin{table}[h]
\caption{Object reliabilities}
\vspace{3pt} \noindent
\begin{tabular}{|p{28pt}|p{63pt}|p{28pt}|p{59pt}|p{34pt}|p{55pt}|}
\hline
\parbox{28pt}{\centering 
Object
} & \parbox{63pt}{\centering 
Reliability
} & \parbox{28pt}{\centering 
Object
} & \parbox{59pt}{\centering 
Reliability
} & \parbox{34pt}{\centering 
Object
} & \parbox{55pt}{\centering 
Reliability
} \\
\hline
\parbox{28pt}{\centering 
A
} & \parbox{63pt}{\centering 
0.840
} & \parbox{28pt}{\centering 
G
} & \parbox{59pt}{\centering 
0.870
} & \parbox{34pt}{\centering 
M
} & \parbox{55pt}{\centering 
0.830
} \\
\hline
\parbox{28pt}{\centering 
B
} & \parbox{63pt}{\centering 
0.860
} & \parbox{28pt}{\centering 
H
} & \parbox{59pt}{\centering 
0.880
} & \parbox{34pt}{\centering 
N
} & \parbox{55pt}{\centering 
0.850
} \\
\hline
\parbox{28pt}{\centering 
C
} & \parbox{63pt}{\centering 
0.890
} & \parbox{28pt}{\centering 
I
} & \parbox{59pt}{\centering 
0.890
} & \parbox{34pt}{\centering 
O
} & \parbox{55pt}{\centering 
0.840
} \\
\hline
\parbox{28pt}{\centering 
D
} & \parbox{63pt}{\centering 
0.860
} & \parbox{28pt}{\centering 
J
} & \parbox{59pt}{\centering 
0.860
} & \parbox{34pt}{\centering 
P
} & \parbox{55pt}{\centering 
0.890
} \\
\hline
\parbox{28pt}{\centering 
E
} & \parbox{63pt}{\centering 
0.870
} & \parbox{28pt}{\centering 
K
} & \parbox{59pt}{\centering 
0.850
} & \parbox{34pt}{\centering 
Q
} & \parbox{55pt}{\centering 
0.890
} \\
\hline
\parbox{28pt}{\centering 
F
} & \parbox{63pt}{\centering 
0.820
} & \parbox{28pt}{\centering 
L
} & \parbox{59pt}{\centering 
0.860
} & \parbox{34pt}{\centering } & \parbox{55pt}{\centering } \\
\hline
\end{tabular}
\vspace{2pt}
\label{tbl-reliability-a:2}\end{table}
\subsubsection{Question B1}
Given this information, determine the system reliability for each of the three
configurations.
\subsubsection{Answer B1}
Configurations A, B, and C are composed of both items connected in series and
parallel. Equations \eqref{eqreliability:17} and \eqref{eqreliability:19} are used.

One possible formula to be used to determine the reliability of configuration A,
B and C are:

\underline{Configuration A} 

\begin{itemize}
	\item B and C are in the same link (Eq. \eqref{eqreliability:17})
	\item D and E are in the same link (Eq. \eqref{eqreliability:17})
	\item \{B,C\} and \{D, E\} are in parallel (Eq. \eqref{eqreliability:19})
	\item G and H are in parallel (Eq. \eqref{eqreliability:19})
	\item Set 1 = [\{(B,C),(D,E)\}, F, (G,H), and I] are in the same link  (Eq. \eqref{eqreliability:17})
	\item K and L are in parallel (Eq. \eqref{eqreliability:19})
	\item O, P, and Q are in parallel (Eq. \eqref{eqreliability:19})
	\item Set 2 = [J, (K,L), M, N, and (O,P,Q)] are in the same link (Eq. \eqref{eqreliability:17})
	\item Set 1 and 2 are in parallel (Eq. \eqref{eqreliability:19})
\end{itemize}
\begin{eqnarray}
&& {\vec R_{configuration{\rm{ }}A}} = {R_A} \cdot {R_{B \to Q}}
\label{eq-reliability-a5}
%(5)
\end{eqnarray}
where  ${R_{B \to Q}}$ represents the the reliability of the sub-system
%
\begin{eqnarray}
&& {R_{B \to Q}} = 1 - (1 - {R_{B \to I}}) \cdot (1 - {R_{J \to Q}})
\label{eq-reliability-a6}
%(6)
\end{eqnarray}
%
\begin{eqnarray}
&& \begin{array}{l}
{R_{B \to I}} = {R_{B \to E}} \cdot {R_F} \cdot {R_{G \to H}} \cdot {R_I}\\
{\rm{          = }}\left[ {1 - (1 - {R_B} \cdot {R_C}) \cdot (1 - {R_D} \cdot
{R_E})} \right] \cdot {R_F} \cdot \left[ {1 - (1 - {R_G}) \cdot (1 - {R_H})}
\right] \cdot {R_I}
\end{array}
\label{eq-reliability-a7}
%(7)
\end{eqnarray}
%%
\begin{eqnarray}
&& \begin{array}{l}
{R_{J \to Q}} = {R_J} \cdot {R_{K \to L}} \cdot {R_M} \cdot {R_N} \cdot {R_{O
\to Q}}\\
{\rm{          = }}{R_J} \cdot \left[ {1 - (1 - {R_K}) \cdot (1 - {R_L})}
\right] \cdot {R_M} \cdot {R_N} \cdot \left[ {1 - (1 - {R_O}) \cdot (1 - {R_P})
\cdot (1 - {R_Q})} \right]
\end{array}
\label{eq-reliability-a8}
%(8)
\end{eqnarray}
Using the reliability values of items in Table \ref{tbl-reliability-a:2}, the reliability of
the configuration A is
\begin{eqnarray}
&& {\vec R_{configuration{\rm{ }}A}} = 0.729 \nonumber
\end{eqnarray}

\underline{Configuration B}

\begin{eqnarray}
{\vec R_{configuration{\rm{ }}B}} = {R_A} \cdot {R_B} \cdot {R_{C \to O}}
\label{eq-reliability-a9}
%(9)
\end{eqnarray}
%
\begin{eqnarray}
&& {R_{C \to O}} = 1 - (1 - {R_{C \to J}} \cdot {R_{K \to M}}) \cdot (1 - {R_{N \to
O}})
\label{eq-reliability-a10}\\
&& {R_{C \to J}} = 1 - (1 - {R_C} \cdot {R_F} \cdot {R_G})(1 - {R_E} \cdot {R_{H
\to J}})
\label{eq-reliability-a11}\\
&&{R_{H \to J}} = 1 - (1 - {R_H}) \cdot (1 - {R_I}) \cdot (1 - {R_J})
\label{eq-reliability-a12}\\
&& {R_{K \to M}} = 1 - (1 - {R_K}) \cdot (1 - {R_L}) \cdot (1 - {R_M})
\label{eq-reliability-a13}\\
%(13)
&& {R_{N \to O}} = {R_N} \cdot {R_O}
\label{eq-reliability-a14}
%(14)
\end{eqnarray}
%
%
Using the reliability values of items in Table \ref{tbl-reliability-a:2}, the reliability of
the configuration B is
%
\begin{eqnarray}
&& {\vec R_{configuration{\rm{ }}B}} = 0.714 \nonumber
\end{eqnarray}

\underline{Configuration C}
\begin{eqnarray}
&& {\vec R_{configuration{\rm{ C}}}} = {R_A} \cdot {R_{B \to K}} \cdot {R_L} \cdot
{R_{M \to N}}
\label{eq-reliability-a15}
%(15)
\end{eqnarray}
\begin{eqnarray}
&& {R_{B \to K}} = 1 - (1 - {R_{B \to F}} \cdot {R_{H \to J}}) \cdot (1 - {R_{G \to
K}})
\label{eq-reliability-a16}\\
%(16)
&& {R_{B \to F}} = 1 - (1 - {R_B} \cdot {R_C}) \cdot (1 - {R_D} \cdot {R_E}) \cdot
(1 - {R_F})
\label{eq-reliability-a17}\\
&& {R_{H \to J}} = 1 - (1 - {R_H}) \cdot (1 - {R_I}) \cdot (1 - {R_J})
\label{eq-reliability-a18}\\
%(18)
&& {R_{G \to K}} = {R_G} \cdot {R_K}
\label{eq-reliability-a19}
%(19)
\end{eqnarray}
Using the reliability values of items in Table \ref{tbl-reliability-a:2}, the reliability of
the configuration C is
\begin{eqnarray}
&& {\vec R_{configuration{\rm{ }}C}} = 0.702\nonumber
\end{eqnarray}

\underline{Result}

The reliability of configurations A, B and C are 0.729, 0.714, and 0.702
respectively. Based on reliability alone configuration A would be chosen.
%
\subsubsection{Question B2}
How high would the failure costs have to be before you would not choose the
network with the lowest upfront costs? (Use a one year time period and assume
that the network would not failure more than once).
%
\subsubsection{Answer B2}
If failure costs nothing than network configuration results in the lowest cost
(Table \ref{tbl-reliability-a:3}).


\begin{table}[h]
\caption{Total cost of networks}
\begin{tabular}{|l|l|l|l|l|}
\hline
\multicolumn{1}{|c|}{Failure cost} & \multicolumn{1}{c|}{Network} & \multicolumn{1}{c|}{Reliability} & \multicolumn{1}{c|}{Construction cost} & \multicolumn{1}{c|}{Total cost in one year} \\ 
\multicolumn{1}{|c|}{($10^3$ x CHF)} & \multicolumn{1}{c|}{} & \multicolumn{1}{c|}{} & \multicolumn{1}{c|}{($10^3$ x CHF)"} & \multicolumn{1}{c|}{($10^3$ x CHF)} \\ 
\hline
\multicolumn{1}{|c|}{} & \multicolumn{1}{c|}{A} & \multicolumn{1}{c|}{0.729} & \multicolumn{1}{c|}{57} & \multicolumn{1}{c|}{57} \\ 
\cline{2-5}
\multicolumn{1}{|c|}{0} & \multicolumn{1}{c|}{B} & \multicolumn{1}{c|}{0.714} & \multicolumn{1}{c|}{39} & \multicolumn{1}{c|}{39} \\ 
\cline{2-5}
\multicolumn{1}{|c|}{} & \multicolumn{1}{c|}{C} & \multicolumn{1}{c|}{0.702} & \multicolumn{1}{c|}{42} & \multicolumn{1}{c|}{42} \\ 
\hline
\multicolumn{1}{|c|}{} & \multicolumn{1}{c|}{A} & \multicolumn{1}{c|}{0.729} & \multicolumn{1}{c|}{57} & \multicolumn{1}{c|}{301} \\ 
\cline{2-5}
\multicolumn{1}{|c|}{900} & \multicolumn{1}{c|}{B} & \multicolumn{1}{c|}{0.714} & \multicolumn{1}{c|}{39} & \multicolumn{1}{c|}{296} \\ 
\cline{2-5}
\multicolumn{1}{|c|}{} & \multicolumn{1}{c|}{C} & \multicolumn{1}{c|}{0.702} & \multicolumn{1}{c|}{42} & \multicolumn{1}{c|}{310} \\ 
\hline
\multicolumn{1}{|c|}{} & \multicolumn{1}{c|}{A} & \multicolumn{1}{c|}{0.729} & \multicolumn{1}{c|}{57} & \multicolumn{1}{c|}{328} \\ 
\cline{2-5}
\multicolumn{1}{|c|}{1000} & \multicolumn{1}{c|}{B} & \multicolumn{1}{c|}{0.714} & \multicolumn{1}{c|}{39} & \multicolumn{1}{c|}{325} \\ 
\cline{2-5}
\multicolumn{1}{|c|}{} & \multicolumn{1}{c|}{C} & \multicolumn{1}{c|}{0.702} & \multicolumn{1}{c|}{42} & \multicolumn{1}{c|}{340} \\ 
\hline
\end{tabular}
\label{tbl-reliability-a:3}
\end{table}
\subsection{Problem C}
You have a series network with 5 infrastructure objects, two pavement sections
and three bridges. The probability of failure in each of the next 5 years of each
bridge is 0.1 (before and after repair) and the probability of failure for each
road section is 0.15 (before and after repair). Each bridge failure will take 60
days to repair and will cost 250'000 mus during which time the road will be
unusable in which users will incur 300'000 mus.  Each road section failure will
take 20 days to repair and will cost 75'000 mus during which time the road will
be unusable in which users will incur 100'000 mus. Failures are considered
mutually exclusive and that only one failure can occur per year.
\subsubsection{Question C1}
Calculate the reliability of the road over the 5 year-period.
\subsubsection{Answer C1}
The reliability is defined \eqref{eqreliability:7}. The hazard rate $\theta$ can be estimated.
%
\begin{eqnarray}
&& ln[R(t)] = \ln [\exp ( - \theta  \cdot t)] \Leftrightarrow \ln [R(t)] =  -
\theta .t
\label{eq-reliability-a21} \\
&& \to \theta  =  - \frac{{\ln [R(t)]}}{t}
\label{eq-reliability-a22}
%(21)
\end{eqnarray}
The reliability in one year of the network $R(t = 1)$is estimated as
\[
R(t = 1) = (1-0.1)^3 \cdot  (1-0.15)^2 = 0.5267025
\]
The join hazard rate will be 
\[
\to \theta  =  - \frac{{\ln [0.5267025]}}{1} = 0.6411194
%(24)
\]
The reliability in 5 year of the network will be
\[
R(t = 5) = \exp ( - 0.6411194 \times 5) = 0.0405347
%(25)
\]
\subsubsection{Question C2}
Calculate the maintainability of the road (in terms of time and money).
\subsubsection{Answer C2}
The formulation to estimate the maintainability is
\begin{eqnarray}
&& \bar M = \frac{{\sum\limits_{n = 1}^N {{p_n} \cdot N \cdot \mu _n^M}
}}{{\sum\limits_{n = 1}^N {{p_n} \cdot N} }}
\label{eq-reliability-a26}
%(26)
\end{eqnarray}
Where:
\begin{adjustwidth}{1cm}{}
\begin{description}
\item[${p_n}$:] is the failure probability of item type $n$,
\item[$N$:] is the number of item type $n$ 
\item[$\mu _n^M$:] is the mean time of impacts occur during the maintenance
\end{description}
%\end{flushright}
\end{adjustwidth}
Thus, in term of time, we have
\[
{\bar M^{time}} = \frac{{0.1 \times 3 \times 60}}{{0.1 \times 3 + 0.15 \times
2}} + \frac{{0.15 \times 2 \times 20}}{{0.1 \times 3 + 0.15 \times 2}} = 40{\rm{
days}}
\]
and in term of money, we have
\[
{\bar M^{money}} = \frac{{0.1 \times 3 \times (250'000 + 300'000)}}{{0.15 \times
2 + 0.1 \times 3}} + \frac{{0.15 \times 2 \times (75'000 + 100'000)}}{{0.15
\times 2 + 0.1 \times 3}} = 362'500{\rm{ mus}}
\]
\subsubsection{Question C3}
What is the impact on reliability and maintainability of repairing one bridge
perfectly, i.e failure will not occur?
\subsubsection{Answer C3}
When one bridge is perfectly repaired, its reliability is 1 and thus the network
can be considered to have only two bridges instead of 3. The same way of
calculation for a new configuration of the network with only 2 bridges and 2 road
sections is used to compute the reliability and maintainability.

The reliability in one year of the network $R(t = 1)$ is estimated as
\[
R(t = 1) = (1-0.1)^2 \times (1-0.15)^2 = 0.585225
\]
\[
\to \theta  =  - \frac{{\ln [0.585225]}}{1} = 0.5357589
\]
The reliability in 5 year of the network will be
\[
R(t = 5) = \exp ( - 0.5357589 \times 5) = 0.06864586
%(25)
\]
\[
{\bar M^{time}} = \frac{{0.15 \times 2 \times 60}}{{0.15 \times 2 + 0.1 \times
2}} + \frac{{0.1 \times 2 \times 20}}{{0.15 \times 2 + 0.1 \times 2}} = 36{\rm{
days}}
\]
and in terms of money, we have
\[
{\bar M^{money}} = \frac{{0.15 \times 2 \times (250'000 + 300'000)}}{{0.15
\times 2 + 0.1 \times 2}} + \frac{{0.1 \times 2 \times (75'000 + 100'000)}}{{0.15
\times 2 + 0.1 \times 2}} = 325'000{\rm{ mus}}
\]
\subsubsection{Question C4}
How much should we be willing to pay to execute such an intervention?
\subsubsection{Answer C4}
The amount of money that we are willing to pay depending on the expected impact
due to failure that can be generated under the two above options. Following table
describes the calculation of the amount of money that we are willing to pay. The
option 1 and 2 are corresponding to the case one bridge is not perfectly repaired
and another case when a bridge is perfectly repaired, respectively. The
reliability and the expect cost to repair are calculated from previous steps.
From the reliability, the failure probability can be straightforward estimated
and it is used to compute the possible cost (expected cost) if failure happens in
the future. As can be seen from the amount of possible cost under option 1 and 2,
option 2 yields less amount of cost than option 1, and the willingness to pay is
the difference of the twos, which is about 45.115 mus.

\begin{table}[h]
\caption{Comparison of two options}

\vspace{3pt} \noindent
\begin{tabular}{|p{38pt}|p{108pt}|p{51pt}|p{60pt}|}
\hline
\parbox{38pt}{\centering 
{\footnotesize Option}
} & \parbox{108pt}{\centering 
{\footnotesize Reliability}
} & \parbox{51pt}{\centering 
{\footnotesize Cost (mus)}
} & \parbox{60pt}{\centering 
{\footnotesize Failure probability}
} \\
\hline
\parbox{38pt}{\centering 
{\footnotesize 1}
} & \parbox{108pt}{\centering 
{\footnotesize 0.0405347}
} & \parbox{51pt}{\centering 
{\footnotesize 362'500}
} & \parbox{60pt}{\centering 
{\footnotesize 0.9594653}
} \\
\hline
\parbox{38pt}{\centering 
{\footnotesize 2}
} & \parbox{108pt}{\centering 
{\footnotesize 0.06864586}
} & \parbox{51pt}{\centering 
{\footnotesize 325'000}
} & \parbox{60pt}{\centering 
{\footnotesize 0.93135414}
}  \\
\hline
\parbox{38pt}{\centering } & \multicolumn{3}{|c|}{\parbox{219pt}{\centering 
{\footnotesize Willingness to pay  = 45.115 mus}
}} \\
\hline
\end{tabular}
\vspace{2pt}
\end{table}
\subsubsection{Question C5}
Discuss the assumptions mutual exclusiveness and that only one failure can occur
per year.
\subsubsection{Answer C5}
The mutual exclusiveness assumed in this example means that only one failure can
occur per year. This assumption can be used but it also bears a limitation that
it might not perfectly reflect the increasing failure rate over time as the
bridges and road sections are aging year by year. In addition, as the bridges and
road sections are independent, they can actually failure to provide adequate
level of services simultaneously in the same year. This possibility should be
taken into consideration of calculating the reliability and maintainability in
reality.
\subsection{Problem D}
A network is connected and represented as shown in the following Figure
\ref{reliability-a-5}. All 6 objects are identical. In one month, on average it is recorded
that 1 out of 6 objects fails and is renewed.

\begin{figure}[h]
% \begin{center}
\includegraphics[width=354pt]{reliability-a-5.eps}
\caption{Network}\label{reliability-a-5}
% \end{center}
\end{figure}
\subsubsection{Question D1}
Calculate the reliability of the network in 5 months, i.e. that it is possible
to send one item from the start node (S) to the end note (E)? Note: you can round
up the values in 2 digits.
\subsubsection{Answer D1}
One month failure probability for each object is 1/6. The reliability for the
network in one month is defined as
\begin{eqnarray}
&& \vec R_{network}^1 = R_1^1 \cdot R_{2 \to 5}^1 \cdot R_6^1
\label{eq-reliability-a30} \\
&& R_{2 \to 5}^1 = 1 - (1 - R_2^1) \cdot \left[ {\left\{ {1 - (1 - R_3^1) \cdot (1
- R_4^1)} \right\}} \right] \cdot (1 - R_5^1)
\label{eq-reliability-a31} \\
&& \vec R_{network}^1 \approx 0.694 \nonumber
%(30)
\end{eqnarray}
The reliability in 5 months of the network will be estimated based on the
failure rate of individual item
\[
\to \theta  =  - \frac{{\ln [1 - 1/6]}}{1} \approx 0.182
%(32)
\]
The reliability of each item in 5 months is
\[
R_{item}^5 = \exp ( - 0.183 \times 5) \approx 0.402
\]
Using equations \eqref{eq-reliability-a30} and \eqref{eq-reliability-a31} for a period of 5 months, we have
\begin{eqnarray}
&& \vec R_{network}^5 = R_1^5 \cdot R_{2 \to 5}^5 \cdot R_6^5
\label{eq-reliability-a34} \\
&& R_{2 \to 5}^5 = 1 - (1 - R_2^5) \cdot \left[ {\left\{ {1 - (1 - R_3^5) \cdot (1
- R_4^5)} \right\}} \right] \cdot (1 - R_5^5)
\label{eq-reliability-a35}\\
&& \vec R_{network}^5 \approx 0.141 \nonumber
%(34)
\end{eqnarray}
\subsubsection{Question D2}
If the network is not working, a loss of 30 \textit{mus} is expected. In order
to improve the overall reliability of the network over the next 5 months, manager
proposes 2 options

\textbf{Option 1:} Replace link 1 and link 6 with links that are 5 \% more
reliable than the existing ones.

\textbf{Option 2:} Replace links 2, 3, 4, and 5 with the links that are 8\% more
reliable than the existing ones.

The differences in replacement costs are negligible and thus not considered.

Which option shall be chosen?
\subsubsection{Answer D2}
\textbf{Option 1:} Replace link 1 and link 6 with links that are 5 \% more
reliable than the existing ones.

Under this option, new reliability of link 1 and link 6 become
\[
R_1^1 = R_6^1 = (1 - \frac{1}{6}) + (1 - \frac{1}{6}) \cdot \frac{5}{{100}} =
0.875
\]

Using the new value of reliability for item 1 and 6 and the values of
reliability for item 2, 3, 4, and 5 ($R_2^1 = R_3^1 = R_4^1 = R_5^1 = (1 -
\frac{1}{6})$) for the equations (30) and (31), we have

\[
 \vec R_{network}^{1,option1} \approx 0.765
\]

\textbf{Option 2:} Replace links 2, 3, 4, and 5 with the links that are 8\% more
reliable than the existing ones.

Under this option, new reliability of link 1 and link 6 become

\[
 R_2^1 = R_3^1 = R_4^1 = R_5^1 = (1 - \frac{1}{6}) + (1 - \frac{1}{6}) \cdot
\frac{8}{{100}} = 0.9
\]

Using the new value of reliability for item 2, 3, 4 and 5 and the old values of
reliability for item 1, and 6 ($R_1^1 = R_6^1 = (1 - \frac{1}{6})$) for the
equations (30) and (31), we have

\[
 \vec R_{network}^{1,option1} \approx 0.694
\]

Similarly, for 5 months, we have

\[
 R_1^5 = R_6^5 \approx 0.402 + 0.402 \cdot \frac{5}{{100}} = 0.4221
\]

Using the new value of reliability for item 1 and 6 and the values of
reliability for item 2, 3, 4, and 5 ($R_2^5 = R_3^5 = R_4^5 = R_5^5 = 0.402$) for
the equations (30) and (31), we have

\[
 \vec R_{network}^{5,option1} \approx 0.155
\]

\textbf{Option 2:} Replace links 2, 3, 4, and 5 with the links that are 8\% more
reliable than the existing ones.

Under this option, new reliability of link 1 and link 6 become

\[
 R_2^2 = R_3^3 = R_4^4 = R_5^5 = 0.402 + 0.402 \cdot \frac{8}{{100}} = 0.434
\]

Using the new value of reliability for item 2, 3, 4 and 5 and the old values of
reliability for item 1, and 6 ($R_1^5 = R_6^5 = 0.402$) for the equations (30)
and (31), we have

\[
 \vec R_{network}^{5,option2} \approx 0.145
\]

Following table compares the expected impact under the two options
\begin{table}
\caption{Comparison of two options}
\label{tbl-reliability-a:4}
\begin{tabular}{|l|l|l|l|l|l|}
\hline
\multicolumn{1}{|c|}{Time} & \multicolumn{1}{c|}{Options} & \multicolumn{1}{c|}{Reliability} & \multicolumn{1}{c|}{Failure} & \multicolumn{1}{c|}{Unit impact} & \multicolumn{1}{c|}{Total impact} \\ 
\multicolumn{1}{|c|}{(months)} & \multicolumn{1}{c|}{} & \multicolumn{1}{c|}{} & \multicolumn{1}{c|}{probability} & \multicolumn{1}{c|}{(mus)} & \multicolumn{1}{c|}{(mus)} \\ 
\hline
\multicolumn{1}{|c|}{1} & \multicolumn{1}{c|}{1} & \multicolumn{1}{c|}{0.765} & \multicolumn{1}{c|}{0.235} & \multicolumn{1}{c|}{30} & \multicolumn{1}{c|}{7.05} \\ 
\cline{2-6}
\multicolumn{1}{|c|}{} & \multicolumn{1}{c|}{2} & \multicolumn{1}{c|}{0.694} & \multicolumn{1}{c|}{0.306} & \multicolumn{1}{c|}{30} & \multicolumn{1}{c|}{9.18} \\ 
\hline
\multicolumn{1}{|c|}{5} & \multicolumn{1}{c|}{1} & \multicolumn{1}{c|}{0.155} & \multicolumn{1}{c|}{0.845} & \multicolumn{1}{c|}{30} & \multicolumn{1}{c|}{25.35} \\ 
\cline{2-6}
\multicolumn{1}{|c|}{} & \multicolumn{1}{c|}{2} & \multicolumn{1}{c|}{0.145} & \multicolumn{1}{c|}{0.855} & \multicolumn{1}{c|}{30} & \multicolumn{1}{c|}{25.65} \\ 
\hline
\end{tabular}
\end{table}
As can be seen from the table, under both 1 month and 5 months the expected
total impacts incurred by option 1 is smaller than that of option 2, thus, option
1 is preferable than option 2.
% \begin{enumerate}
\subsection{Problem E}
% \end{enumerate}

You are evaluating two offers for replacement of an existing bridge on a
highway. You have determined that the link to which the existing bridge belongs
has a reliability of 0.9 over the time period of interest, if it is assumed that
the bridge is perfectly reliable. The consequences of not being able to use the
link due to failure of the bridge are 100 million CHF, principally due to lost
travel time and economic impact.

\begin{itemize}
	\item Offer A, for 15 million CHF, is a wide bridge that can accommodate two lanes of
traffic in both directions and has reliability over the time period of interest
of 0.95.
	\item Offer B, for 19 million CHF, is for two narrow bridges that can accommodate two
lanes of traffic each, and each have a reliability of 0.95 over the same time
period.
\end{itemize}
% \begin{enumerate}
\subsubsection{Question E1}
% \end{enumerate}


If it is assumed that the only differences between the two offers in terms of
costs and benefits are the cost of construction and the cost of not being able to
use the link if failure occurs,

which offer should be chosen?

Back up your argument with exact numbers. In words explain how your decision
would be affected if you discovered that you over-estimated the reliability of
the rest of the link.

% \begin{enumerate}
\subsubsection{Answer E1}
% \end{enumerate}

Option B without the link has a higher reliability than the option A without the
link. Therefore if the reliability of the link is lower than assumed, the failure
cost of both options will be increasing accordingly. Therefore, option B will
remain as the best option. The calculations can be seen in Table \ref{tbl-reliability-a:4a}.
\begin{table}
\caption{Calculations}
\label{tbl-reliability-a:4a}
{\tabcolsep2pt
\begin{tabular}{|l|l|l|l|l|l|l|l|l|l|l|}
\hline
\multicolumn{1}{|P{90}{2.2cm}|}{Offers} & \multicolumn{1}{P{90}{2.2cm}|}{Cost} & \multicolumn{1}{P{90}{2.2cm}|}{Object reliability} & \multicolumn{1}{P{90}{2.2cm}|}{Link reliability w/o the object} & \multicolumn{1}{P{90}{2.0cm}|}{Link reliability w object} & \multicolumn{1}{P{90}{2.2cm}|}{Link failure probability} & \multicolumn{1}{P{90}{2.2cm}|}{Costs due to not being able to use the link if failure occurs} & \multicolumn{1}{P{90}{2.2cm}|}{Expected costs due to not being able to use the link} & \multicolumn{1}{P{90}{1.6cm}|}{Differential cost} & \multicolumn{1}{P{90}{2.2cm}|}{Differential benefit} & \multicolumn{1}{P{90}{2.2cm}|}{ Net benefit} \\ 
\hline
\multicolumn{1}{|c|}{A} & \multicolumn{1}{c|}{15} & \multicolumn{1}{c|}{0.95} & \multicolumn{1}{c|}{0.9} & \multicolumn{1}{c|}{0.855} & \multicolumn{1}{c|}{0.145} & \multicolumn{1}{c|}{100} & \multicolumn{1}{c|}{14.500} & \multicolumn{1}{c|}{0} & \multicolumn{1}{c|}{0} & \multicolumn{1}{c|}{} \\ 
\hline
\multicolumn{1}{|c|}{B} & \multicolumn{1}{c|}{19} & \multicolumn{1}{c|}{0.95} & \multicolumn{1}{c|}{0.9} & \multicolumn{1}{c|}{} & \multicolumn{1}{c|}{} & \multicolumn{1}{c|}{100} & \multicolumn{1}{c|}{10.225} & \multicolumn{1}{c|}{4} & \multicolumn{1}{c|}{4.275} & \multicolumn{1}{c|}{0.275} \\ 
\cline{3-3}
\multicolumn{1}{|c|}{} & \multicolumn{1}{c|}{} & \multicolumn{1}{c|}{0.95} & \multicolumn{1}{c|}{} & \multicolumn{1}{c|}{} & \multicolumn{1}{c|}{} & \multicolumn{1}{c|}{} & \multicolumn{1}{c|}{} & \multicolumn{1}{c|}{} & \multicolumn{1}{c|}{} & \multicolumn{1}{c|}{} \\ 
\hline
\end{tabular}
}
\end{table}
\subsection{Problem F}
Water is one of the fundamental needs of millions of people living in a
megacity. Water of sufficient quality and quantity must be provided around the
clock. In order to fulfill this need, a city depends on its water distribution
infrastructure, which includes pipes made of different materials and laid at
different times. These pipes are affected by processes of different types and
deteriorate at different rates. The consequences related to pipe failure vary
significantly depending on the type of failure, e.g. a pipe break, or a leaking
pipe, as does the reaction time required to fix the pipe. For example, if a pipe
breaks, a corrective intervention must be executed immediately, and if it is
noticed that there is progressive water loss over time then a preventive
intervention can be planned before there is an inadequate level of service.

Part of the water distribution network in mega-city Q is shown in Figure
\ref{fig24}. The pipe characteristics are given in Table \ref{tbl-reliability-a:5}.
% \begin{figure}[h]
% % \begin{center}
% \includegraphics[width=292pt]{reliability-a-6.eps}
% \caption{Simplified network of water supply pipes}\label{reliability-a-6}
% % \end{center}
% \end{figure}

\begin{table}
\caption{Pipe and their attributes}
\label{tbl-reliability-a:5}
{\tabcolsep2pt
\begin{tabular}{|l|l|l|l|l|l|l|}
\hline
\multicolumn{1}{|c|}{Pipe} & \multicolumn{1}{c|}{Material} & \multicolumn{1}{c|}{Demand} & \multicolumn{1}{c|}{Length} & \multicolumn{1}{c|}{Year of construction} & \multicolumn{1}{c|}{Usage} & \multicolumn{1}{c|}{Distribution of PS in year 2014} \\ 
\multicolumn{1}{|c|}{} & \multicolumn{1}{c|}{} & \multicolumn{1}{c|}{($m^3/day$} & \multicolumn{1}{c|}{($m$)} & \multicolumn{1}{c|}{} & \multicolumn{1}{c|}{} & \multicolumn{1}{c|}{} \\ 
\hline
\multicolumn{1}{|c|}{M} & \multicolumn{1}{c|}{Concrete} & \multicolumn{1}{c|}{} & \multicolumn{1}{c|}{5'000} & \multicolumn{1}{c|}{1998} & \multicolumn{1}{c|}{Distribution only} & \multicolumn{1}{c|}{[0.9; 0.1; 0; 0; 0]} \\ 
\hline
\multicolumn{1}{|c|}{A.1} & \multicolumn{1}{c|}{PVC} & \multicolumn{1}{c|}{} & \multicolumn{1}{c|}{600} & \multicolumn{1}{c|}{2003} & \multicolumn{1}{c|}{Distribution only} & \multicolumn{1}{c|}{} \\ 
\cline{1-2}\cline{4-6}
\multicolumn{1}{|c|}{A.2} & \multicolumn{1}{c|}{PVC} & \multicolumn{1}{c|}{} & \multicolumn{1}{c|}{600} & \multicolumn{1}{c|}{2003} & \multicolumn{1}{c|}{Distribution only} & \multicolumn{1}{c|}{} \\ 
\cline{1-2}\cline{4-6}
\multicolumn{1}{|c|}{A.3} & \multicolumn{1}{c|}{PVC} & \multicolumn{1}{c|}{250'000} & \multicolumn{1}{c|}{600} & \multicolumn{1}{c|}{2003} & \multicolumn{1}{c|}{Distribution and connection to buildings} & \multicolumn{1}{c|}{[0.85; 0.1; 0.05; 0; 0]} \\ 
\cline{1-2}\cline{4-6}
\multicolumn{1}{|c|}{A.4} & \multicolumn{1}{c|}{PVC} & \multicolumn{1}{c|}{} & \multicolumn{1}{c|}{600} & \multicolumn{1}{c|}{2003} & \multicolumn{1}{c|}{Distribution only} & \multicolumn{1}{c|}{} \\ 
\cline{1-2}\cline{4-6}
\multicolumn{1}{|c|}{A.5} & \multicolumn{1}{c|}{PVC} & \multicolumn{1}{c|}{} & \multicolumn{1}{c|}{600} & \multicolumn{1}{c|}{2003} & \multicolumn{1}{c|}{Distribution and connection to buildings} & \multicolumn{1}{c|}{} \\ 
\hline
\multicolumn{1}{|c|}{B} & \multicolumn{1}{c|}{Cast iron type 1} & \multicolumn{1}{c|}{200'000} & \multicolumn{1}{c|}{2'500} & \multicolumn{1}{c|}{1993} & \multicolumn{1}{c|}{Distribution and connection to buildings} & \multicolumn{1}{c|}{[0.8; 0.2; 0; 0; 0]} \\ 
\hline
\multicolumn{1}{|c|}{C} & \multicolumn{1}{c|}{Cast iron type 2} & \multicolumn{1}{c|}{100'000} & \multicolumn{1}{c|}{1'600} & \multicolumn{1}{c|}{1993} & \multicolumn{1}{c|}{Distribution and connection to buildings} & \multicolumn{1}{c|}{[0.83; 0.1; 0.07; 0; 0]} \\ 
\hline
\multicolumn{1}{|c|}{D} & \multicolumn{1}{c|}{Cast iron type 3} & \multicolumn{1}{c|}{350'000} & \multicolumn{1}{c|}{4'000} & \multicolumn{1}{c|}{1983} & \multicolumn{1}{c|}{Distribution and connection to buildings} & \multicolumn{1}{c|}{[0.8; 0.1; 0.1; 0; 0]} \\ 
\hline
\end{tabular}
}
\end{table}

When newly built the pipes to not leak but over time they do. The amount of
water lost due to these leaks increases over time. In order to model the increase
in leakage over time the pipes the performance states in Table \ref{tbl-reliability-a:6} are
used. The one year transition probabilities for pipe M and A are given in Table
\ref{tbl-reliability-a:7}.

\begin{table}%[h]
\caption{Performance states used to model water leakage over time}

\vspace{3pt} \noindent
\begin{tabular}{|p{95pt}|p{111pt}|}
\hline
\parbox{95pt}{\centering 
Performance

state
} & \parbox{111pt}{\centering 
Water loss
\\
(\%)
} \\
\hline
\parbox{95pt}{\centering 
1
} & \parbox{111pt}{\centering 
W $<$ 5\%
} \\
\hline
\parbox{95pt}{\centering 
2
} & \parbox{111pt}{\centering 
5\% $<$= W $<$ 10\%
} \\
\hline
\parbox{95pt}{\centering 
3
} & \parbox{111pt}{\centering 
10\% $<$= W $<$ 15\%
} \\
\hline
\parbox{95pt}{\centering 
4
} & \parbox{111pt}{\centering 
15\% $<$= W $<$ 20\%
} \\
\hline
\parbox{95pt}{\centering 
5
} & \parbox{111pt}{\centering 
20\% $<$= W
} \\
\hline
\end{tabular}
\vspace{2pt}
\label{tbl-reliability-a:6}\end{table}

\begin{table}[h]
\caption{One year transition probabilities for pipe M and A}
\label{tbl-reliability-a:7}
{\tabcolsep2pt
\begin{tabular}{|l|l|l|l|l|l|l|l|l|l|l|}
\hline
\multicolumn{1}{|c|}{Pipe} & \multicolumn{5}{c|}{M} & \multicolumn{5}{c|}{A} \\ 
\cline{2-11}
\multicolumn{1}{|c|}{PS} & \multicolumn{1}{c|}{1} & \multicolumn{1}{c|}{2} & \multicolumn{1}{c|}{3} & \multicolumn{1}{c|}{4} & \multicolumn{1}{c|}{5} & \multicolumn{1}{c|}{1} & \multicolumn{1}{c|}{2} & \multicolumn{1}{c|}{3} & \multicolumn{1}{c|}{4} & \multicolumn{1}{c|}{5} \\ 
\hline
\multicolumn{1}{|c|}{1} & \multicolumn{1}{c|}{0.93} & \multicolumn{1}{c|}{0.07} & \multicolumn{1}{c|}{0} & \multicolumn{1}{c|}{0} & \multicolumn{1}{c|}{0} & \multicolumn{1}{c|}{0.91} & \multicolumn{1}{c|}{0.09} & \multicolumn{1}{c|}{0} & \multicolumn{1}{c|}{0} & \multicolumn{1}{c|}{0} \\ 
\hline
\multicolumn{1}{|c|}{2} & \multicolumn{1}{c|}{0} & \multicolumn{1}{c|}{0.87} & \multicolumn{1}{c|}{0.13} & \multicolumn{1}{c|}{0} & \multicolumn{1}{c|}{0} & \multicolumn{1}{c|}{0} & \multicolumn{1}{c|}{0.88} & \multicolumn{1}{c|}{0.12} & \multicolumn{1}{c|}{0} & \multicolumn{1}{c|}{0} \\ 
\hline
\multicolumn{1}{|c|}{3} & \multicolumn{1}{c|}{0} & \multicolumn{1}{c|}{0} & \multicolumn{1}{c|}{0.85} & \multicolumn{1}{c|}{0.15} & \multicolumn{1}{c|}{0} & \multicolumn{1}{c|}{0} & \multicolumn{1}{c|}{0} & \multicolumn{1}{c|}{0.85} & \multicolumn{1}{c|}{0.15} & \multicolumn{1}{c|}{0} \\ 
\hline
\multicolumn{1}{|c|}{4} & \multicolumn{1}{c|}{0} & \multicolumn{1}{c|}{0} & \multicolumn{1}{c|}{0} & \multicolumn{1}{c|}{0.8} & \multicolumn{1}{c|}{0.2} & \multicolumn{1}{c|}{0} & \multicolumn{1}{c|}{0} & \multicolumn{1}{c|}{0} & \multicolumn{1}{c|}{0.82} & \multicolumn{1}{c|}{0.18} \\ 
\hline
\multicolumn{1}{|c|}{5} & \multicolumn{1}{c|}{0} & \multicolumn{1}{c|}{0} & \multicolumn{1}{c|}{0} & \multicolumn{1}{c|}{0} & \multicolumn{1}{c|}{1} & \multicolumn{1}{c|}{0} & \multicolumn{1}{c|}{0} & \multicolumn{1}{c|}{0} & \multicolumn{1}{c|}{0} & \multicolumn{1}{c|}{1} \\ 
\hline
\end{tabular}
}
\end{table}

In addition to the pipes leaking they sometimes break. The processes that lead
to these breaks are not the same as the ones that cause leakage. The reliability
of the pipes affected by these processes are modeled with the Weibull function
 with the values of the parameters given in Table \ref{tbl-reliability-a:8}.


\begin{table}%[h]
\caption{Deterioration parameters and year of construction and intervention information}
\label{tbl-reliability-a:8}
{\tabcolsep2pt
\begin{tabular}{|l|l|l|l|l|l|}
\hline
\multicolumn{1}{|c|}{Zone} & \multicolumn{1}{c|}{Section} & \multicolumn{1}{c|}{Scale parameter} & \multicolumn{1}{c|}{Shape parameter} & \multicolumn{1}{c|}{Duration of corrective intervention} & \multicolumn{1}{c|}{Cost of corrective intervention} \\ 
\multicolumn{1}{|c|}{} & \multicolumn{1}{c|}{} & \multicolumn{1}{c|}{$\alpha$} & \multicolumn{1}{c|}{$m$} & \multicolumn{1}{c|}{(days)} & \multicolumn{1}{c|}{(mus)} \\ 
\hline
\multicolumn{1}{|c|}{} & \multicolumn{1}{c|}{A.1} & \multicolumn{1}{c|}{0.005} & \multicolumn{1}{c|}{1.3} & \multicolumn{1}{c|}{5} & \multicolumn{1}{c|}{8'000} \\ 
\cline{2-6}
\multicolumn{1}{|c|}{} & \multicolumn{1}{c|}{A.2} & \multicolumn{1}{c|}{0.004} & \multicolumn{1}{c|}{1.4} & \multicolumn{1}{c|}{2} & \multicolumn{1}{c|}{12'000} \\ 
\cline{2-6}
\multicolumn{1}{|c|}{A} & \multicolumn{1}{c|}{A.3} & \multicolumn{1}{c|}{0.003} & \multicolumn{1}{c|}{1.5} & \multicolumn{1}{c|}{8} & \multicolumn{1}{c|}{10'000} \\ 
\cline{2-6}
\multicolumn{1}{|c|}{} & \multicolumn{1}{c|}{A.4} & \multicolumn{1}{c|}{0.004} & \multicolumn{1}{c|}{1.5} & \multicolumn{1}{c|}{6} & \multicolumn{1}{c|}{20'000} \\ 
\cline{2-6}
\multicolumn{1}{|c|}{} & \multicolumn{1}{c|}{A.5} & \multicolumn{1}{c|}{0.004} & \multicolumn{1}{c|}{1.3} & \multicolumn{1}{c|}{4} & \multicolumn{1}{c|}{15'000} \\ 
\hline
\multicolumn{1}{|c|}{B} & \multicolumn{1}{c|}{B} & \multicolumn{1}{c|}{0.003} & \multicolumn{1}{c|}{1.3} & \multicolumn{1}{c|}{9} & \multicolumn{1}{c|}{45'000} \\ 
\hline
\multicolumn{1}{|c|}{C} & \multicolumn{1}{c|}{C} & \multicolumn{1}{c|}{0.001} & \multicolumn{1}{c|}{1.7} & \multicolumn{1}{c|}{10} & \multicolumn{1}{c|}{60'000} \\ 
\hline
\multicolumn{1}{|c|}{D} & \multicolumn{1}{c|}{D} & \multicolumn{1}{c|}{0.002} & \multicolumn{1}{c|}{1.5} & \multicolumn{1}{c|}{9} & \multicolumn{1}{c|}{50'000} \\ 
\hline
\multicolumn{1}{|c|}{M} & \multicolumn{1}{c|}{M} & \multicolumn{1}{c|}{0.004} & \multicolumn{1}{c|}{1.2} & \multicolumn{1}{c|}{11} & \multicolumn{1}{c|}{70'000} \\ 
\hline
\end{tabular}
}
\end{table}

The types of intervention considered possible for the pipes, as well as their
unit costs, are given in Table \ref{tbl-reliability-a:9}. If a pipe breaks, it will be replaced
for a cost of 500'000 CHF/100 m. The types of inspection that can be included in
monitoring strategies are given in Table \ref{tbl-reliability-a:10}.

\begin{table}[h]
\caption{Interventions}
\begin{tabular}{|l|l|l|}
\hline
\multicolumn{1}{|c|}{Performance} & \multicolumn{1}{c|}{Intervention} & \multicolumn{1}{c|}{Unit cost} \\ 
\multicolumn{1}{|c|}{state} & \multicolumn{1}{c|}{type} & \multicolumn{1}{c|}{$10^3$ mus/100m} \\ 
\hline
\multicolumn{1}{|c|}{1,2,3,4} & \multicolumn{1}{c|}{Do nothing} & \multicolumn{1}{c|}{0} \\ 
\hline
\multicolumn{1}{|c|}{3} & \multicolumn{1}{c|}{Rehabilitation} & \multicolumn{1}{c|}{25} \\ 
\hline
\multicolumn{1}{|c|}{4} & \multicolumn{1}{c|}{Rehabilitation} & \multicolumn{1}{c|}{30} \\ 
\hline
\multicolumn{1}{|c|}{3,4,5} & \multicolumn{1}{c|}{Replacement} & \multicolumn{1}{c|}{100} \\ 
\hline
\multicolumn{1}{|c|}{1,2,3,4,5} & \multicolumn{1}{c|}{Replacement} & \multicolumn{1}{c|}{500} \\ 
\hline
\end{tabular}
\label{tbl-reliability-a:9}
\end{table}

\begin{table}[h]
\caption{Inspection}
\begin{tabular}{|l|l|}
\hline
\multicolumn{1}{|c|}{Intervention} & \multicolumn{1}{c|}{Unit cost} \\ 
\multicolumn{1}{|c|}{type} & \multicolumn{1}{c|}{$10^3$ mus/100m} \\ 
\hline
\multicolumn{1}{|c|}{Ultra-sonic} & \multicolumn{1}{c|}{3} \\ 
\hline
\multicolumn{1}{|c|}{Acoustic emission} & \multicolumn{1}{c|}{2} \\ 
\hline
\multicolumn{1}{|c|}{Wireless sensor} & \multicolumn{1}{c|}{4} \\ 
\hline
\multicolumn{1}{|c|}{Camera} & \multicolumn{1}{c|}{5} \\ 
\hline
\end{tabular}
\label{tbl-reliability-a:10}
\end{table}
\subsubsection{Question F1}
% \newpage

Estimate the \textit{failure rate} for performance states 1-4 of pipe M

\subsubsection{Answer F1}
%

The failure rate for each performance state is estimated from the reliability of
that state. The reliability of each state equals to the survival probability of
that state. The value of the reliability of each state is therefore equivalent to
the transition probability that the pipe will remains in that state. Values of
transition probability in Table \ref{tbl-reliability-a:7} will be used for computing the
failure rate of each performance state.

Using equation \eqref{eq-reliability-a22}, following results are shown.

\[
 {\theta _1} =  - \frac{{\ln [{R_1}(t = 1)]}}{{t = 1}} =  - \frac{{\ln
[0.93]}}{{t = 1}} \approx 0.073
\]
\[
 {\theta _2} =  - \frac{{\ln [{R_2}(t = 1)]}}{{t = 1}} =  - \frac{{\ln
[0.87]}}{{t = 1}} \approx 0.139
\]
\[
 {\theta _3} =  - \frac{{\ln [{R_3}(t = 1)]}}{{t = 1}} =  - \frac{{\ln
[0.85]}}{{t = 1}} \approx 0.162
\]
\[
 {\theta _4} =  - \frac{{\ln [{R_4}(t = 1)]}}{{t = 1}} =  - \frac{{\ln
[0.8]}}{{t = 1}} \approx 0.223
\]
\subsubsection{Question F2}
Estimate the expected amount of time that pipe M remains in each performance
state
\subsubsection{Answer F2}
The amount of time that pipe M remains in each performance state can be directly
estimated from the reliability function, that is
\begin{eqnarray}
&& {\Theta _i} = \int\limits_0^\infty  {{{\tilde F}_i}(t)dt}  =
\int\limits_0^\infty  {\exp ( - {\theta _i} \cdot t)dt}  = \frac{1}{{{\theta
_i}}}
\label{eq-reliability-a38}
%(38)
\end{eqnarray}
In the case of exponential distribution, the amount of time is the invert value
of the failure rate

\[
 \Theta _1 = \frac{1}{{{\theta _1}}} = \frac{1}{{0.073}} = 13.69 {\rm{ 
years}}
\]
\[
 {\Theta _2} = \frac{1}{{{\theta _2}}} = \frac{1}{{0.139}} = 7.19{\rm{  years}}
\]
\[
 {\Theta _3} = \frac{1}{{{\theta _3}}} = \frac{1}{{0.162}} = 6.17{\rm{  years}}
\]
\[
 {\Theta _4} = \frac{1}{{{\theta _4}}} = \frac{1}{{0.223}} = 4.48{\rm{  years}}
\]

\subsubsection{Question F3}
What is the probability that there will be an adequate level of service in Zone
A in year 2015 (PS4, PS5 = inadequate level of service)? The suspected amount of
water loss in 2014 is given with the following probabilities (PS1 = 0.85, PS2 =
0.10, PS3 = 0.05, PS4 = 0.00; PS5 = 0.00).

\subsubsection{Answer F3}
Using the following equation (in Chapter describing Markov model)

\begin{eqnarray}
&& \overrightarrow {{x^{t + 1}}}  = \left[ {\begin{array}{*{20}{c}}
{{x_1}}&{{x_2}}&{{x_3}}&{{x_4}}&{{x_5}}
\end{array}} \right] \cdot \left[ {\begin{array}{*{20}{c}}
{{p_{11}}}&{{p_{12}}}&{{p_{13}}}&{{p_{14}}}&{{p_{15}}}\\
0&{{p_{22}}}&{{p_{23}}}&{{p_{24}}}&{{p_{25}}}\\
0&0&{{p_{33}}}&{{p_{34}}}&{{p_{35}}}\\
0&0&0&{{p_{44}}}&{{p_{45}}}\\
0&0&0&0&{{p_{55}}}
\end{array}} \right]
\label{eq-reliability-a39}
%(39)
\end{eqnarray}
to estimate the state probability in year 2015
%
\begin{eqnarray}
\begin{array}{l}
\overrightarrow {{x^{t + 1}}}  = \left[ {\begin{array}{*{20}{c}}
{0.85}&{0.1}&{0.05}&0&0
\end{array}} \right] \cdot \left[ {\begin{array}{*{20}{c}}
{0.91}&{0.09}&0&0&0\\
0&{0.88}&{0.12}&0&0\\
0&0&{0.85}&{0.15}&0\\
0&0&0&{0.82}&{0.18}\\
0&0&0&0&1
\end{array}} \right]\\
\;\;\;\;\;\\
\;\;\;\;\; = \left[ {\begin{array}{*{20}{c}}
{0.7735}&{0.1645}&{0.0545}&{0.0075}&0
\end{array}} \right]
\end{array}
\label{eq-reliability-a40}
%(40)
\end{eqnarray}
The probability that pipe in zone A will be in an adequate level of service is
the sum of the state probability from CS 1 to CS 3, that is
\[
\Pr {o^{2015}}\left[ {{\text{Zone A in adequate level of service}}} \right] =
\sum\limits_{i = 1}^3 {x_i^{2015} = 0.7735 + 0.1645 + 0.0545 = 0.9925}
\]
\subsubsection{Question F4}
Calculate the probability of there being at least one pipe break in zone A in
year 2020 (5 points).
\subsubsection{Answer F4}

\underline{Step 1:} Calculate the reliability for each pipe A1, A2, A3, A4, A5 in Zone A. The failure probability of each pipe is independent from each other.

Example is done for pipe A1
\[
{R^{A1}}(t = 22) = {e^{ - {{(0.005 \cdot 22)}^{1.3}}}} = 0.945
\]
the failure probability is
\[
{F^{A1}}(t = 22) = 1 - {R^{A1}}(t = 22) = 1 - 0.945 = 0.055
\]
\underline{Step 2:} Since the break of pipe is not dependent, it is not joint
probability and thus, it is simple the average failure probability, which equals
to 0.025.

The calculations are shown in Table \ref{tbl-reliability-a:11}.
\begin{table}[h]
\caption{Calculations (2020)}
\begin{tabular}{|l|l|l|l|l|l|l|l|}
\hline
\multicolumn{1}{|c|}{Zone} & \multicolumn{1}{c|}{Section} & \multicolumn{1}{c|}{Scale parameter} & \multicolumn{1}{c|}{Shape parameter} & \multicolumn{1}{c|}{Elapsed time till 2013} & \multicolumn{1}{c|}{Elapsed time till 2020} & \multicolumn{1}{c|}{Reliability} & \multicolumn{1}{c|}{Failure probability} \\ 
\multicolumn{1}{|c|}{} & \multicolumn{1}{c|}{} & \multicolumn{1}{c|}{$\alpha$} & \multicolumn{1}{c|}{$m$} & \multicolumn{1}{c|}{(years)} & \multicolumn{1}{c|}{(years)} & \multicolumn{1}{c|}{($R(t)$)} & \multicolumn{1}{c|}{($F(t)$)} \\ 
\hline
\multicolumn{1}{|c|}{} & \multicolumn{1}{c|}{A.1} & \multicolumn{1}{c|}{0.005} & \multicolumn{1}{c|}{1.3} & \multicolumn{1}{c|}{15} & \multicolumn{1}{c|}{22} & \multicolumn{1}{c|}{0.945} & \multicolumn{1}{c|}{0.055} \\ 
\cline{2-8}
\multicolumn{1}{|c|}{} & \multicolumn{1}{c|}{A.2} & \multicolumn{1}{c|}{0.004} & \multicolumn{1}{c|}{1.4} & \multicolumn{1}{c|}{10} & \multicolumn{1}{c|}{17} & \multicolumn{1}{c|}{0.977} & \multicolumn{1}{c|}{0.023} \\ 
\cline{2-8}
\multicolumn{1}{|c|}{A} & \multicolumn{1}{c|}{A.3} & \multicolumn{1}{c|}{0.003} & \multicolumn{1}{c|}{1.5} & \multicolumn{1}{c|}{10} & \multicolumn{1}{c|}{17} & \multicolumn{1}{c|}{0.988} & \multicolumn{1}{c|}{0.012} \\ 
\cline{2-8}
\multicolumn{1}{|c|}{} & \multicolumn{1}{c|}{A.4} & \multicolumn{1}{c|}{0.004} & \multicolumn{1}{c|}{1.5} & \multicolumn{1}{c|}{10} & \multicolumn{1}{c|}{17} & \multicolumn{1}{c|}{0.982} & \multicolumn{1}{c|}{0.018} \\ 
\cline{2-8}
\multicolumn{1}{|c|}{} & \multicolumn{1}{c|}{A.5} & \multicolumn{1}{c|}{0.004} & \multicolumn{1}{c|}{1.3} & \multicolumn{1}{c|}{10} & \multicolumn{1}{c|}{17} & \multicolumn{1}{c|}{0.982} & \multicolumn{1}{c|}{0.018} \\ 
\hline
\end{tabular}
\label{tbl-reliability-a:11}
\end{table}

\subsubsection{Question F5}
When is it expected that the annual probability of pipe A1 breaking will exceed
0.3 given that it did not break earlier?

\subsubsection{Answer F5}
Value of time t can be estimated from from equation \eqref{eqreliability:10}
\[
\begin{array}{l}
R(t) = {e^{ - {{(\alpha  \cdot t)}^m}}} \Leftrightarrow \ln \left[ {R(t)}
\right] = \ln \left[ {{e^{ - {{(\alpha  \cdot t)}^m}}}} \right] =  - {(\alpha 
\cdot t)^m}\\
\to {t^m} =  - \frac{{\ln \left[ {R(t)} \right]}}{{{\alpha ^m}}} =  -
\frac{{\ln \left[ {0.945} \right]}}{{{{0.005}^{1.3}}}} = 55.45 \approx 56{\rm{
years}}
\end{array}
\]
\subsubsection{Question F6}
Calculate the reliability of being able to provide an adequate level of service
to zone A in year 2014 (Keep in mind that water can flow in both directions
through the pipes).
\subsubsection{Answer F6}
\underline{Step 1:} Calculate the reliability of each pipe in Zone A using given
reliability function. This is exactly the same calculation as shown in subsection
6.8, except the elapsed time till 2014 for each pipe is different. Table
\ref{tbl-reliability-a:12} shows the results of estimation for reliability and failure in year
2014.

\begin{table}[h]
\caption{Calculations (2014)}
\begin{tabular}{|l|l|l|l|l|l|l|l|}
\hline
\multicolumn{1}{|c|}{Zone} & \multicolumn{1}{c|}{Section} & \multicolumn{1}{c|}{Scale parameter} & \multicolumn{1}{c|}{Shape parameter} & \multicolumn{1}{c|}{Elapsed time till 2013} & \multicolumn{1}{c|}{Elapsed time till 2020} & \multicolumn{1}{c|}{Reliability} & \multicolumn{1}{c|}{Failure probability} \\ 
\multicolumn{1}{|c|}{} & \multicolumn{1}{c|}{} & \multicolumn{1}{c|}{$\alpha$} & \multicolumn{1}{c|}{$m$} & \multicolumn{1}{c|}{(years)} & \multicolumn{1}{c|}{(years)} & \multicolumn{1}{c|}{($R(t)$)} & \multicolumn{1}{c|}{($F(t)$)} \\ 
\hline
\multicolumn{1}{|c|}{} & \multicolumn{1}{c|}{A.1} & \multicolumn{1}{c|}{0.005} & \multicolumn{1}{c|}{1.3} & \multicolumn{1}{c|}{15} & \multicolumn{1}{c|}{16} & \multicolumn{1}{c|}{0.963} & \multicolumn{1}{c|}{0.037} \\ 
\cline{2-8}
\multicolumn{1}{|c|}{} & \multicolumn{1}{c|}{A.2} & \multicolumn{1}{c|}{0.004} & \multicolumn{1}{c|}{1.4} & \multicolumn{1}{c|}{10} & \multicolumn{1}{c|}{11} & \multicolumn{1}{c|}{0.987} & \multicolumn{1}{c|}{0.013} \\ 
\cline{2-8}
\multicolumn{1}{|c|}{A} & \multicolumn{1}{c|}{A.3} & \multicolumn{1}{c|}{0.003} & \multicolumn{1}{c|}{1.5} & \multicolumn{1}{c|}{10} & \multicolumn{1}{c|}{11} & \multicolumn{1}{c|}{0.994} & \multicolumn{1}{c|}{0.006} \\ 
\cline{2-8}
\multicolumn{1}{|c|}{} & \multicolumn{1}{c|}{A.4} & \multicolumn{1}{c|}{0.004} & \multicolumn{1}{c|}{1.5} & \multicolumn{1}{c|}{10} & \multicolumn{1}{c|}{11} & \multicolumn{1}{c|}{0.991} & \multicolumn{1}{c|}{0.009} \\ 
\cline{2-8}
\multicolumn{1}{|c|}{} & \multicolumn{1}{c|}{A.5} & \multicolumn{1}{c|}{0.004} & \multicolumn{1}{c|}{1.3} & \multicolumn{1}{c|}{10} & \multicolumn{1}{c|}{11} & \multicolumn{1}{c|}{0.991} & \multicolumn{1}{c|}{0.009} \\ 
\hline
\end{tabular}
\label{tbl-reliability-a:12}
\end{table}
\underline{Step 2:} if for example pipe break at A.2 or A.4, the system is still
functioning because the water can flow into both direction. And even A.3 break,
it can still provide adequate level of service. In that situation, the system
turns to be like a series network with independent reliability.

3 scenarios can be seen

\begin{itemize}
	\item Scenario 1: If A.1 breaks, it affects the entire zone A
	\item Scenario 2: If A.2 breaks, it does not, because water can flow into A4
	\item Scenario 3: If A.4 breaks, it does not, because water can flow in to A2
\end{itemize}
Following is the calculation for reliability of each scenario
\[
 {R^{scenario1}}(2014) = {R^{A1}}(2014) = 0.963
\]
\[
 {R^{scenario2}}(2014) = {R^{A1}}(2014) \cdot {R^{A3}}(2014) \cdot
{R^{A4}}(2014) \cdot {R^{A5}}(2014) = 0.940
\]
\[
 {R^{scenario3}}(2014) = {R^{A1}}(2014) \cdot {R^{A2}}(2014) \cdot
{R^{A3}}(2014) \cdot {R^{A5}}(2014) = 0.937
\]
As each scenario can occurs independently from other scenarios, the average
reliability is then evaluated as the average reliability of the three scenario
\[
\begin{array}{l}
{R^{{\rm{Adequate level of service}}}}(2014) = \frac{{{R^{{\rm{scenario
1}}}}(2014) + {R^{{\rm{scenario 2}}}}(2014) + {R^{{\rm{scenario
3}}}}(2014)}}{3}\\
{\rm{                                         =  }}\frac{{0.963 + 0.940 +
0.937}}{3} = 0.947
\end{array}
\]
\subsection{Problem G}
The details of the number of fan units used in road tunnels retired at different
ages are given in Table \ref{tbl-reliability-a:13}.
\begin{table}[h]
\caption{Record of fan units failed}
\begin{tabular}{|l|l|l|l|l|l|l|l|l|}
\hline
\multicolumn{1}{|c|}{Item} & \multicolumn{1}{c|}{time} & \multicolumn{1}{c|}{status} & \multicolumn{1}{c|}{Item} & \multicolumn{1}{c|}{time} & \multicolumn{1}{c|}{status} & \multicolumn{1}{c|}{Item} & \multicolumn{1}{c|}{time} & \multicolumn{1}{c|}{status} \\ 
\hline
\multicolumn{1}{|c|}{1} & \multicolumn{1}{c|}{15} & \multicolumn{1}{c|}{0} & \multicolumn{1}{c|}{13} & \multicolumn{1}{c|}{14} & \multicolumn{1}{c|}{0} & \multicolumn{1}{c|}{25} & \multicolumn{1}{c|}{12} & \multicolumn{1}{c|}{1} \\ 
\hline
\multicolumn{1}{|c|}{2} & \multicolumn{1}{c|}{11} & \multicolumn{1}{c|}{1} & \multicolumn{1}{c|}{14} & \multicolumn{1}{c|}{11} & \multicolumn{1}{c|}{1} & \multicolumn{1}{c|}{26} & \multicolumn{1}{c|}{13} & \multicolumn{1}{c|}{0} \\ 
\hline
\multicolumn{1}{|c|}{3} & \multicolumn{1}{c|}{10} & \multicolumn{1}{c|}{0} & \multicolumn{1}{c|}{15} & \multicolumn{1}{c|}{16} & \multicolumn{1}{c|}{0} & \multicolumn{1}{c|}{27} & \multicolumn{1}{c|}{13} & \multicolumn{1}{c|}{0} \\ 
\hline
\multicolumn{1}{|c|}{4} & \multicolumn{1}{c|}{12} & \multicolumn{1}{c|}{0} & \multicolumn{1}{c|}{16} & \multicolumn{1}{c|}{11} & \multicolumn{1}{c|}{0} & \multicolumn{1}{c|}{28} & \multicolumn{1}{c|}{14} & \multicolumn{1}{c|}{0} \\ 
\hline
\multicolumn{1}{|c|}{5} & \multicolumn{1}{c|}{18} & \multicolumn{1}{c|}{0} & \multicolumn{1}{c|}{17} & \multicolumn{1}{c|}{14} & \multicolumn{1}{c|}{1} & \multicolumn{1}{c|}{29} & \multicolumn{1}{c|}{19} & \multicolumn{1}{c|}{0} \\ 
\hline
\multicolumn{1}{|c|}{6} & \multicolumn{1}{c|}{12} & \multicolumn{1}{c|}{0} & \multicolumn{1}{c|}{18} & \multicolumn{1}{c|}{18} & \multicolumn{1}{c|}{1} & \multicolumn{1}{c|}{30} & \multicolumn{1}{c|}{11} & \multicolumn{1}{c|}{0} \\ 
\hline
\multicolumn{1}{|c|}{7} & \multicolumn{1}{c|}{16} & \multicolumn{1}{c|}{1} & \multicolumn{1}{c|}{19} & \multicolumn{1}{c|}{17} & \multicolumn{1}{c|}{0} & \multicolumn{1}{c|}{31} & \multicolumn{1}{c|}{20} & \multicolumn{1}{c|}{1} \\ 
\hline
\multicolumn{1}{|c|}{8} & \multicolumn{1}{c|}{19} & \multicolumn{1}{c|}{0} & \multicolumn{1}{c|}{20} & \multicolumn{1}{c|}{17} & \multicolumn{1}{c|}{0} & \multicolumn{1}{c|}{32} & \multicolumn{1}{c|}{12} & \multicolumn{1}{c|}{1} \\ 
\hline
\multicolumn{1}{|c|}{9} & \multicolumn{1}{c|}{12} & \multicolumn{1}{c|}{0} & \multicolumn{1}{c|}{21} & \multicolumn{1}{c|}{16} & \multicolumn{1}{c|}{1} & \multicolumn{1}{c|}{33} & \multicolumn{1}{c|}{10} & \multicolumn{1}{c|}{0} \\ 
\hline
\multicolumn{1}{|c|}{10} & \multicolumn{1}{c|}{13} & \multicolumn{1}{c|}{0} & \multicolumn{1}{c|}{22} & \multicolumn{1}{c|}{20} & \multicolumn{1}{c|}{1} & \multicolumn{1}{c|}{34} & \multicolumn{1}{c|}{14} & \multicolumn{1}{c|}{1} \\ 
\hline
\multicolumn{1}{|c|}{11} & \multicolumn{1}{c|}{20} & \multicolumn{1}{c|}{1} & \multicolumn{1}{c|}{23} & \multicolumn{1}{c|}{12} & \multicolumn{1}{c|}{1} & \multicolumn{1}{c|}{35} & \multicolumn{1}{c|}{14} & \multicolumn{1}{c|}{1} \\ 
\hline
\multicolumn{1}{|c|}{12} & \multicolumn{1}{c|}{16} & \multicolumn{1}{c|}{0} & \multicolumn{1}{c|}{24} & \multicolumn{1}{c|}{15} & \multicolumn{1}{c|}{1} & \multicolumn{1}{c|}{36} & \multicolumn{1}{c|}{16} & \multicolumn{1}{c|}{0} \\ 
\hline
\end{tabular}
\label{tbl-reliability-a:13}
\end{table}
Value in the column ``status'' has binary value 1 and 0. Value 1 and 0 represent
the ``fail'' or ``not fail'' of the tunnel fans at the inspection time.
\subsubsection{Question G1}
What is the average expected life of the units that have survived until year 4?
\subsubsection{Answer G1}
The observed data infers a case that the failure rate is not a constant value.
The failure rate of the fan is a time-varying variable. The failure rate can be
followed a Weibull function (Equation \eqref{eqreliability:12}).

The average life of the units that have survived until a certain year will be
defined as
\begin{eqnarray}
&& \Phi \left[ {\tau  \ge t} \right] = \left[ {\int\limits_t^\infty  {R(\tau )d\tau
} } \right] = \left[ {\int\limits_t^\infty  {{e^{ - {{(\alpha  \cdot \tau
)}^m}}}d\tau } } \right]
\label{eq-reliability-a44}
%(44)
\end{eqnarray}
In this example, t=4 years. The unknown parameters are value of $\alpha $ and m.
Values of these parameters can be estimated with the Maximum likelihood
estimation approach (MLE) (see the R code in the Appendix).

Using the MLE, value of $\alpha $ and m are 0.0542 and 5.8582, respectively. The
average life of the units that have survived after 4 years will be
\[
\Phi \left[ {\tau  \ge 4} \right] = \left[ {\int\limits_4^\infty  {{e^{ -
{{(0.0542 \cdot \tau )}^{5.8584}}}}d\tau } } \right] \approx 13{\rm{ years}}
\]
\bibliographystyle{plainnat}
\bibliography{reference}

\pagebreak

\begin{subappendices}
% \label{appendix3}
\section{Reliability of an item with time invariant failure rate}\label{appen31}
% \textbf{Reliability of an item with time invariant failure rate}
\lstinputlisting[basicstyle=\ttfamily\scriptsize]{./Programs/IMP-reliability/IMP-example1-time-invarying.r}
\pagebreak
\section{Reliability of an item with time-variant failure rate}\label{appen32}
% \textbf{Reliability of an item with time-variant failure rate}
\lstinputlisting[basicstyle=\ttfamily\scriptsize]{./Programs/IMP-reliability/IMP-example2-time-varying.r}
\pagebreak
\section{Maximum likelihood estimation for Weibull model}\label{appen33}
% \textbf{Reliability of an item with time-variant failure rate}
\lstinputlisting[basicstyle=\ttfamily\scriptsize]{./Programs/IMP-reliability/IMP-A-time-varying-fan.r}
\end{subappendices}