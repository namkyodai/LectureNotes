%%%%%%%%%%%%%%%%%%%%% chapter.tex %%%%%%%%%%%%%%%%%%%%%%%%%%%%%%%%%
%
% sample chapter
%
% Use this file as a template for your own input.
%
%%%%%%%%%%%%%%%%%%%%%%%% Springer-Verlag %%%%%%%%%%%%%%%%%%%%%%%%%%
%\motto{Use the template \emph{chapter.tex} to style the various elements of your chapter content.}
\chapter{Availability and maintainability}
\label{avaimain} % Always give a unique label
% use \chaptermark{}
% to alter or adjust the chapter heading in the running head
\chapterauthor{Bryan T. Adey and Nam Lethanh}
\section{Maintainability of an item}
When managing infrastructure the maintainability of an item is often a useful
bit of information, i.e. how much effort is it to restore the item so that it
once again provides an adequate LOS. For example, if the speeds of trains need to
be reduced on a rail link due to deformations in a track, how much will it cost
to fix the rail link in a way that the trains can once again run at normal speed,
and how long will this take?

The maintainability of an item only has to do with the execution of
interventions on the item. The maintainability and reliability together give the
availability of an item.

\subsection{Time}
In order to estimate how long it will take to restore an item so that it
provides an adequate LOS, it is necessary to estimate the steps to be taken from
the instant that an inadequate LOS is not provided until it is restored. These
steps include the determination that a failure has occurred, the determination of
what exactly went wrong, the determination of how to restore the item, the
execution of the intervention, the testing that the item again can provide an
adequate LOS and then putting it back in operation so that it does provide an
adequate LOS. An illustration of this for the restoration of a part of an
electricity distribution network using Business Process Modelling Notation is
shown in Figure \ref{avaimain:1}.

\begin{figure}[h]
% \begin{center}
\includegraphics[width=450pt]{avaimain-1.eps}
\caption{Simplified process for the restoration of a part of an electricity
distribution network}\label{avaimain:1}
% \end{center}
\end{figure}
Once the process has been identified it is necessary to estimate the length of
time that each of the activities will take. This depends on many factors,
including the quantity and skill levels of persons required and available, the
quantity of required replacement parts required and available, and the amount of
time required to react to an indication that something is not working, sometimes
referred to as responsiveness.
\subsection{Indicators}
Indicators that are often used to express maintainability are the
mode\footnote{The duration or costs that occurs the most often},
mean\footnote{The average of the sum of all durations or costs divided by the
number of values} and maximum\footnote{The largest of all durations or costs
observed or expected. This is also often given as a percentile of the largest of
all durations or costs.} amount of duration or cost of interventions. Indicators
are usually developed for specific types of interventions, e.g. the mean duration
of a corrective intervention is 3 months, or the mean duration of a preventive
intervention is 1 week.
\section{Maintainability of an item comprised of sub-items}
The maintainability of an item can be estimated by taking into consideration the
structure of the sub-items in the item, and the maintainability of the sub-items.
This is advantageous if there is little data on the interventions that have been
executed on the item itself, e.g. a bridge, but abundant data on interventions
that have been executed on sub-items similar to those of which the item is
composed, e.g. a concrete abutment.

If it can be assumed that each intervention is only used to restore one sub-item
so that it provides an adequate LOS, the durations of the interventions can be
simply combined, e.g. if it takes 1 week to repair an abutment and 2 weeks to
repair a deck and both happen once every 10 years, the average time spent to
repair a bridge per year, i.e. its' maintainability, is 0.30 weeks ((1+2)/10).
If, however, 50\% of the time the abutment and deck are repaired at the same time
which takes 2 weeks, 50\% of the time they are repaired separately than the
average time spent to repair a bridge per year, i.e. its maintainability, is
0.175 weeks ((0.5(2)+0.25(1+2))/10).

The basic steps to combine sub-items, if it can be assumed that each
intervention is only used to restore one sub-item so that it provides an adequate
LOS, are:
\begin{itemize}
	\item divide all sub-items by type, so it can be assumed that each sub-item per
sub-item type behaves in the same way,
	\item determine the rate of occurrence of the type of intervention of interest, e.g.
corrective intervention, per sub-item type,
	\item estimate the rate of occurrence of the type of intervention, per sub-item type
by multiplying the rate of occurrence of the type of intervention by the number
of sub-items per sub-type,
	\item estimate the total duration of the type of intervention per sub-item type by
multiplying the rate of occurrence of the type of intervention by the mean
duration of the type of intervention per sub-item type
	\item estimate the rate of occurrence of an intervention on the item by summing the
rates of occurrence of the types of interventions per sub-item
	\item estimate the expected intervention duration, or maintainability, of the item by
summing the total duration of the type of intervention per sub-item.
\end{itemize}
\section{Availability of an item}
The availability of an item is based on both the reliability of an item and the
maintainability of an item. Simply, availability is:
\begin{eqnarray}
&& {A_i} = \frac{{E\left[ {t_i^{adequate}} \right]}}{{E\left[ {t_i^{adequate}}
\right] + E\left[ {t_i^{inadequate}} \right]}}
\label{avaimaineq:1}
%(1)
\end{eqnarray}
Where:
\begin{adjustwidth}{1cm}{}
\begin{description}
\item[$t_i^{adequate}$:] the duration of time item \textit{i} provides an adequate LOS,
\item[$t_i^{inadequate}$:] the duration of time item \textit{i} provides an inadequate LOS
\end{description}
%\end{flushright}
\end{adjustwidth}
When investigating the availability of an item the exact definition of
availability used is important. One should be explicit about
\begin{itemize}
	\item the types of interventions being considered, e.g. preventive and corrective
interventions or only corrective interventions
	\item the type of time being considered, e.g. is the time spent waiting for parts and
the time on-site to be considered, or only the time on-site.
\end{itemize}
Different measure of availability serve different purposes.
\subsection{Example}
A road has been in service for 10 years (3'652 days). Over those 10 years, 6
interventions have been executed. Some of those have been due to rock falls, some
due to scour and some due to chloride induced corrosion. The total amount of time
that an adequate LOS was not provided was 156 days, i.e. traffic flow was
restricted for 156 days in the last 3'652 days.
\subsubsection{Question A}
What was the availability of the road link?
\subsubsection{Answer A}
The total length of time when an adequate LOS was provided was 3'496 days (3'652
-- 156). The rate of occurrence of interventions was 0.001716 interventions/day
(6/3'496). The mean time between interventions is therefore 582.7 days
(1/0.001716). The availability is then
\begin{eqnarray}
&& \begin{array}{l}
 {A_i} = \frac{{E\left[ {t_i^a} \right]}}{{E\left[ {t_i^a} \right] + E\left[
{t_i^n} \right]}} = \frac{{582.7}}{{582.7 + 156/6}} = 0.9573
\end{array} \label{avaimaineq:2}
%(2)
\end{eqnarray}
\subsubsection{Question B}
Is this also the availability of the road next year?
\subsubsection{Answer B}
It depends on the types of failures and how they were repaired. In order to
state that the availability of the road next year is also 0.9573, it would be
necessary to be able to model the time between interventions as an exponential
distribution, meaning that rate of occurrence of failure can be modelled as
constant. This is essentially saying that no matter how many failure have
occurred in the past there will be on average the same number in the future. This
is in many cases not correct. For example, if there is an area in which a
landslide damages a road, it may be that there will never be another landslide in
this area. Or, if many interventions has to be executed due the corrosion of the
reinforcement in deck slabs and waterproofing layers on all than the number of
interventions per year in the future would change.
\section{Availability of an item comprised of sub-items}
In determining the availability of an item comprised of sub-items it is
important to explicitly take into consideration the structure of the item. The
availability of item \textit{i} comprised of sub-items connected in series is
given by:
\begin{eqnarray}
&& {A_i} = \prod\limits_j^J {\left( {\frac{{E\left[ {t_j^a} \right]}}{{E\left[
{t_j^a} \right] + E\left[ {t_j^n} \right]}}} \right)}
\label{avaimaineq:3}
%(3)
\end{eqnarray}
\subsection{Example}
You own a water distribution network that is connected as shown in Figure
\ref{avaimain:2}. The five pipes in the network have life expectancies of 20 years.
The life expectancies can be modelled using an exponential distribution. It takes
1 week to repair pipes 1 and 4, 2 weeks for pipes 2 and 5 and 3 weeks for pipe 3,
once it breaks.
\begin{figure}[h]
% \begin{center}
\includegraphics[width=450pt]{avaimain-2.eps}
\caption{Water distribution network}\label{avaimain:2}
% \end{center}
\end{figure}
\subsubsection{Question C}
What is the reliability of the network?
\subsubsection{Question D}
What is the availability of the network?
\subsubsection{Answer C}
Using the pivotal decomposition method, the structure function of the network is
\begin{eqnarray}
&& \begin{array}{l}
\phi (\vec x) = {x_1}{x_3}{x_4} + {x_1}{x_3}{x_5} + {x_2}{x_3}{x_4} +
{x_2}{x_3}{x_5} + {x_1}{x_2}{x_4}{x_5}\\
{\rm{          }} - {x_1}{x_2}{x_3}{x_4} - {x_1}{x_3}{x_4}{x_5} -
{x_1}{x_2}{x_3}{x_5} - {x_2}{x_3}{x_4}{x_5}
\end{array}
\label{avaimaineq:4}
%(4)
\end{eqnarray}
The steps to arrive at the above function has been described in section 3.3 of
the script of week 9.

In term of reliability, the structure function can be rewritten as
\begin{eqnarray}
&& \begin{array}{l}
{R_{network}}(t) = {R_1}(t) \cdot {R_3}(t) \cdot {R_4}(t) + {R_1}(t) \cdot
{R_3}(t) \cdot {R_5}(t) + {R_2}(t) \cdot {R_3}(t) \cdot {R_4}(t) + {R_2}(t) \cdot
{R_3}(t) \cdot {R_5}(t)\\
+ {R_1}(t) \cdot {R_2}(t) \cdot {R_4}(t) \cdot {R_5}(t){\rm{ }} - {R_1}(t)
\cdot {R_2}(t) \cdot {R_3}(t) \cdot {R_4}(t)\\
- {R_1}(t) \cdot {R_3}(t) \cdot {R_4}(t) \cdot {R_5}(t) - {R_1}(t) \cdot
{R_2}(t) \cdot {R_3}(t) \cdot {R_5}(t) - {R_2}(t) \cdot {R_3}(t) \cdot {R_4}(t)
\cdot {R_5}(t)
\end{array}
\label{avaimaineq:5}
%(5)
\end{eqnarray}
The reliability of any item \textit{i} at time \textit{t} is defined in \eqref{eqreliability:7}

As the expected time until an inadequate LOS is provided is about 20 years, the
``failure rate'' can be estimated as follows:

\[
 \int\limits_0^{ + \infty } {{R_i}(t)} dt = \int\limits_0^{ + \infty } {\exp
\left( { - {\theta _i} \cdot t} \right)dt}  = \frac{1}{{{\theta _i}}} = 20{\rm{
years}}
\]
\[
 {\theta _i} = \frac{1}{{20}} = 0.05
\]

Using equations \eqref{eqreliability:7} the reliability of each sub-item, and the item, in 30 years
can be estimated (Table \ref{tblavaimain:1}). For example, the probability that sub-item
1 will provide an adequate LOS until the start of year 14 is 0.522 and the
probability that the item will provide an adequate LOS until year 14 is 0.346.
This is illustrated in Figure \ref{avaimain:3}. Only one curve is used to represent
the reliability of a sub-item, since the reliability for all sub-items is the
same.
\begin{table}
\caption{Reliability}
\begin{tabular}{|l|l|l|l|l|l|l|}
\hline
\multicolumn{1}{|c|}{Time} & \multicolumn{6}{c|}{Reliability} \\ 
\cline{2-7}
\multicolumn{1}{|c|}{(years)} & \multicolumn{1}{c|}{1} & \multicolumn{1}{c|}{2} & \multicolumn{1}{c|}{3} & \multicolumn{1}{c|}{4} & \multicolumn{1}{c|}{5} & \multicolumn{1}{c|}{Item} \\ 
\hline
\multicolumn{1}{|c|}{1} & \multicolumn{1}{c|}{1} & \multicolumn{1}{c|}{1} & \multicolumn{1}{c|}{1} & \multicolumn{1}{c|}{1} & \multicolumn{1}{c|}{1} & \multicolumn{1}{c|}{1} \\ 
\hline
\multicolumn{1}{|c|}{2} & \multicolumn{1}{c|}{0.951} & \multicolumn{1}{c|}{0.951} & \multicolumn{1}{c|}{0.951} & \multicolumn{1}{c|}{0.951} & \multicolumn{1}{c|}{0.951} & \multicolumn{1}{c|}{0.987} \\ 
\hline
\multicolumn{1}{|c|}{3} & \multicolumn{1}{c|}{0.905} & \multicolumn{1}{c|}{0.905} & \multicolumn{1}{c|}{0.905} & \multicolumn{1}{c|}{0.905} & \multicolumn{1}{c|}{0.905} & \multicolumn{1}{c|}{0.952} \\ 
\hline
\multicolumn{1}{|c|}{4} & \multicolumn{1}{c|}{0.861} & \multicolumn{1}{c|}{0.861} & \multicolumn{1}{c|}{0.861} & \multicolumn{1}{c|}{0.861} & \multicolumn{1}{c|}{0.861} & \multicolumn{1}{c|}{0.904} \\ 
\hline
\multicolumn{1}{|c|}{5} & \multicolumn{1}{c|}{0.819} & \multicolumn{1}{c|}{0.819} & \multicolumn{1}{c|}{0.819} & \multicolumn{1}{c|}{0.819} & \multicolumn{1}{c|}{0.819} & \multicolumn{1}{c|}{0.847} \\ 
\hline
\multicolumn{1}{|c|}{6} & \multicolumn{1}{c|}{0.779} & \multicolumn{1}{c|}{0.779} & \multicolumn{1}{c|}{0.779} & \multicolumn{1}{c|}{0.779} & \multicolumn{1}{c|}{0.779} & \multicolumn{1}{c|}{0.786} \\ 
\hline
\multicolumn{1}{|c|}{7} & \multicolumn{1}{c|}{0.741} & \multicolumn{1}{c|}{0.741} & \multicolumn{1}{c|}{0.741} & \multicolumn{1}{c|}{0.741} & \multicolumn{1}{c|}{0.741} & \multicolumn{1}{c|}{0.723} \\ 
\hline
\multicolumn{1}{|c|}{8} & \multicolumn{1}{c|}{0.705} & \multicolumn{1}{c|}{0.705} & \multicolumn{1}{c|}{0.705} & \multicolumn{1}{c|}{0.705} & \multicolumn{1}{c|}{0.705} & \multicolumn{1}{c|}{0.66} \\ 
\hline
\multicolumn{1}{|c|}{9} & \multicolumn{1}{c|}{0.670} & \multicolumn{1}{c|}{0.670} & \multicolumn{1}{c|}{0.670} & \multicolumn{1}{c|}{0.670} & \multicolumn{1}{c|}{0.670} & \multicolumn{1}{c|}{0.599} \\ 
\hline
\multicolumn{1}{|c|}{10} & \multicolumn{1}{c|}{0.638} & \multicolumn{1}{c|}{0.638} & \multicolumn{1}{c|}{0.638} & \multicolumn{1}{c|}{0.638} & \multicolumn{1}{c|}{0.638} & \multicolumn{1}{c|}{0.541} \\ 
\hline
\multicolumn{1}{|c|}{11} & \multicolumn{1}{c|}{0.607} & \multicolumn{1}{c|}{0.607} & \multicolumn{1}{c|}{0.607} & \multicolumn{1}{c|}{0.607} & \multicolumn{1}{c|}{0.607} & \multicolumn{1}{c|}{0.487} \\ 
\hline
\multicolumn{1}{|c|}{12} & \multicolumn{1}{c|}{0.577} & \multicolumn{1}{c|}{0.577} & \multicolumn{1}{c|}{0.577} & \multicolumn{1}{c|}{0.577} & \multicolumn{1}{c|}{0.577} & \multicolumn{1}{c|}{0.436} \\ 
\hline
\multicolumn{1}{|c|}{13} & \multicolumn{1}{c|}{0.549} & \multicolumn{1}{c|}{0.549} & \multicolumn{1}{c|}{0.549} & \multicolumn{1}{c|}{0.549} & \multicolumn{1}{c|}{0.549} & \multicolumn{1}{c|}{0.389} \\ 
\hline
\multicolumn{1}{|c|}{14} & \multicolumn{1}{c|}{0.522} & \multicolumn{1}{c|}{0.522} & \multicolumn{1}{c|}{0.522} & \multicolumn{1}{c|}{0.522} & \multicolumn{1}{c|}{0.522} & \multicolumn{1}{c|}{0.346} \\ 
\hline
\multicolumn{1}{|c|}{15} & \multicolumn{1}{c|}{0.497} & \multicolumn{1}{c|}{0.497} & \multicolumn{1}{c|}{0.497} & \multicolumn{1}{c|}{0.497} & \multicolumn{1}{c|}{0.497} & \multicolumn{1}{c|}{0.307} \\ 
\hline
\multicolumn{1}{|c|}{16} & \multicolumn{1}{c|}{0.472} & \multicolumn{1}{c|}{0.472} & \multicolumn{1}{c|}{0.472} & \multicolumn{1}{c|}{0.472} & \multicolumn{1}{c|}{0.472} & \multicolumn{1}{c|}{0.272} \\ 
\hline
\multicolumn{1}{|c|}{17} & \multicolumn{1}{c|}{0.449} & \multicolumn{1}{c|}{0.449} & \multicolumn{1}{c|}{0.449} & \multicolumn{1}{c|}{0.449} & \multicolumn{1}{c|}{0.449} & \multicolumn{1}{c|}{0.241} \\ 
\hline
\multicolumn{1}{|c|}{18} & \multicolumn{1}{c|}{0.427} & \multicolumn{1}{c|}{0.427} & \multicolumn{1}{c|}{0.427} & \multicolumn{1}{c|}{0.427} & \multicolumn{1}{c|}{0.427} & \multicolumn{1}{c|}{0.212} \\ 
\hline
\multicolumn{1}{|c|}{19} & \multicolumn{1}{c|}{0.407} & \multicolumn{1}{c|}{0.407} & \multicolumn{1}{c|}{0.407} & \multicolumn{1}{c|}{0.407} & \multicolumn{1}{c|}{0.407} & \multicolumn{1}{c|}{0.187} \\ 
\hline
\multicolumn{1}{|c|}{20} & \multicolumn{1}{c|}{0.387} & \multicolumn{1}{c|}{0.387} & \multicolumn{1}{c|}{0.387} & \multicolumn{1}{c|}{0.387} & \multicolumn{1}{c|}{0.387} & \multicolumn{1}{c|}{0.164} \\ 
\hline
\multicolumn{1}{|c|}{21} & \multicolumn{1}{c|}{0.368} & \multicolumn{1}{c|}{0.368} & \multicolumn{1}{c|}{0.368} & \multicolumn{1}{c|}{0.368} & \multicolumn{1}{c|}{0.368} & \multicolumn{1}{c|}{0.144} \\ 
\hline
\multicolumn{1}{|c|}{22} & \multicolumn{1}{c|}{0.350} & \multicolumn{1}{c|}{0.350} & \multicolumn{1}{c|}{0.350} & \multicolumn{1}{c|}{0.350} & \multicolumn{1}{c|}{0.350} & \multicolumn{1}{c|}{0.126} \\ 
\hline
\multicolumn{1}{|c|}{23} & \multicolumn{1}{c|}{0.333} & \multicolumn{1}{c|}{0.333} & \multicolumn{1}{c|}{0.333} & \multicolumn{1}{c|}{0.333} & \multicolumn{1}{c|}{0.333} & \multicolumn{1}{c|}{0.111} \\ 
\hline
\multicolumn{1}{|c|}{24} & \multicolumn{1}{c|}{0.317} & \multicolumn{1}{c|}{0.317} & \multicolumn{1}{c|}{0.317} & \multicolumn{1}{c|}{0.317} & \multicolumn{1}{c|}{0.317} & \multicolumn{1}{c|}{0.097} \\ 
\hline
\multicolumn{1}{|c|}{25} & \multicolumn{1}{c|}{0.301} & \multicolumn{1}{c|}{0.301} & \multicolumn{1}{c|}{0.301} & \multicolumn{1}{c|}{0.301} & \multicolumn{1}{c|}{0.301} & \multicolumn{1}{c|}{0.085} \\ 
\hline
\multicolumn{1}{|c|}{26} & \multicolumn{1}{c|}{0.287} & \multicolumn{1}{c|}{0.287} & \multicolumn{1}{c|}{0.287} & \multicolumn{1}{c|}{0.287} & \multicolumn{1}{c|}{0.287} & \multicolumn{1}{c|}{0.074} \\ 
\hline
\multicolumn{1}{|c|}{27} & \multicolumn{1}{c|}{0.273} & \multicolumn{1}{c|}{0.273} & \multicolumn{1}{c|}{0.273} & \multicolumn{1}{c|}{0.273} & \multicolumn{1}{c|}{0.273} & \multicolumn{1}{c|}{0.064} \\ 
\hline
\multicolumn{1}{|c|}{28} & \multicolumn{1}{c|}{0.259} & \multicolumn{1}{c|}{0.259} & \multicolumn{1}{c|}{0.259} & \multicolumn{1}{c|}{0.259} & \multicolumn{1}{c|}{0.259} & \multicolumn{1}{c|}{0.056} \\ 
\hline
\multicolumn{1}{|c|}{29} & \multicolumn{1}{c|}{0.247} & \multicolumn{1}{c|}{0.247} & \multicolumn{1}{c|}{0.247} & \multicolumn{1}{c|}{0.247} & \multicolumn{1}{c|}{0.247} & \multicolumn{1}{c|}{0.049} \\ 
\hline
\multicolumn{1}{|c|}{30} & \multicolumn{1}{c|}{0.235} & \multicolumn{1}{c|}{0.235} & \multicolumn{1}{c|}{0.235} & \multicolumn{1}{c|}{0.235} & \multicolumn{1}{c|}{0.235} & \multicolumn{1}{c|}{0.043} \\ 
\hline
\end{tabular}
\label{tblavaimain:1}
\end{table}


\begin{figure}[h]
% \begin{center}
\includegraphics[width=415pt]{avaimain-3.eps}
\caption{Reliability and availability}\label{avaimain:3}
% \end{center}
\end{figure}
\subsubsection{Answer D}
The function form for calculating the availability is:
\begin{eqnarray}
&& \begin{array}{l}
{A_{network}}(t) = {A_1}(t) \cdot {A_3}(t) \cdot {A_4}(t) + {A_1}(t) \cdot
{A_3}(t) \cdot {A_5}(t) + {A_2}(t) \cdot {A_3}(t) \cdot {A_4}(t) + {A_2}(t) \cdot
{A_3}(t) \cdot {A_5}(t)\\
+ {A_1}(t) \cdot {A_2}(t) \cdot {A_4}(t) \cdot {A_5}(t){\rm{ }} - {A_1}(t)
\cdot {A_2}(t) \cdot {A_3}(t) \cdot {A_4}(t)\\
- {A_1}(t) \cdot {A_3}(t) \cdot {A_4}(t) \cdot {A_5}(t) - {A_1}(t) \cdot
{A_2}(t) \cdot {A_3}(t) \cdot {A_5}(t) - {A_2}(t) \cdot {A_3}(t) \cdot {A_4}(t)
\cdot {A_5}(t)
\end{array}
\label{avaimaineq:6}
%(9)
\end{eqnarray}
where ${A_i}(t)$is the availability of sub-item \textit{i}.

The availability of each sub-item is calculated with equation \eqref{avaimaineq:1}, in which, it
is necessary to calculate the mean time to repair or the mean time (duration) the
sub-item is in adequate level of service. This calculation involves the value of
reliability for each sub-item calculated in previous step.

\begin{eqnarray}
E(t_i^n) = {\Delta _i} \cdot {F_i}(t) = {\Delta _i} \cdot \left[ {1 - {R_i}(t)}
\right]
\label{avaimaineq:7}
%(10)
\end{eqnarray}


In equation \eqref{avaimaineq:7}, ${F_i}(t)$is the failure probability at time \textit{t}.

The availability of each sub-item in year \textit{t} will be
\begin{eqnarray}
A(t_i^{}) = \frac{{E\left[ {t_i^a} \right]}}{{E\left[ {t_i^a} \right] + E\left[
{t_i^n} \right]}} = \frac{{\int\limits_0^{tu} {{R_i}(t)dt} }}{{\int\limits_0^{tu}
{{R_i}(t)dt}  + {\Delta _i} \cdot \left[ {1 - {R_i}(tu)} \right]}}
\label{avaimaineq:8}
%(11)
\end{eqnarray}
The availability of each item and the network in 30 years.

Using equations \eqref{eqreliability:7} the availability of each sub-item, and the item, for periods
of time up to 30 years can be calculated (Table \ref{tblavaimain:2}). For example, the
availability of sub-item 1 for a 14 year time period is 0.999090421, and the
availability of the item for a 14 year time period is 0.999947505. This means
that on average over periods of time that are 14 years in length that one can
expect that sub-item 1 would provide an adequate LOS for 13.9872 years and an
inadequate LOS for 0.01273411 years (4.647949 days), and that the item would
provide an adequate LOS for 13.99927 years and an inadequate LOS for 0.00073
years (0.26645 days). The longer the period of time investigated the lower the
availability because the reliability of the sub-items decreases with age, when
there is a constant failure rate. This is illustrated in Figure \ref{avaimain:3}.

\begin{table}[h]
\caption{Availability}
\begin{tabular}{|l|l|l|l|l|l|l|}
\hline
\multicolumn{1}{|c|}{Time} & \multicolumn{6}{c|}{Availability} \\ 
\cline{2-7}
\multicolumn{1}{|c|}{(years)} & \multicolumn{1}{c|}{1} & \multicolumn{1}{c|}{2} & \multicolumn{1}{c|}{3} & \multicolumn{1}{c|}{4} & \multicolumn{1}{c|}{5} & \multicolumn{1}{c|}{Item} \\ 
\hline
\multicolumn{1}{|c|}{1} & \multicolumn{1}{c|}{1} & \multicolumn{1}{c|}{1} & \multicolumn{1}{c|}{1} & \multicolumn{1}{c|}{1} & \multicolumn{1}{c|}{1} & \multicolumn{1}{c|}{1} \\ 
\hline
\multicolumn{1}{|c|}{2} & \multicolumn{1}{c|}{0.999509} & \multicolumn{1}{c|}{0.999018} & \multicolumn{1}{c|}{0.997549} & \multicolumn{1}{c|}{0.998038} & \multicolumn{1}{c|}{0.998528} & \multicolumn{1}{c|}{0.999985} \\ 
\hline
\multicolumn{1}{|c|}{3} & \multicolumn{1}{c|}{0.999345} & \multicolumn{1}{c|}{0.998691} & \multicolumn{1}{c|}{0.996735} & \multicolumn{1}{c|}{0.997386} & \multicolumn{1}{c|}{0.998039} & \multicolumn{1}{c|}{0.999973} \\ 
\hline
\multicolumn{1}{|c|}{4} & \multicolumn{1}{c|}{0.999264} & \multicolumn{1}{c|}{0.998528} & \multicolumn{1}{c|}{0.996329} & \multicolumn{1}{c|}{0.997061} & \multicolumn{1}{c|}{0.997794} & \multicolumn{1}{c|}{0.999966} \\ 
\hline
\multicolumn{1}{|c|}{5} & \multicolumn{1}{c|}{0.999215} & \multicolumn{1}{c|}{0.998431} & \multicolumn{1}{c|}{0.996086} & \multicolumn{1}{c|}{0.996867} & \multicolumn{1}{c|}{0.997648} & \multicolumn{1}{c|}{0.999961} \\ 
\hline
\multicolumn{1}{|c|}{6} & \multicolumn{1}{c|}{0.999182} & \multicolumn{1}{c|}{0.998366} & \multicolumn{1}{c|}{0.995925} & \multicolumn{1}{c|}{0.996737} & \multicolumn{1}{c|}{0.997551} & \multicolumn{1}{c|}{0.999958} \\ 
\hline
\multicolumn{1}{|c|}{7} & \multicolumn{1}{c|}{0.999159} & \multicolumn{1}{c|}{0.99832} & \multicolumn{1}{c|}{0.995810} & \multicolumn{1}{c|}{0.996645} & \multicolumn{1}{c|}{0.997482} & \multicolumn{1}{c|}{0.999955} \\ 
\hline
\multicolumn{1}{|c|}{8} & \multicolumn{1}{c|}{0.999142} & \multicolumn{1}{c|}{0.998285} & \multicolumn{1}{c|}{0.995724} & \multicolumn{1}{c|}{0.996576} & \multicolumn{1}{c|}{0.99743} & \multicolumn{1}{c|}{0.999953} \\ 
\hline
\multicolumn{1}{|c|}{9} & \multicolumn{1}{c|}{0.999128} & \multicolumn{1}{c|}{0.998258} & \multicolumn{1}{c|}{0.995657} & \multicolumn{1}{c|}{0.996523} & \multicolumn{1}{c|}{0.99739} & \multicolumn{1}{c|}{0.999952} \\ 
\hline
\multicolumn{1}{|c|}{10} & \multicolumn{1}{c|}{0.999118} & \multicolumn{1}{c|}{0.998237} & \multicolumn{1}{c|}{0.995604} & \multicolumn{1}{c|}{0.99648} & \multicolumn{1}{c|}{0.997358} & \multicolumn{1}{c|}{0.999951} \\ 
\hline
\multicolumn{1}{|c|}{11} & \multicolumn{1}{c|}{0.999109} & \multicolumn{1}{c|}{0.998219} & \multicolumn{1}{c|}{0.995561} & \multicolumn{1}{c|}{0.996445} & \multicolumn{1}{c|}{0.997332} & \multicolumn{1}{c|}{0.99995} \\ 
\hline
\multicolumn{1}{|c|}{12} & \multicolumn{1}{c|}{0.999102} & \multicolumn{1}{c|}{0.998205} & \multicolumn{1}{c|}{0.995525} & \multicolumn{1}{c|}{0.996416} & \multicolumn{1}{c|}{0.99731} & \multicolumn{1}{c|}{0.999949} \\ 
\hline
\multicolumn{1}{|c|}{13} & \multicolumn{1}{c|}{0.999096} & \multicolumn{1}{c|}{0.998193} & \multicolumn{1}{c|}{0.995494} & \multicolumn{1}{c|}{0.996392} & \multicolumn{1}{c|}{0.997292} & \multicolumn{1}{c|}{0.999948} \\ 
\hline
\multicolumn{1}{|c|}{14} & \multicolumn{1}{c|}{0.99909} & \multicolumn{1}{c|}{0.998182} & \multicolumn{1}{c|}{0.995469} & \multicolumn{1}{c|}{0.996372} & \multicolumn{1}{c|}{0.997276} & \multicolumn{1}{c|}{0.999948} \\ 
\hline
\multicolumn{1}{|c|}{15} & \multicolumn{1}{c|}{0.999086} & \multicolumn{1}{c|}{0.998174} & \multicolumn{1}{c|}{0.995446} & \multicolumn{1}{c|}{0.996354} & \multicolumn{1}{c|}{0.997263} & \multicolumn{1}{c|}{0.999947} \\ 
\hline
\multicolumn{1}{|c|}{16} & \multicolumn{1}{c|}{0.999082} & \multicolumn{1}{c|}{0.998166} & \multicolumn{1}{c|}{0.995427} & \multicolumn{1}{c|}{0.996338} & \multicolumn{1}{c|}{0.997251} & \multicolumn{1}{c|}{0.999947} \\ 
\hline
\multicolumn{1}{|c|}{17} & \multicolumn{1}{c|}{0.999079} & \multicolumn{1}{c|}{0.998159} & \multicolumn{1}{c|}{0.995410} & \multicolumn{1}{c|}{0.996325} & \multicolumn{1}{c|}{0.997241} & \multicolumn{1}{c|}{0.999946} \\ 
\hline
\multicolumn{1}{|c|}{18} & \multicolumn{1}{c|}{0.999076} & \multicolumn{1}{c|}{0.998153} & \multicolumn{1}{c|}{0.995395} & \multicolumn{1}{c|}{0.996313} & \multicolumn{1}{c|}{0.997232} & \multicolumn{1}{c|}{0.999946} \\ 
\hline
\multicolumn{1}{|c|}{19} & \multicolumn{1}{c|}{0.999073} & \multicolumn{1}{c|}{0.998148} & \multicolumn{1}{c|}{0.995382} & \multicolumn{1}{c|}{0.996302} & \multicolumn{1}{c|}{0.997224} & \multicolumn{1}{c|}{0.999945} \\ 
\hline
\multicolumn{1}{|c|}{20} & \multicolumn{1}{c|}{0.999071} & \multicolumn{1}{c|}{0.998143} & \multicolumn{1}{c|}{0.99537} & \multicolumn{1}{c|}{0.996293} & \multicolumn{1}{c|}{0.997217} & \multicolumn{1}{c|}{0.999945} \\ 
\hline
\multicolumn{1}{|c|}{21} & \multicolumn{1}{c|}{0.999068} & \multicolumn{1}{c|}{0.998139} & \multicolumn{1}{c|}{0.995359} & \multicolumn{1}{c|}{0.996284} & \multicolumn{1}{c|}{0.99721} & \multicolumn{1}{c|}{0.999945} \\ 
\hline
\multicolumn{1}{|c|}{22} & \multicolumn{1}{c|}{0.999066} & \multicolumn{1}{c|}{0.998135} & \multicolumn{1}{c|}{0.99535} & \multicolumn{1}{c|}{0.996276} & \multicolumn{1}{c|}{0.997205} & \multicolumn{1}{c|}{0.999945} \\ 
\hline
\multicolumn{1}{|c|}{23} & \multicolumn{1}{c|}{0.999065} & \multicolumn{1}{c|}{0.998131} & \multicolumn{1}{c|}{0.995341} & \multicolumn{1}{c|}{0.996269} & \multicolumn{1}{c|}{0.997199} & \multicolumn{1}{c|}{0.999945} \\ 
\hline
\multicolumn{1}{|c|}{24} & \multicolumn{1}{c|}{0.999063} & \multicolumn{1}{c|}{0.998128} & \multicolumn{1}{c|}{0.995333} & \multicolumn{1}{c|}{0.996263} & \multicolumn{1}{c|}{0.997195} & \multicolumn{1}{c|}{0.999944} \\ 
\hline
\multicolumn{1}{|c|}{25} & \multicolumn{1}{c|}{0.999062} & \multicolumn{1}{c|}{0.998125} & \multicolumn{1}{c|}{0.995326} & \multicolumn{1}{c|}{0.996257} & \multicolumn{1}{c|}{0.99719} & \multicolumn{1}{c|}{0.999944} \\ 
\hline
\multicolumn{1}{|c|}{26} & \multicolumn{1}{c|}{0.99906} & \multicolumn{1}{c|}{0.998123} & \multicolumn{1}{c|}{0.99532} & \multicolumn{1}{c|}{0.996252} & \multicolumn{1}{c|}{0.997186} & \multicolumn{1}{c|}{0.999944} \\ 
\hline
\multicolumn{1}{|c|}{27} & \multicolumn{1}{c|}{0.999059} & \multicolumn{1}{c|}{0.99812} & \multicolumn{1}{c|}{0.995314} & \multicolumn{1}{c|}{0.996247} & \multicolumn{1}{c|}{0.997183} & \multicolumn{1}{c|}{0.999944} \\ 
\hline
\multicolumn{1}{|c|}{28} & \multicolumn{1}{c|}{0.999058} & \multicolumn{1}{c|}{0.998118} & \multicolumn{1}{c|}{0.995308} & \multicolumn{1}{c|}{0.996243} & \multicolumn{1}{c|}{0.99718} & \multicolumn{1}{c|}{0.999944} \\ 
\hline
\multicolumn{1}{|c|}{29} & \multicolumn{1}{c|}{0.999057} & \multicolumn{1}{c|}{0.998116} & \multicolumn{1}{c|}{0.995303} & \multicolumn{1}{c|}{0.996239} & \multicolumn{1}{c|}{0.997176} & \multicolumn{1}{c|}{0.999944} \\ 
\hline
\multicolumn{1}{|c|}{30} & \multicolumn{1}{c|}{0.999056} & \multicolumn{1}{c|}{0.998114} & \multicolumn{1}{c|}{0.995298} & \multicolumn{1}{c|}{0.996235} & \multicolumn{1}{c|}{0.997174} & \multicolumn{1}{c|}{0.999943} \\ 
\hline
\end{tabular}
\label{tblavaimain:2}
\end{table}

The code to compute this example is given in the Appendix \ref{appavai-mai1}
\section{Trade-offs}
Indicators of performance are just that, indicators. One can minimize the amount
of time required to execute interventions on an item, i.e. maximize
maintainability, by constructing an expensive but easy to repair item. This would
most likely increase the availability of the item but may have little effect on
the reliability of the item.

One can maximize the reliability of an item by constructing an expensive but
difficult to repair item. This would most likely decrease the maintainability of
the item but may have little effect on the availability of the item.

Once can maximize the availability of an item by constructing an expensive item
that is fairly reliable and fairly easy to repair.

In each of these cases, however, the most important question for the
infrastructure manager is whether the improvements in reliability, availability
or maintainability are worth the money spent. For example, if a road has very
little traffic it may not be worthwhile to spend a large amount of money to
shorten the durations of interventions, to increase the reliability of the bridge
in the road, or to improve the availability of the road.
\subsection{Example}
You are the manager of a tunnel that is considering the construction of tunnel
with five possible configurations. Each configuration is comprised of five
components, but three of the components can be one of two different types. The
configurations to be considered, as well as the mean time between failures for
each component in each configuration, the initial cost of each configuration, and
the mean intervention duration are given in Table \ref{tblavaimain:3}.
\begin{table}[h]
\caption{Tunnel components}
\begin{tabular}{|l|l|l|l|l|l|l|}
\hline
\multicolumn{2}{|c|}{Components} & \multicolumn{1}{c|}{Option 1} & \multicolumn{1}{c|}{Option 2} & \multicolumn{1}{c|}{Option 3} & \multicolumn{1}{c|}{Option 4} & \multicolumn{1}{c|}{Option 5} \\ 
\hline
\multicolumn{1}{|c|}{No.} & Description & \multicolumn{5}{c|}{mean time between failures (years)} \\ 
\hline
\multicolumn{1}{|c|}{1} & a power supply system & \multicolumn{1}{c|}{245} & \multicolumn{1}{c|}{9'612} & \multicolumn{1}{c|}{245} & \multicolumn{1}{c|}{9'612} & \multicolumn{1}{c|}{9'612} \\ 
\hline
\multicolumn{1}{|c|}{2} & a ventilation system & \multicolumn{1}{c|}{4'819} & \multicolumn{1}{c|}{4'819} & \multicolumn{1}{c|}{36'955} & \multicolumn{1}{c|}{4'819} & \multicolumn{1}{c|}{36'955} \\ 
\hline
\multicolumn{1}{|c|}{3} & a fire services system & \multicolumn{1}{c|}{1.1x106} & \multicolumn{1}{c|}{1.1x106} & \multicolumn{1}{c|}{1.1x106} & \multicolumn{1}{c|}{1.1x106} & \multicolumn{1}{c|}{1.1x106} \\ 
\hline
\multicolumn{1}{|c|}{4} & a drainage system & \multicolumn{1}{c|}{1} & \multicolumn{1}{c|}{1} & \multicolumn{1}{c|}{1} & \multicolumn{1}{c|}{2'194} & \multicolumn{1}{c|}{2'194} \\ 
\hline
\multicolumn{1}{|c|}{5} & a central monitoring and control system & \multicolumn{1}{c|}{2.4x105} & \multicolumn{1}{c|}{2.4x105} & \multicolumn{1}{c|}{2.4x105} & \multicolumn{1}{c|}{2.4x105} & \multicolumn{1}{c|}{2.4x105} \\ 
\hline
\multicolumn{1}{|l}{} & Initial cost (mus) & \multicolumn{1}{c|}{0} & \multicolumn{1}{c|}{2} & \multicolumn{1}{c|}{4} & \multicolumn{1}{c|}{6} & \multicolumn{1}{c|}{12} \\ 
\hline
\multicolumn{1}{|l}{} & Mean intervention duration (tu) & \multicolumn{1}{c|}{0.25} & \multicolumn{1}{c|}{0.28} & \multicolumn{1}{c|}{0.3} & \multicolumn{1}{c|}{1.5} & \multicolumn{1}{c|}{2} \\ 
\hline
\end{tabular}
\label{tblavaimain:3}
\end{table}
\subsubsection{Question}
Which configuration has the highest reliability? The highest maintainability?
The highest availability? And which configuration results in the lowest overall
net costs over a 20 \textit{tu} time period?
\subsubsection{Answer}
\paragraph{Reliability}
In the estimation of the reliability of the tunnel item one first has to
recognize that the 5 sub-items are conceptually connected in series, i.e. if one
sub-item fails the item fails. The next step is to calculate the failure rate of
each component, which if it is assumed that the failures of each component
follows can be modelled using an exponential distribution, is 1/the mean time
between failures. These are shown in Table \ref{tblavaimain:4}.
${\theta _1}$ ${\theta _2}$ ${\theta _3}$ ${\theta _4}$ ${\theta _5}$
The mean time between failures for the tunnel is then 1 over the sum of the
failure rates as shown in equation \eqref{avaimaineq:9}:
\begin{eqnarray}
&& \frac{1}{{\sum\limits_i^I {{\theta _i}} }}
\label{avaimaineq:9}
%(12)
\end{eqnarray}
Where i is the index of tunnel components.

And the reliability of the tunnel in one year is given by:
\begin{eqnarray}
&& R = \exp \left[ { - \left( {\sum\limits_i^I {{\theta _i}} } \right) \cdot 1}
\right]
\label{avaimaineq:10}
%(13)
\end{eqnarray}
These values for each of the options are shown in Table \ref{tblavaimain:4}. It can be
seen that option 5 is the most reliable, although not significantly more reliable
than option 4.
\begin{table}[h]
\caption{Failure rates of tunnel components, and mean time between failure and reliability of the tunnel}
\begin{tabular}{|l|l|l|l|l|l|l|l|l|}
\hline
\multicolumn{1}{|c|}{Option} & \multicolumn{1}{c|}{Item} & \multicolumn{5}{c|}{Components} & \multicolumn{1}{m{1.2cm}|}{MTBF of Tunnel} & \multicolumn{1}{c|}{Reliability} \\ 
\cline{3-7}
\multicolumn{1}{|c|}{} & \multicolumn{1}{c|}{} & \multicolumn{1}{m{1.2cm}|}{Power Supply System} & \multicolumn{1}{m{1.2cm}|}{Ventilation System} & \multicolumn{1}{m{1.2cm}|}{Fire Services System} & \multicolumn{1}{m{1.2cm}|}{Drainage System} & \multicolumn{1}{m{1.5cm}|}{Central monitoring and control system} & \multicolumn{1}{c|}{} & \multicolumn{1}{c|}{} \\ 
\hline
\multicolumn{1}{|c|}{1} & \multicolumn{1}{c|}{MTBF (years)} & \multicolumn{1}{c|}{245} & \multicolumn{1}{c|}{4'819} & \multicolumn{1}{c|}{1.10E+06} & \multicolumn{1}{c|}{1} & \multicolumn{1}{c|}{2.40E+05} & \multicolumn{1}{c|}{1} & \multicolumn{1}{c|}{0.3663} \\ 
\cline{2-7}
\multicolumn{1}{|c|}{} & \multicolumn{1}{c|}{$\theta_1$} & \multicolumn{1}{c|}{4.08E-03} & \multicolumn{1}{c|}{2.08E-04} & \multicolumn{1}{c|}{9.09E-07} & \multicolumn{1}{c|}{1.00E+00} & \multicolumn{1}{c|}{4.17E-06} & \multicolumn{1}{c|}{} & \multicolumn{1}{c|}{} \\ 
\hline
\multicolumn{1}{|c|}{2} & \multicolumn{1}{c|}{MTBF (years)} & \multicolumn{1}{c|}{9'612} & \multicolumn{1}{c|}{4'819} & \multicolumn{1}{c|}{1.10E+06} & \multicolumn{1}{c|}{1} & \multicolumn{1}{c|}{2.40E+05} & \multicolumn{1}{c|}{1} & \multicolumn{1}{c|}{0.3678} \\ 
\cline{2-7}
\multicolumn{1}{|c|}{} & \multicolumn{1}{c|}{$\theta_2$} & \multicolumn{1}{c|}{1.04E-04} & \multicolumn{1}{c|}{2.08E-04} & \multicolumn{1}{c|}{9.09E-07} & \multicolumn{1}{c|}{1.00E+00} & \multicolumn{1}{c|}{4.17E-06} & \multicolumn{1}{c|}{} & \multicolumn{1}{c|}{} \\ 
\hline
\multicolumn{1}{|c|}{3} & \multicolumn{1}{c|}{MTBF (years)} & \multicolumn{1}{c|}{245} & \multicolumn{1}{c|}{36'955} & \multicolumn{1}{c|}{1.10E+06} & \multicolumn{1}{c|}{1} & \multicolumn{1}{c|}{2.40E+05} & \multicolumn{1}{c|}{1} & \multicolumn{1}{c|}{0.3664} \\ 
\cline{2-7}
\multicolumn{1}{|c|}{} & \multicolumn{1}{c|}{$\theta_3$} & \multicolumn{1}{c|}{4.08E-03} & \multicolumn{1}{c|}{2.71E-05} & \multicolumn{1}{c|}{9.09E-07} & \multicolumn{1}{c|}{1.00E+00} & \multicolumn{1}{c|}{4.17E-06} & \multicolumn{1}{c|}{} & \multicolumn{1}{c|}{} \\ 
\hline
\multicolumn{1}{|c|}{4} & \multicolumn{1}{c|}{MTBF (years)} & \multicolumn{1}{c|}{9'612} & \multicolumn{1}{c|}{4'819} & \multicolumn{1}{c|}{1.10E+06} & \multicolumn{1}{c|}{2'194} & \multicolumn{1}{c|}{2.40E+05} & \multicolumn{1}{c|}{1'295} & \multicolumn{1}{c|}{0.9992} \\ 
\cline{2-7}
\multicolumn{1}{|c|}{} & \multicolumn{1}{c|}{$\theta_4$} & \multicolumn{1}{c|}{1.04E-04} & \multicolumn{1}{c|}{2.08E-04} & \multicolumn{1}{c|}{9.09E-07} & \multicolumn{1}{c|}{4.56E-04} & \multicolumn{1}{c|}{4.17E-06} & \multicolumn{1}{c|}{} & \multicolumn{1}{c|}{} \\ 
\hline
\multicolumn{1}{|c|}{5} & \multicolumn{1}{c|}{MTBF (years)} & \multicolumn{1}{c|}{9'612} & \multicolumn{1}{c|}{36'955} & \multicolumn{1}{c|}{1.10E+06} & \multicolumn{1}{c|}{2'194} & \multicolumn{1}{c|}{2.40E+05} & \multicolumn{1}{c|}{1'689} & \multicolumn{1}{c|}{0.9994} \\ 
\cline{2-7}
\multicolumn{1}{|c|}{} & \multicolumn{1}{c|}{$\theta_5$} & \multicolumn{1}{c|}{1.04E-004} & \multicolumn{1}{c|}{2.71E-005} & \multicolumn{1}{c|}{9.09E-007} & \multicolumn{1}{c|}{4.56E-004} & \multicolumn{1}{c|}{4.17E-006} & \multicolumn{1}{c|}{} & \multicolumn{1}{c|}{} \\ 
\hline
\end{tabular}
\label{tblavaimain:4}
\end{table}
\paragraph{Maintainability}
In order to calculate the maintainability we can look simply at the average
length of time it will take to restore each configuration when it fails. It is
here clear that the configuration 1, which has an average down time of 0.25
months if it fails is the most maintainable. This can be seen in Table
\ref{tblavaimain:5}.
\begin{table}[h]
\caption{Initial cost vs reliability, maintainability, availability and net benefit of the five
configurations}
\begin{tabular}{|l|l|l|l|l|}
\hline
\multicolumn{1}{|c|}{Configuration} & \multicolumn{1}{c|}{Initial Cost} & \multicolumn{1}{c|}{Reliability} & \multicolumn{1}{c|}{Maintainability} & \multicolumn{1}{c|}{Availability} \\ 
\multicolumn{1}{|c|}{} & \multicolumn{1}{c|}{(mus)} & \multicolumn{1}{c|}{} & \multicolumn{1}{c|}{(months)} & \multicolumn{1}{c|}{} \\ 
\hline
\multicolumn{1}{|c|}{1} & \multicolumn{1}{c|}{0} & \multicolumn{1}{c|}{0.3663} & \multicolumn{1}{c|}{\cellcolor{blue!25} 0.25} & \multicolumn{1}{c|}{0.979506} \\ 
\hline
\multicolumn{1}{|c|}{2} & \multicolumn{1}{c|}{2} & \multicolumn{1}{c|}{0.3678} & \multicolumn{1}{c|}{0.28} & \multicolumn{1}{c|}{0.97759} \\ 
\hline
\multicolumn{1}{|c|}{3} & \multicolumn{1}{c|}{4} & \multicolumn{1}{c|}{0.3664} & \multicolumn{1}{c|}{0.3} & \multicolumn{1}{c|}{0.975512} \\ 
\hline
\multicolumn{1}{|c|}{4} & \multicolumn{1}{c|}{6} & \multicolumn{1}{c|}{0.9992} & \multicolumn{1}{c|}{1.5} & \multicolumn{1}{c|}{\cellcolor{blue!25}0.999903} \\ 
\hline
\multicolumn{1}{|c|}{5} & \multicolumn{1}{c|}{12} & \multicolumn{1}{c|}{\cellcolor{blue!25} 0.9994} & \multicolumn{1}{c|}{2} & \multicolumn{1}{c|}{0.999901} \\ 
\hline
\end{tabular}
\label{tblavaimain:5}
\end{table}
\paragraph{Availability}
The availability of each configuration is given by taking the mean time between
failures, i.e. the mean time the item works, and dividing it by the average time
it takes to restore the configuration if it fails plus the mean time it works,
i.e. the average amount of time from starting to providing an adequate LOS to
providing an adequate LOS again. One can also refer to this as the renewal
period. The configuration with the highest availability is configuration 4 (Table
\ref{tblavaimain:5}).

\paragraph{Costs}
The net costs are the initial costs plus the costs incurred over the 20 year
period due to the costs due to lost service when an inadequate LOS is provided,
which in this case are considered to include the cost of interventions when
executed. Assuming that these costs are directly and uniformly related to the
amount of time required to restore service, the annual maintenance costs for each
configuration are given by:
\begin{eqnarray}
&& {C_{am}} = \frac{{E\left[ {t_i^n} \right] \cdot {I_{cr}}/12}}{{E\left[ {t_i^a}
\right] + E\left[ {t_i^n} \right]}}
%(13)
\end{eqnarray}
Where \textit{I$_{cr}$} represents the cost of executing an intervention
multiplied by the probability of executing an intervention

The annual maintenance costs as well as the net benefit excluding upfront costs
and including upfront costs are shown for each configuration in Table
\ref{tblavaimain:6}. The net benefit of a configuration is the cost of configuration 1
minus the costs of the configuration being investigated. It can be seen that
configuration has the highest net benefit over the 20 \textit{tu} period.

\begin{eqnarray}
&& N{B^i} = NB_c^i + NB_m^i\,\,\,\,\,\,\,\,\,\, = \left( {C_c^r - C_c^i} \right) +
\left( {C_m^r - C_m^i} \right)
\label{avaimaineq:11}
%(15)
\end{eqnarray}
\begin{adjustwidth}{1cm}{}
\begin{description}
\item[$N{B^i}$:] is the net benefit of configuration i,
\item[$C_m^r$:] is the maintenance costs of the reference configuration
\item[$C_m^i$:] is the maintenance costs of configuration i
\end{description}
%\end{flushright}
\end{adjustwidth}

\begin{table}[h]
\caption{ Initial cost versus total costs of the five configurations}
\begin{tabular}{|l|l|l|l|l|l|l|l|}
\hline
\multicolumn{1}{|c|}{Configuration} & \multicolumn{5}{c|}{Cost} & \multicolumn{1}{c|}{Net benefit} & \multicolumn{1}{c|}{Net benefit} \\ 
\cline{2-6}
\multicolumn{1}{|c|}{} & \multicolumn{1}{c|}{initial} & \multicolumn{1}{c|}{ Cost} & \multicolumn{1}{c|}{Cost} & \multicolumn{1}{c|}{Cost per tu} & \multicolumn{1}{c|}{Cost for 20 tus} & \multicolumn{1}{c|}{excluding} & \multicolumn{1}{c|}{including} \\ 
\multicolumn{1}{|c|}{} & \multicolumn{1}{c|}{(mus)} & \multicolumn{1}{c|}{mus/month} & \multicolumn{1}{c|}{(mus)} & \multicolumn{1}{c|}{} & \multicolumn{1}{c|}{} & \multicolumn{1}{c|}{upfront costs} & \multicolumn{1}{c|}{upfront costs} \\ 
\multicolumn{1}{|c|}{} & \multicolumn{1}{c|}{$C_c$} & \multicolumn{1}{c|}{$I_{cr}$} & \multicolumn{1}{c|}{$E[t_i^n]\cdot I_{cr}$} & \multicolumn{1}{c|}{$C_{am}$} & \multicolumn{1}{c|}{$C_{am}\cdot 20$} & \multicolumn{1}{c|}{$NB_m^i$} & \multicolumn{1}{c|}{$NB^i$} \\ 
\hline
\multicolumn{1}{|c|}{1} & \multicolumn{1}{c|}{0} & \multicolumn{1}{c|}{5} & \multicolumn{1}{c|}{1.25} & \multicolumn{1}{c|}{$7.79x10^{-1}$} & \multicolumn{1}{c|}{15.5844} & \multicolumn{1}{c|}{0} & \multicolumn{1}{c|}{0} \\ 
\hline
\multicolumn{1}{|c|}{2} & \multicolumn{1}{c|}{2} & \multicolumn{1}{c|}{5} & \multicolumn{1}{c|}{1.38} & \multicolumn{1}{c|}{$8.50x10^{-1}$} & \multicolumn{1}{c|}{17.0023} & \multicolumn{1}{c|}{-1.418} & \multicolumn{1}{c|}{-3.418} \\ 
\hline
\multicolumn{1}{|c|}{3} & \multicolumn{1}{c|}{4} & \multicolumn{1}{c|}{5} & \multicolumn{1}{c|}{1.5} & \multicolumn{1}{c|}{$9.31x10^{-1}$} & \multicolumn{1}{c|}{18.6197} & \multicolumn{1}{c|}{-3.035} & \multicolumn{1}{c|}{-7.035} \\ 
\hline
\multicolumn{1}{|c|}{4} & \multicolumn{1}{c|}{6} & \multicolumn{1}{c|}{5} & \multicolumn{1}{c|}{7.5} & \multicolumn{1}{c|}{$4.47x10^{-6}$} & \multicolumn{1}{c|}{0.0001} & \multicolumn{1}{c|}{15.584} & \multicolumn{1}{c|}{9.584} \\ 
\hline
\multicolumn{1}{|c|}{5} & \multicolumn{1}{c|}{12} & \multicolumn{1}{c|}{5} & \multicolumn{1}{c|}{10} & \multicolumn{1}{c|}{$3.50x10^{-6}$} & \multicolumn{1}{c|}{0.0001} & \multicolumn{1}{c|}{15.584} & \multicolumn{1}{c|}{3.584} \\ 
\hline
\end{tabular}
\label{tblavaimain:6}
\end{table}

\paragraph{Summary}
As can be seen in Table \ref{tblavaimain:7} that estimating the availability was the
only performance indicator that allowed the optimal configuration to be
determined. This will, however, not always be the case. This should be kept in
mind when selecting performance indicators to evaluate performance.
\begin{table}[h]
\caption{Initial cost vs reliability, maintainability, availability and net benefit of the five
configurations}
\begin{tabular}{|l|l|l|l|l|l|}
\hline
\multicolumn{1}{|c|}{Configuration} & \multicolumn{1}{c|}{Initial Cost} & \multicolumn{1}{c|}{Reliability} & \multicolumn{1}{c|}{Maintainability} & \multicolumn{1}{c|}{Availability} & \multicolumn{1}{c|}{Net benefit} \\ 
\multicolumn{1}{|c|}{} & \multicolumn{1}{c|}{(mus)} & \multicolumn{1}{c|}{} & \multicolumn{1}{c|}{(months)} & \multicolumn{1}{c|}{} & \multicolumn{1}{c|}{including upfront costs} \\ 
\hline
\multicolumn{1}{|c|}{1} & \multicolumn{1}{c|}{0} & \multicolumn{1}{c|}{0.3663} & \multicolumn{1}{c|}{\cellcolor{blue!25} 0.25} & \multicolumn{1}{c|}{0.979506} & \multicolumn{1}{c|}{0.000} \\ 
\hline
\multicolumn{1}{|c|}{2} & \multicolumn{1}{c|}{2} & \multicolumn{1}{c|}{0.3678} & \multicolumn{1}{c|}{0.28} & \multicolumn{1}{c|}{0.97759} & \multicolumn{1}{c|}{-3.418} \\ 
\hline
\multicolumn{1}{|c|}{3} & \multicolumn{1}{c|}{4} & \multicolumn{1}{c|}{0.3664} & \multicolumn{1}{c|}{0.3} & \multicolumn{1}{c|}{0.975512} & \multicolumn{1}{c|}{-7.035} \\ 
\hline
\multicolumn{1}{|c|}{4} & \multicolumn{1}{c|}{6} & \multicolumn{1}{c|}{0.9992} & \multicolumn{1}{c|}{1.5} & \multicolumn{1}{c|}{\cellcolor{blue!25} 0.999903} & \multicolumn{1}{c|}{\cellcolor{blue!25} 9.584} \\ 
\hline
\multicolumn{1}{|c|}{5} & \multicolumn{1}{c|}{12} & \multicolumn{1}{c|}{\cellcolor{blue!25} 0.9994} & \multicolumn{1}{c|}{2} & \multicolumn{1}{c|}{0.999901} & \multicolumn{1}{c|}{3.584} \\ 
\hline
\end{tabular}
\label{tblavaimain:7}
\end{table}
The R code for this example is given in Appendix \ref{appavai-mai2}

\section{Assignments}
\subsection{Problem A}
An infrastructure manager is responsible for maintaining a number of
facilities over a period of time. The length of time required for past corrective
interventions are given in Table \ref{tblavaimain:8}

\begin{table}[h]
\caption{ Corrective maintenance task times}
\begin{tabular}{|c|c|}
\hline
Task time & Frequency \\ 
\hline
11 & 2 \\ 
\hline
13 & 3 \\ 
\hline
15 & 8 \\ 
\hline
17 & 12 \\ 
\hline
19 & 12 \\ 
\hline
21 & 14 \\ 
\hline
23 & 13 \\ 
\hline
25 & 10 \\ 
\hline
27 & 10 \\ 
\hline
29 & 8 \\ 
\hline
31 & 7 \\ 
\hline
33 & 6 \\ 
\hline
35 & 5 \\ 
\hline
36 & 5 \\ 
\hline
37 & 4 \\ 
\hline
39 & 3 \\ 
\hline
41 & 2 \\ 
\hline
47 & 2 \\ 
\hline
\end{tabular}
\label{tblavaimain:8}
\end{table}
\subsubsection{Question A1}
What is the range of observations?
\subsubsection{Answer A1}
The range of observations is the difference between the smallest and the
largest value in the data set. In this respect, the range of the above data set
is
\begin{eqnarray}
&& Range = \max (data) - \min (data)
\label{eqavaimain:1}
%(1)
\end{eqnarray}
\[
Range = 14 - 2 = 12
\]
\subsubsection{Question A2s}
Using a class interval of 4, determine the number of class intervals.
Plot the data and construct a curve. What is the most likely type of distribution
is indicated by the curve?
\subsubsection{Answer A2}
With a class interval of 4, one can construct a range of observations.
This is equivalent to the class interval of 4.

\begin{table}[h]
\caption{Range of observation}
\begin{tabular}{|l|l|}
\hline
\multicolumn{1}{|c|}{Range} & \multicolumn{1}{c|}{Frequency} \\ 
\hline
\multicolumn{1}{|c|}{9-12} & \multicolumn{1}{c|}{2} \\ 
\hline
\multicolumn{1}{|c|}{13-16} & \multicolumn{1}{c|}{11} \\ 
\hline
\multicolumn{1}{|c|}{17-20} & \multicolumn{1}{c|}{24} \\ 
\hline
\multicolumn{1}{|c|}{21-24} & \multicolumn{1}{c|}{27} \\ 
\hline
\multicolumn{1}{|c|}{25-28} & \multicolumn{1}{c|}{20} \\ 
\hline
\multicolumn{1}{|c|}{29-32} & \multicolumn{1}{c|}{15} \\ 
\hline
\multicolumn{1}{|c|}{33-36} & \multicolumn{1}{c|}{16} \\ 
\hline
\multicolumn{1}{|c|}{37-40} & \multicolumn{1}{c|}{7} \\ 
\hline
\multicolumn{1}{|c|}{41-44} & \multicolumn{1}{c|}{2} \\ 
\hline
\multicolumn{1}{|c|}{45-48} & \multicolumn{1}{c|}{2} \\ 
\hline
\end{tabular}
\label{tblavaimain:9}
\end{table}

From this table, a histogram of frequency can be constructed (Figure
\ref{figavaimain-a:1})

\begin{figure}[h]
% \begin{center}
\includegraphics[width=376pt]{avaimain-a-9.eps}
\caption{Histogram of data}\label{figavaimain-a:1}
% \end{center}
\end{figure}

By observing the shape of the histogram, one can conclude that the data
is distributed close to the density of the lognormal distribution function, which
has a skew tail on the right side of the density function.

The lognormal distribution function has following properties

Probability density function
\begin{eqnarray}
&& {f_X}(x,\mu ,\sigma ) = \frac{1}{{x \cdot \sigma \sqrt {2\pi } }}{e^{ -
\frac{{{{(\ln x - \mu )}^2}}}{{2{\sigma ^2}}}}}{\rm{,   }}x > 0
\label{eqavaimain:2}
%(2)
\end{eqnarray}
Cumulative distribution function
\begin{eqnarray}
&& {F_X}(x,\mu ,\sigma ) = \frac{1}{2} + \frac{1}{2}erf\left[ {\frac{{(\ln x - \mu
)}}{{\sqrt {2\sigma } }}} \right]
\label{eqavaimain:3}
%(3)
\end{eqnarray}
where:
\begin{adjustwidth}{1cm}{}
\begin{description}
\item[$x$:] is the random variable of the distribution,
\item[$\mu$:] is the mean of the variable's natural logarithm,
\item[$\sigma $:] is the standard deviation of the variable's natural logarithm,
\item[$erf $:] is the error function.
\end{description}
%\end{flushright}
\end{adjustwidth}
\subsubsection{Question A3}
What is the mean length of time for a corrective intervention,
\textit{M$_{ct}$}?
\subsubsection{Answer A3}
For estimating the mean corrective intervention duration
\begin{eqnarray}
&& {\overline M _{ct}} = \frac{{\sum {\left( {{f_{c{t_i}}}} \right)\left(
{{M_{c{t_i}}}} \right)} }}{{\sum {{f_{c{t_i}}}} }}
\label{eqavaimain:4}
%(4)
\end{eqnarray}
where:
\begin{adjustwidth}{1cm}{}
\begin{description}
\item[$f_{cti}$:] is the frequency of the \textit{i$^{th}$} corrective intervention in interventions per system operating time unit,
\item[$M_{cti}$:] is the elapsed time required for the \textit{i$^{th}$} corrective intervention.
\end{description}
%\end{flushright}
\end{adjustwidth}
The answer is 25.15 days
\begin{table}[h]
\caption{Mean time of intervention duration}
\begin{tabular}{|c|c|c|c|}
\hline
Time & Frequency & Mean time & Variance \\ 
\hline
(1) & (2) & (3)=(1)*(2) & $(4)=(2)*[(1)- \bar M_{ct}]^²$ \\ 
\hline
11 & 2 & 22 & 400.49 \\ 
\hline
13 & 3 & 39 & 442.93 \\ 
\hline
15 & 8 & 120 & 824.31 \\ 
\hline
17 & 12 & 204 & 797.23 \\ 
\hline
19 & 12 & 228 & 453.99 \\ 
\hline
21 & 14 & 294 & 241.21 \\ 
\hline
23 & 13 & 299 & 60.14 \\ 
\hline
25 & 10 & 250 & 0.23 \\ 
\hline
27 & 10 & 270 & 34.2 \\ 
\hline
29 & 8 & 232 & 118.53 \\ 
\hline
31 & 7 & 217 & 239.49 \\ 
\hline
33 & 6 & 198 & 369.66 \\ 
\hline
35 & 5 & 175 & 485.03 \\ 
\hline
36 & 5 & 180 & 588.53 \\ 
\hline
37 & 4 & 148 & 561.61 \\ 
\hline
39 & 3 & 117 & 575.4 \\ 
\hline
41 & 2 & 82 & 502.39 \\ 
\hline
47 & 2 & 94 & 954.78 \\ 
\hline
\textbf{Total} & \textbf{126} & \textbf{3'169} & \textbf{7'650.13} \\ 
\hline
\end{tabular}
\label{tblavaimain:10}
\end{table}
  \[
   {\overline M _{ct}} = \frac{{3'169}}{{126}} = 25.15
  \]
\subsubsection{Question A4}
What is the geometric mean of the repair times?
\subsubsection{Answer A4}
For calculating geometric mean, refer to general statistical materials
\begin{eqnarray}
&& {\bar \mu _{geo}} = {e^\mu }
\label{eqavaimai:5}
%(4)
\end{eqnarray}
Where $\mu{}$ is the mean that is calculated based on the mean ${\overline M _{ct}}$ in this case)
\begin{eqnarray}
&& \mu  = \ln (E\left[ X \right]) - \frac{1}{2}{\sigma ^2} \label{eqavaimai:6}
\end{eqnarray}
\begin{eqnarray}
&& {\sigma ^2} = \ln \left( {1 + \frac{{Var\left[ X \right]}}{{{{(E\left[ X \right])}^2}}}} \right)
\end{eqnarray}
from data, we have
\[
Var\left[ X \right] = \sqrt {\frac{{7'650.13}}{{126}}}  = 7.79
\]
\[
E\left[ X \right] = {\bar M_{ct}} = 25.15
\]
\[
\mu  = \ln (25.15) - \frac{1}{2}\left[ {\ln \left( {1 +
\frac{{{{7.79}^2}}}{{{{25.15}^2}}}} \right)} \right] = 3.179
\]
The geometric mean is then
\[
{\bar \mu _{geo}} = {e^{3.179}} = 24.02
\]
\subsubsection{Question A5}
What is the standard deviation of the sample data?
\subsubsection{Answer A5}
The standard deviation is 7.79, which has been calculated in previous
steps.
\subsubsection{Question A6}
Assume 90\% confidence level, what is the \textit{M$_{max}$} value?
\subsubsection{Answer A6}
For obtaining the the \textit{M$_{max}$} value with assumption of a 90\%
confidence level, refer to the formulation of the cumulative density function in
equation \eqref{eqavaimain:3} of the log-normal distribution and interpolate the data.

The Mmax is 27.

\begin{figure}[h]
% \begin{center}
\includegraphics[width=396pt]{avaimain-a-10.eps}
\caption{Cumulative probability}
% \end{center}
\end{figure}
R code is given in Appendix \ref{appavai-mai3} for this assignment

\subsection{Problem B}
A transportation network connecting city A to city B is shown in Figure
\ref{figavaimain-a:2}. In order to go from city A to city B and vice versa, vehicles can
travel in both directions of every link. Links are either a road section, a
bridge or a tunnels. It is periodically affected by natural hazards such as
flood, avalanches, and rockfall. The probability of the links not providing an
adequate LOS due to the natural hazards can be modelled using the exponential or
the Weibull distributions as shown in Table \ref{tblavaimain:11}. The mean time required
to restore an adequate LOS following the occurrence of a natural hazard on each
of the links is also is included in Table \ref{tblavaimain:11}.

\begin{figure}[h]
\begin{center}
\includegraphics[width=392pt]{avaimain-a-1.eps}
\caption{A transportation network connecting city A to city B}\label{figavaimain-a:2}
\end{center}
\end{figure}

\begin{table}[h]
\caption{Characteristics of links in the network}
\begin{tabular}{|l|l|l|l|l|l|}
\hline
\multicolumn{1}{|c|}{Link object} & \multicolumn{1}{c|}{Deterioration distribution} & \multicolumn{2}{c|}{Weibul} & \multicolumn{1}{c|}{Exponential} & \multicolumn{1}{c|}{Mean time of intervention} \\ 
\cline{3-4}
\multicolumn{1}{|c|}{} & \multicolumn{1}{c|}{} & \multicolumn{1}{c|}{$\alpha$} & \multicolumn{1}{c|}{$m$} & \multicolumn{1}{c|}{$\theta$} & \multicolumn{1}{c|}{(days)} \\ 
\hline
\multicolumn{1}{|c|}{Road A-1} & \multicolumn{1}{c|}{Exponential} & \multicolumn{1}{c|}{NA} & \multicolumn{1}{c|}{NA} & \multicolumn{1}{c|}{0.024} & \multicolumn{1}{c|}{20} \\ 
\hline
\multicolumn{1}{|c|}{Road 2-3} & \multicolumn{1}{c|}{Exponential} & \multicolumn{1}{c|}{NA} & \multicolumn{1}{c|}{NA} & \multicolumn{1}{c|}{0.023} & \multicolumn{1}{c|}{15} \\ 
\hline
\multicolumn{1}{|c|}{Road 3-4} & \multicolumn{1}{c|}{Exponential} & \multicolumn{1}{c|}{NA} & \multicolumn{1}{c|}{NA} & \multicolumn{1}{c|}{0.03} & \multicolumn{1}{c|}{17} \\ 
\hline
\multicolumn{1}{|c|}{Road 3-5} & \multicolumn{1}{c|}{Exponential} & \multicolumn{1}{c|}{NA} & \multicolumn{1}{c|}{NA} & \multicolumn{1}{c|}{0.06} & \multicolumn{1}{c|}{33} \\ 
\hline
\multicolumn{1}{|c|}{Road 4-5} & \multicolumn{1}{c|}{Exponential} & \multicolumn{1}{c|}{NA} & \multicolumn{1}{c|}{NA} & \multicolumn{1}{c|}{0.04} & \multicolumn{1}{c|}{40} \\ 
\hline
\multicolumn{1}{|c|}{Bridge 1-2} & \multicolumn{1}{c|}{Weibull} & \multicolumn{1}{c|}{0.025} & \multicolumn{1}{c|}{1.5} & \multicolumn{1}{c|}{NA} & \multicolumn{1}{c|}{30} \\ 
\hline
\multicolumn{1}{|c|}{Bridge 1-3} & \multicolumn{1}{c|}{Weibull} & \multicolumn{1}{c|}{0.03} & \multicolumn{1}{c|}{2} & \multicolumn{1}{c|}{NA} & \multicolumn{1}{c|}{50} \\ 
\hline
\multicolumn{1}{|c|}{Bridge A-4} & \multicolumn{1}{c|}{Weibull} & \multicolumn{1}{c|}{0.025} & \multicolumn{1}{c|}{2.3} & \multicolumn{1}{c|}{NA} & \multicolumn{1}{c|}{70} \\ 
\hline
\multicolumn{1}{|c|}{Tunnel B-2} & \multicolumn{1}{c|}{Exponential} & \multicolumn{1}{c|}{NA} & \multicolumn{1}{c|}{NA} & \multicolumn{1}{c|}{0.04} & \multicolumn{1}{c|}{60} \\ 
\hline
\multicolumn{1}{|c|}{Tunnel B-5} & \multicolumn{1}{c|}{Exponential} & \multicolumn{1}{c|}{NA} & \multicolumn{1}{c|}{NA} & \multicolumn{1}{c|}{0.03} & \multicolumn{1}{c|}{45} \\ 
\hline
\end{tabular}
\label{tblavaimain:11}
\end{table}

The impacts incurred when each link does not provide an adequate LOS,
and all others do, are given in Table \ref{tblavaimain:12}, in terms of monetary units
(\textit{mus}) incurred per day of inadequate service

\begin{table}[h]
\caption{Impacts}
\begin{tabular}{|l|l|l|l|l|}
\hline
\multicolumn{1}{|c|}{Link} & \multicolumn{4}{c|}{Impacts (mus)} \\ 
\cline{2-5}
\multicolumn{1}{|c|}{} & \multicolumn{1}{c|}{Owner} & \multicolumn{1}{c|}{Users} & \multicolumn{1}{c|}{DAP} & \multicolumn{1}{c|}{IAP} \\ 
\hline
\multicolumn{1}{|c|}{Road A-1} & \multicolumn{1}{c|}{5} & \multicolumn{1}{c|}{6} & \multicolumn{1}{c|}{3} & \multicolumn{1}{c|}{3} \\ 
\hline
\multicolumn{1}{|c|}{Road 2-3} & \multicolumn{1}{c|}{7} & \multicolumn{1}{c|}{5} & \multicolumn{1}{c|}{4} & \multicolumn{1}{c|}{2} \\ 
\hline
\multicolumn{1}{|c|}{Road 3-4} & \multicolumn{1}{c|}{5} & \multicolumn{1}{c|}{7} & \multicolumn{1}{c|}{4} & \multicolumn{1}{c|}{3} \\ 
\hline
\multicolumn{1}{|c|}{Road 3-5} & \multicolumn{1}{c|}{4} & \multicolumn{1}{c|}{6} & \multicolumn{1}{c|}{4} & \multicolumn{1}{c|}{2} \\ 
\hline
\multicolumn{1}{|c|}{Road 4-5} & \multicolumn{1}{c|}{4} & \multicolumn{1}{c|}{4} & \multicolumn{1}{c|}{3} & \multicolumn{1}{c|}{1} \\ 
\hline
\multicolumn{1}{|c|}{Bridge 1-2} & \multicolumn{1}{c|}{7} & \multicolumn{1}{c|}{6} & \multicolumn{1}{c|}{4} & \multicolumn{1}{c|}{3} \\ 
\hline
\multicolumn{1}{|c|}{Bridge 1-3} & \multicolumn{1}{c|}{8} & \multicolumn{1}{c|}{5} & \multicolumn{1}{c|}{4} & \multicolumn{1}{c|}{2} \\ 
\hline
\multicolumn{1}{|c|}{Bridge A-4} & \multicolumn{1}{c|}{7} & \multicolumn{1}{c|}{7} & \multicolumn{1}{c|}{3} & \multicolumn{1}{c|}{1} \\ 
\hline
\multicolumn{1}{|c|}{Tunnel B-2} & \multicolumn{1}{c|}{6} & \multicolumn{1}{c|}{4} & \multicolumn{1}{c|}{4} & \multicolumn{1}{c|}{2} \\ 
\hline
\multicolumn{1}{|c|}{Tunnel B-5} & \multicolumn{1}{c|}{4} & \multicolumn{1}{c|}{3} & \multicolumn{1}{c|}{4} & \multicolumn{1}{c|}{2} \\ 
\hline
\end{tabular}\\
Note: DAP and IAP stand for directly affected public and indirectly
affected public
\label{tblavaimain:12}
\end{table}
\subsubsection{Question B1}

Estimate the reliability and availability of each link and of the entire
network for each year in 20 years.
\subsubsection{Answer B1}
\underline{Step 1:} Estimate the reliability of each link

The network can be simplified with the circles representing the links
and the connecting point representing the cities and towns. The reliability of
each link is given in Table \ref{tblavaimain:13} and illustrated in Figure \ref{figavaimain-a:3}.

\begin{figure}[h]
% \begin{center}
\includegraphics[width=386pt]{avaimain-a-2.eps}
\caption{Network simplification}
% \end{center}
\end{figure}

\begin{table}[h]
\caption{Reliability}
\begin{tabular}{|c|c|c|c|c|c|c|c|c|c|c|c|}
\hline
Time & \multicolumn{10}{c|}{Links} & Networks \\ 
\cline{2-11}
(years) & \multicolumn{5}{c|}{Road sections} & \multicolumn{3}{c|}{Bridge} & \multicolumn{2}{c|}{Tunnels} &  \\ 
\cline{2-11}
 & A-1 & 2-3 & 3-4 & 3-5 & 4-5 & 1-2 & 1-3 & A-4 & B-2 & B-5 &  \\ 
\hline
1 & 1 & 1 & 1 & 1 & 1 & 1 & 1 & 1 & 1 & 1 & 1 \\ 
\hline
2 & 0.98 & 0.98 & 0.97 & 0.94 & 0.96 & 1 & 1 & 1 & 0.96 & 0.97 & 1 \\ 
\hline
3 & 0.95 & 0.96 & 0.94 & 0.89 & 0.92 & 0.99 & 1 & 1 & 0.92 & 0.94 & 0.99 \\ 
\hline
4 & 0.93 & 0.93 & 0.91 & 0.84 & 0.89 & 0.98 & 0.99 & 1 & 0.89 & 0.91 & 0.98 \\ 
\hline
5 & 0.91 & 0.91 & 0.89 & 0.79 & 0.85 & 0.97 & 0.99 & 1 & 0.85 & 0.89 & 0.97 \\ 
\hline
6 & 0.89 & 0.89 & 0.86 & 0.74 & 0.82 & 0.96 & 0.98 & 0.99 & 0.82 & 0.86 & 0.95 \\ 
\hline
7 & 0.87 & 0.87 & 0.84 & 0.7 & 0.79 & 0.94 & 0.97 & 0.99 & 0.79 & 0.84 & 0.92 \\ 
\hline
8 & 0.85 & 0.85 & 0.81 & 0.66 & 0.76 & 0.93 & 0.96 & 0.98 & 0.76 & 0.81 & 0.89 \\ 
\hline
9 & 0.83 & 0.83 & 0.79 & 0.62 & 0.73 & 0.91 & 0.94 & 0.98 & 0.73 & 0.79 & 0.86 \\ 
\hline
10 & 0.81 & 0.81 & 0.76 & 0.58 & 0.7 & 0.9 & 0.93 & 0.97 & 0.7 & 0.76 & 0.82 \\ 
\hline
11 & 0.79 & 0.79 & 0.74 & 0.55 & 0.67 & 0.88 & 0.91 & 0.96 & 0.67 & 0.74 & 0.78 \\ 
\hline
12 & 0.77 & 0.78 & 0.72 & 0.52 & 0.64 & 0.87 & 0.9 & 0.95 & 0.64 & 0.72 & 0.74 \\ 
\hline
13 & 0.75 & 0.76 & 0.7 & 0.49 & 0.62 & 0.85 & 0.88 & 0.94 & 0.62 & 0.7 & 0.69 \\ 
\hline
14 & 0.73 & 0.74 & 0.68 & 0.46 & 0.59 & 0.83 & 0.86 & 0.93 & 0.59 & 0.68 & 0.65 \\ 
\hline
15 & 0.71 & 0.72 & 0.66 & 0.43 & 0.57 & 0.81 & 0.84 & 0.91 & 0.57 & 0.66 & 0.6 \\ 
\hline
16 & 0.7 & 0.71 & 0.64 & 0.41 & 0.55 & 0.79 & 0.82 & 0.9 & 0.55 & 0.64 & 0.56 \\ 
\hline
17 & 0.68 & 0.69 & 0.62 & 0.38 & 0.53 & 0.78 & 0.79 & 0.89 & 0.53 & 0.62 & 0.52 \\ 
\hline
18 & 0.66 & 0.68 & 0.6 & 0.36 & 0.51 & 0.76 & 0.77 & 0.87 & 0.51 & 0.6 & 0.47 \\ 
\hline
19 & 0.65 & 0.66 & 0.58 & 0.34 & 0.49 & 0.74 & 0.75 & 0.85 & 0.49 & 0.58 & 0.43 \\ 
\hline
20 & 0.63 & 0.65 & 0.57 & 0.32 & 0.47 & 0.72 & 0.72 & 0.83 & 0.47 & 0.57 & 0.39 \\ 
\hline
\end{tabular}
\label{tblavaimain:13}
\end{table}

\begin{figure}[h]
% \begin{center}
\includegraphics[width=429pt]{avaimain-a-11.eps}
\caption{Reliability of each link and network + network availability over 20
years}\label{figavaimain-a:3}
% \end{center}
\end{figure}

\underline{Step 2:} Estimate the reliability of the network

First, it is necessary to determine the structure function, which can be
calculated using the pivotal decomposition method.

First, link 1-3 is selected as the pivot. When link 1-3 is perfect, the
original network can be further simplified as shown in Figure \ref{figavaimain-a:3}, and when link
1-3 is not perfect, the original network diagram can be further simplified as
shown in Figure \ref{figavaimain-a:4}

\begin{figure}[h]
% \begin{center}
\includegraphics[width=340pt]{avaimain-a-3.eps}
\caption{Network simplification - decomposition when when ${x_{1 - 3}} = 1$}\label{figavaimain-a:3}
% \end{center}
\end{figure}

\begin{figure}[h]
% \begin{center}
\includegraphics[width=340pt]{avaimain-a-4.eps}
\caption{Network simplification - decomposition when when ${x_{1 - 3}} = 0$}\label{figavaimain-a:4}
% \end{center}
\end{figure}
The structure function of the network can then be expressed as:
\begin{eqnarray}
&& \phi (\vec x) = {x_{1 - 3}} \cdot r({1_{1 - 3}},\vec x) + (1 - {x_{1 - 3}})
\cdot r({0_{1 - 3}},\vec x)
\label{eqavaimain:5}
%(8)
\end{eqnarray}
However, even with the newly simplified network, still it is difficult
to estimate the value of the sub-structure function $r({1_{1 - 3}},\vec x)$ and
$r({0_{1 - 3}},\vec x)$directly. It is, therefore, necessary to further decompose
the simplified sub-networks.

With respect to the network $r({1_{1 - 3}},\vec x)$, the link 3-5 is
selected to be the pivot. When the link 3-5 is 100\% reliable, the network in
Figure 6 becomes the one shown in Figure \ref{figavaimain-a:5}, and when the link 3-5 is
0\% reliable, the network in Figure 6 becomes the one shown in Figure
\ref{figavaimain-a:6}.

\begin{figure}[h]
% \begin{center}
\includegraphics[width=347pt]{avaimain-a-5.eps}
\caption{Network simplification - decomposition when $x_{3-5}=1$ and $x_{1-3}=1$}\label{figavaimain-a:5}
% \end{center}
\end{figure}

\begin{figure}[h]
% \begin{center}
\includegraphics[width=374pt]{avaimain-a-6.eps}
\caption{Network simplification - decomposition when $x_{3-5}=0$ and $x_{1-3}=1$}\label{figavaimain-a:6}
% \end{center}
\end{figure}

The structure function $r({1_{1 - 3}},\vec x)$ is then:
\begin{eqnarray}
&& r({1_{1 - 3}},\vec x) = {x_{3 - 5}} \cdot r_{1 - 3}^1({1_{3 - 5}},\vec x) + (1 -
{x_{3 - 5}})r_{1 - 3}^1({0_{3 - 5}},\vec x)
\label{eqavaimain:6}
%(9)
\end{eqnarray}
where $r_{1 - 3}^1({1_{3 - 5}},\vec x)$ and $r_{1 - 3}^1({0_{3 -
5}},\vec x)$refer to the sub-structure functions after link 1-3 and link 3-5 have
been decomposed when link 1-3 is perfect.

The value of $r_{1 - 3}^1({1_{3 - 5}},\vec x)$ and $r_{1 - 3}^1({0_{3 -
5}},\vec x)$can now be directly computed as follows:
\begin{eqnarray}
&& \begin{array}{l}
r_{1 - 3}^1({1_{3 - 5}},\vec x) = \left[ {1 - (1 - {x_{A - 1}})(1 - {x_{A - 4}}
\cdot \left\{ {1 - (1 - {x_{3 - 4}}) \cdot (1 - {x_{4 - 5}})} \right\})} \right]
\cdot \\
\left[ {1 - (1 - {x_{B - 5}}) \cdot (1 - {x_{B - 2}} \cdot \left\{ {1 - (1 -
{x_{1 - 2}}) \cdot (1 - {x_{2 - 3}})} \right\}} \right]
\end{array}
\label{eqavaimain:7}
%(10)
\end{eqnarray}
\begin{eqnarray}
&& \begin{array}{l}
r_{1 - 3}^1({0_{3 - 5}},\vec x) = {x_{A - 1}}{x_{3 - 4}} \cdot {x_{B - 2}} \cdot
\left[ {1 - (1 - {x_{1 - 2}}) \cdot (1 - {x_{2 - 3}})} \right]\\
{\rm{                 }} + {x_{A - 1}}{x_{3 - 4}}{x_{4 - 5}}{x_{B - 5}}\\
{\rm{                 }} + {x_{A - 4}}{x_{3 - 4}}{x_{B - 2}} \cdot \left[ {1 -
(1 - {x_{1 - 2}}) \cdot (1 - {x_{2 - 3}})} \right]\\
{\rm{                 }} + {x_{A - 4}}{x_{3 - 4}}{x_{4 - 5}}{x_{B - 5}}\\
{\rm{                 }} + {x_{A - 1}}{x_{A - 4}}{x_{4 - 5}}{x_{B - 5}}{x_{B -
2}} \cdot \left[ {1 - (1 - {x_{1 - 2}}) \cdot (1 - {x_{2 - 3}})} \right]\\
{\rm{                 }} - {x_{A - 1}}{x_{A - 4}}{x_{3 - 4}}{x_{B - 2}} \cdot
\left[ {1 - (1 - {x_{1 - 2}}) \cdot (1 - {x_{2 - 3}})} \right]\\
{\rm{                 }} - {x_{A - 1}}{x_{3 - 4}}{x_{B - 2}} \cdot \left[ {1 -
(1 - {x_{1 - 2}}) \cdot (1 - {x_{2 - 3}})} \right]{x_{4 - 5}}{x_{B - 5}}\\
{\rm{                 }} - {x_{A - 1}}{x_{A - 4}}{x_{3 - 4}}{x_{4 - 5}}{x_{B -
5}}\\
{\rm{                 }} - {x_{A - 4}}{x_{3 - 4}}{x_{B - 2}} \cdot \left[ {1 -
(1 - {x_{1 - 2}}) \cdot (1 - {x_{2 - 3}})} \right]{x_{4 - 5}}{x_{B - 5}}
\end{array}
\label{eqavaimain:8}
%(11)
\end{eqnarray}
% Equation \eqref{eqavaimain:7} is described based on the results of structure function obtained in section 3.3 of the script of week 9.
With respect to the network $r({0_{1 - 3}},\vec x)$, the link 3-5 is
choosen to be the pivot. When link 3-5 is 100\% reliable, the network in Figure
\ref{figavaimain-a:4} becomes that shown in Figure \ref{figavaimain-a:7} and when the link 3-5 is 0\%
reliable, the network in Figure \ref{figavaimain-a:4} becomes that in Figure \ref{figavaimain-a:8}

\begin{figure}[h]
% \begin{center}
\includegraphics[width=386pt]{avaimain-a-7.eps}
\caption{Network simplification - decomposition $x_{3-5}=1$ and $x_{1-3}=0$}\label{figavaimain-a:7}
% \end{center}
\end{figure}

\begin{figure}[h]
% \begin{center}
\includegraphics[width=391pt]{avaimain-a-8.eps}
\caption{Network simplification - decomposition $x_{3-5}=0$ and $x_{1-3}=0$}\label{figavaimain-a:8}
% \end{center}
\end{figure}
The structure function $r({0_{1 - 3}},\vec x)$ is:
\begin{eqnarray}
&& r({0_{1 - 3}},\vec x) = {x_{3 - 5}} \cdot r_{1 - 3}^0({1_{3 - 5}},\vec x) + (1 -
{x_{3 - 5}}) \cdot r_{1 - 3}^0({0_{3 - 5}},\vec x)
\label{eqavaimain:9}
%(12)
\end{eqnarray}
where $r_{1 - 3}^0({1_{3 - 5}},\vec x)$ and $r_{1 - 3}^0({0_{3 -
5}},\vec x)$refer to the sub-structure function after link 1-3 and link 3-5 have
been decomposed when link 1-3 is 0\% reliable.

The values of $r_{1 - 3}^0({1_{3 - 5}},\vec x)$ and $r_{1 - 3}^0({0_{3 -
5}},\vec x)$ can now be directly computed

\begin{eqnarray}
&& \begin{array}{l}
r_{1 - 3}^0({1_{3 - 5}},\vec x) = {x_{A - 1}} \cdot {x_{1 - 2}} \cdot {x_{2 -
3}} \cdot {x_{B - 2}}\\
{\rm{                 }} + {x_{A - 1}} \cdot {x_{1 - 2}} \cdot {x_{2 - 3}} \cdot
{x_{B - 5}}\\
{\rm{                 }} + {x_{A - 4}} \cdot \left[ {1 - (1 - {x_{3 - 4}}) \cdot
(1 - {x_{4 - 5}})} \right] \cdot {x_{2 - 3}} \cdot {x_{B - 2}}\\
{\rm{                 }} + {x_{A - 4}} \cdot \left[ {1 - (1 - {x_{3 - 4}}) \cdot
(1 - {x_{4 - 5}})} \right] \cdot {x_{2 - 3}} \cdot {x_{B - 5}}\\
{\rm{                 }} + {x_{A - 1}} \cdot {x_{1 - 2}} \cdot {x_{A - 4}} \cdot
\left[ {1 - (1 - {x_{3 - 4}}) \cdot (1 - {x_{4 - 5}})} \right] \cdot {x_{B - 5}}
\cdot {x_{B - 2}}\\
{\rm{                 }} - {x_{A - 1}} \cdot {x_{1 - 2}} \cdot {x_{A - 4}} \cdot
\left[ {1 - (1 - {x_{3 - 4}}) \cdot (1 - {x_{4 - 5}})} \right] \cdot {x_{2 - 3}}
\cdot {x_{B - 2}}\\
{\rm{                 }} - {x_{A - 1}} \cdot {x_{1 - 2}} \cdot {x_{2 - 3}} \cdot
{x_{B - 2}} \cdot {x_{B - 5}}\\
{\rm{                 }} - {x_{A - 1}} \cdot {x_{1 - 2}} \cdot {x_{A - 4}} \cdot
\left[ {1 - (1 - {x_{3 - 4}}) \cdot (1 - {x_{4 - 5}})} \right] \cdot {x_{2 - 3}}
\cdot {x_{B - 5}}\\
{\rm{                 }} - {x_{A - 4}} \cdot \left[ {1 - (1 - {x_{3 - 4}}) \cdot
(1 - {x_{4 - 5}})} \right] \cdot {x_{2 - 3}} \cdot {x_{B - 5}} \cdot {x_{B - 2}}
\end{array}
\label{eqavaimain:10}
%(13)
\end{eqnarray}
The set of equation from equation \eqref{eqavaimain:5} to equation \eqref{eqavaimain:11} can be used
directly to calculate the reliability of the network given the reliability of the
individual link every year. The reliability of each link and the network for
periods of time ranging from 0 to 20 years are given in Table \ref{tblavaimain:13} and
illustrated in Figure \ref{figavaimain-a:3}. 

R code for this example is given in Appendix B.

\underline{Step 3:} Estimate the availability of the network

Similar to reliability, the set of structure functions above can be used
to compute the availability of the network based on the reliability of individual
link. The availability of individual link is calculated based on following
equation
\begin{eqnarray}
&& A(t_i^{}) = \frac{{E\left[ {t_i^a} \right]}}{{E\left[ {t_i^a} \right] + E\left[
{t_i^n} \right]}} = \frac{{\int\limits_0^{tu} {{R_i}(t)dt} }}{{\int\limits_0^{tu}
{{R_i}(t)dt}  + {\Delta _i} \cdot \left[ {1 - {R_i}(tu)} \right]}}
\label{eqavaimain:11}
%(15)
\end{eqnarray}
where ${R_i}(t)$ is the reliability of link i at time t and ${\Delta
_i}$is the mean time of intervention for link \textit{i}.

The availabilities of each link and the network for periods of time from
0 to 20 years are given in Table \ref{tblavaimain:14}, and illustrated in Figure
\ref{figavaimain-a:3}.

\begin{table}[h]
\caption{Availability}
\begin{tabular}{|c|c|c|c|c|c|c|c|c|c|c|c|}
\hline
Time & \multicolumn{10}{c|}{Links} & Networks \\ 
\cline{2-11}
(years) & \multicolumn{5}{c|}{Road sections} & \multicolumn{3}{c|}{Bridge} & \multicolumn{2}{c|}{Tunnels} &  \\ 
\cline{2-11}
 & A-1 & 2-3 & 3-4 & 3-5 & 4-5 & 1-2 & 1-3 & A-4 & B-2 & B-5 &  \\ 
\hline
1 & 1 & 1 & 1 & 1 & 1 & 1 & 1 & 1 & 1 & 1 & 1 \\ 
\hline
2 & 0.999335 & 0.999502 & 0.999431 & 0.998864 & 0.99865 & 0.999022 & 0.998376 & 0.99773 & 0.997976 & 0.998496 & 0.999995 \\ 
\hline
3 & 0.999114 & 0.999336 & 0.99924 & 0.998459 & 0.998187 & 0.998707 & 0.997859 & 0.997011 & 0.997282 & 0.99799 & 0.999992 \\ 
\hline
4 & 0.999003 & 0.999253 & 0.999142 & 0.998237 & 0.997945 & 0.998557 & 0.997615 & 0.996675 & 0.99692 & 0.997733 & 0.99999 \\ 
\hline
5 & 0.998937 & 0.999204 & 0.999083 & 0.998088 & 0.997791 & 0.998471 & 0.99748 & 0.996492 & 0.996691 & 0.997575 & 0.999988 \\ 
\hline
6 & 0.998892 & 0.999171 & 0.999042 & 0.997975 & 0.997682 & 0.998417 & 0.997397 & 0.996385 & 0.996527 & 0.997467 & 0.999987 \\ 
\hline
7 & 0.998861 & 0.999148 & 0.999012 & 0.997883 & 0.997599 & 0.998382 & 0.997344 & 0.996319 & 0.996402 & 0.997388 & 0.999986 \\ 
\hline
8 & 0.998837 & 0.999131 & 0.998988 & 0.997804 & 0.997531 & 0.998357 & 0.997308 & 0.996279 & 0.996301 & 0.997327 & 0.999986 \\ 
\hline
9 & 0.998819 & 0.999118 & 0.998969 & 0.997734 & 0.997473 & 0.998339 & 0.997284 & 0.996255 & 0.996215 & 0.997277 & 0.999985 \\ 
\hline
10 & 0.998804 & 0.999107 & 0.998954 & 0.99767 & 0.997424 & 0.998325 & 0.997266 & 0.996242 & 0.99614 & 0.997235 & 0.999985 \\ 
\hline
11 & 0.998792 & 0.999099 & 0.99894 & 0.99761 & 0.997379 & 0.998315 & 0.997254 & 0.996236 & 0.996074 & 0.9972 & 0.999984 \\ 
\hline
12 & 0.998782 & 0.999092 & 0.998929 & 0.997554 & 0.997339 & 0.998308 & 0.997244 & 0.996236 & 0.996014 & 0.997169 & 0.999984 \\ 
\hline
13 & 0.998774 & 0.999086 & 0.998918 & 0.9975 & 0.997302 & 0.998302 & 0.997236 & 0.996238 & 0.995958 & 0.997142 & 0.999984 \\ 
\hline
14 & 0.998766 & 0.999081 & 0.998909 & 0.997448 & 0.997268 & 0.998297 & 0.99723 & 0.996243 & 0.995907 & 0.997118 & 0.999983 \\ 
\hline
15 & 0.99876 & 0.999076 & 0.998901 & 0.997398 & 0.997235 & 0.998293 & 0.997225 & 0.99625 & 0.995858 & 0.997096 & 0.999983 \\ 
\hline
16 & 0.998755 & 0.999073 & 0.998893 & 0.997349 & 0.997204 & 0.998289 & 0.997219 & 0.996257 & 0.995812 & 0.997075 & 0.999983 \\ 
\hline
17 & 0.99875 & 0.99907 & 0.998886 & 0.997302 & 0.997175 & 0.998286 & 0.997214 & 0.996264 & 0.995768 & 0.997057 & 0.999983 \\ 
\hline
18 & 0.998746 & 0.999067 & 0.998879 & 0.997256 & 0.997147 & 0.998284 & 0.997208 & 0.996272 & 0.995726 & 0.997039 & 0.999983 \\ 
\hline
19 & 0.998742 & 0.999064 & 0.998873 & 0.99721 & 0.99712 & 0.998281 & 0.997202 & 0.996279 & 0.995686 & 0.997023 & 0.999982 \\ 
\hline
20 & 0.998738 & 0.999062 & 0.998867 & 0.997165 & 0.997093 & 0.998279 & 0.997196 & 0.996286 & 0.995647 & 0.997007 & 0.999982 \\ 
\hline
\end{tabular}
\label{tblavaimain:14}
\end{table}
\subsubsection{Question B2}
Calculate the total impacts due to links not being able to provide an
adequate LOS and the network over a 20 year period. (Assume that link failures
are mutually exclusive.)
\subsubsection{Answer B2}
The expected cost is estimated based on following equation 
\begin{eqnarray}
&& {C_{am}} = \frac{{E\left[ {t_i^n} \right] \cdot {I_{cr}}/12}}{{E\left[ {t_i^a}
\right] + E\left[ {t_i^n} \right]}}
\label{eqavaimain:12}
%(15)
\end{eqnarray}
The mean time between failures can be calculated with following
equation:
\begin{eqnarray}
&& E\left[ {t_i^a} \right] = {\Theta _i} = \int\limits_0^\infty  {{r_i}(t)dt} 
\label{eqavaimain:13}
%(15)
\end{eqnarray}
where ${r_i}(t)$ is the reliability of link \textit{i}.

Table \ref{tblavaimain:15} shows the summary of total cost for each link and the
total cost for the network

\begin{table}[h]
\caption{Cost}
\begin{tabular}{|c|c|c|c|c|c|c|c|c|c|c|c|}
\hline
Time & \multicolumn{10}{c|}{Links} & Total \\ 
\cline{2-11}
(years) & \multicolumn{5}{c|}{Road sections} & \multicolumn{3}{c|}{Bridge} & \multicolumn{2}{c|}{Tunnels} & costs \\ 
\cline{2-11}
 & A-1 & 2-3 & 3-4 & 3-5 & 4-5 & 1-2 & 1-3 & A-4 & B-2 & B-5 &  \\ 
\hline
Owner & 18.27 & 18.56 & 18.39 & 44.18 & 42.18 & 29.96 & 57.86 & 43.03 & 94.78 & 38.87 & 406.08 \\ 
\hline
User & 21.93 & 13.26 & 25.74 & 66.27 & 42.18 & 25.68 & 36.16 & 43.03 & 63.19 & 29.15 & 366.59 \\ 
\hline
DAP & 10.96 & 10.61 & 14.71 & 44.18 & 31.64 & 17.12 & 28.93 & 18.44 & 63.19 & 38.87 & 278.64 \\ 
\hline
IAP & 10.96 & 5.3 & 11.03 & 22.09 & 10.55 & 12.84 & 14.46 & 6.15 & 31.59 & 19.43 & 144.42 \\ 
\hline
\textbf{Total} & 62.13 & 47.74 & 69.87 & 176.72 & 126.54 & 85.6 & 137.41 & 110.64 & 252.75 & 126.32 & 1195.73 \\ 
\hline
\end{tabular}
\label{tblavaimain:15}
\end{table}

\bibliographystyle{plainnat}
\bibliography{reference}
% \newpage
\pagebreak

\begin{subappendices}
% \label{appendix3}
\section{Example - Reliability \& Availability}\label{appavai-mai1}
\lstinputlisting[basicstyle=\ttfamily\scriptsize]{./Programs/IMP-avai-main/IMP-E-availability.R}
\pagebreak
\section{Example - HongKong tunnel}\label{appavai-mai2}
\lstinputlisting[basicstyle=\ttfamily\scriptsize]{./Programs/IMP-avai-main/IMP-E-tunnel.R}
\pagebreak
\section{Assignment - Availability-log-normal distribution}\label{appavai-mai3}
\lstinputlisting[basicstyle=\ttfamily\scriptsize]{./Programs/IMP-avai-main/IMP-A-availability.r}
\pagebreak
\section{Assignment - availability, reliability, and impact estimation for the network}\label{appavai-mai3}
\lstinputlisting[basicstyle=\ttfamily\scriptsize]{./Programs/IMP-avai-main/IMP-A-availability-network.R}
\end{subappendices}